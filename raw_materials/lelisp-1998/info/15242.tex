\documentstyle{ilogmanuel}


\Begin
\Title {Diffusion Le-Lisp}
\SuperTitle {Le-Lisp  de  l'INRIA  version 15.24.2}
\bigskip
\SuperTitre {Version interm\'{e}daire de maintenance}

\Author {Ilog}
\Author {Ao\^{u}t 1991}

                 
                            
\chapter {Version "15.24.2"}
Voici les corrections et les extensions de la nouvelle diffusion de
\LeLisp\ version 15.24.2 dat\'{e}e du 5 Ao\^{u}t 1991.

Il s'agit de la deuxi\^{e}me diffusion de maintenance de la version
intitul\'{e}e 15.24.  Cette diffusion comprend uniquement des 
contournements et corrections effectu\'{e}s en Lisp et en C.  
Comme d\'{e}cid\'{e} au Club des Porteurs, aucune expansion LLM3 n'est n\'{e}cessaire.

\Section {Contenu de cette diffusion}


La bande de diffusion est explicit\'{e}e ci-dessous:

\begin {itemize}
\item   68k:
\item   benchmarks:
\item   common: 
\item   info:
\item   llib:           
\item   llmod:
\item   llobj:
\item   lltest:         
\item   llub:   
\item   manl:
\item   sun:
\item   vaxunix:
\item   virbitmap:
\item   virtty: 
\end {itemize}


\Section {Survol de 15.24.2}

La version 15.24.2 n'apporte de pas de grande nouveaut\'{e} mais
s'inscrit dans la continuit\'{e} du travail de normalisation de Le-Lisp
version 15.2 sur l'ensemble de ses portages, avec un maximun de traits
pr\'{e}sents sur chacun d'eux. La compatibilit\'{e} ascendante est totale et
n'impose que la refabrication des cores pour les logiciels utilisant
Le-Lisp.

Un certain nombres de choses ont toutefois
\'{e}t\'{e} revues ou corrig\'{e}es \`{a} l'occasion de cette version:
On peut citer:

\begin {enumerate}

\item Intervention d'ordre g\'{e}n\'{e}ral sur le code Le-Lisp:

\begin {itemize}

\item Un gros travail a \'{e}t\'{e} accompli sur le module des PATHNAMES,
aussi bien en correction d'anomalies qu'en ajout de s\'{e}mantique,
en particulier le probl\`{e}me de la longueur des noms de fichiers (sur
Sys5 et sur OS/2). On notera la cr\'{e}ation d'un outil de
d\'{e}tection de conflit de noms: {\tt lelisp/llub/verifpath.ll}

\item Changement de l'ordre des r\'{e}pertoires dans la variable
\#:SYSTEM:PATH:
le r\'{e}pertoire {\tt system/} est plac\'{e} devant les librairies.
Cela permet de mieux g\'{e}rer l'introduction de nouveaux portages en
cours d'ann\'{e}e, lesquels n\'{e}cessitent parfois une intervention locale dans
un fichier de la librairie.

\item Passage de {\bf \#:system:23-compatibility-flag} \`{a} NIL. Ce flag
introduit en 15.23.2 permettait aux anciens (RMARGIN 256) de simuler
l'effet de (RMARGIN (1+ (SLEN (OUTBUF)))). 

\item Reprise de certaines bizareries autour des longueurs des buffers
d'entr\'{e}e sortie (harmonisation LL/C, mise \`{a} niveau de stringio.ll).

\end {itemize}

\item Intervention sur la technologie de portage:

\begin {itemize}

\item Pr\'{e}sence unique du LOADER dans le r\'{e}pertoire de la machine.
Le nombre et la complexit\'{e} des portages allant en augmentant, il
apparait que chaque machine exige une ou plusieurs sp\'{e}cificit\'{e}s 
qui lui est
propre, qui la diff\'{e}rencie des autres machines utilisant le m\^{e}me
type de processeur.
exemples: big endian/little indian sur lapmips, introduction su 68040
sur lap68k, etc 

\item Am\'{e}nagement de la cha\^{\i}ne de production du binaire Le-Lisp
sur UNIX:
introduction du fichier {\tt lelispllm3.o} ne contenant strictement que
l'expansion LLM3, et garantissant ainsi la non-expansion LLM3.
Avec le temps, la complexit\'{e} des portages augmentant, 
un certains nombres de
choses sont venues par dessus l'expansion LLM3 (kern.o, plombage,
specificit\'{e}s C de certains portages etc) et sont int\'{e}gr\'{e}es dans le
fichier {\tt lelispbin.o} qui lui est fourni aux clients commes base
de recompilation du syst\`{e}me. C'est toujours le makefile {\tt
Makeport} qui fabrique {\tt lelispbin.o}, mais c'est maintenant un
makefile {\tt makellm3} qui, en amont du {\tt Makeport}, 
fabrique le fichier {\tt lelispllm3.o}. Ce
makefile n'est lanc\'{e} qu'une fois par an \`{a} l'occasion des versions
majeures. 

\item Homog\'{e}n\'{e}it\'{e} plus pouss\'{e}e sur les traits de portages tels que
CLOAD, 31BITFLOATS et 64BITFLOATS, scheduleur.

\item Encore plus de g\'{e}n\'{e}ration automatique: {\tt complice} et les
{\it configurateurs} d'images m\'{e}moire ({\tt conf/*conf.ll}) sont
engendr\'{e}s automatiquement. C'est pourquoi les noms configurateurs
commencent maintenant par une majuscule. Le {\tt Makefile} int\`{e}gre
cela. 

\item Introduction de la recette interpr\'{e}t\'{e}e.

\item Etant donn\'{e} le temps que mettait la recette du module {\it
KerN} \`{a} s'ex\'{e}cuter sur cetaines machines, les tests 13, 18 et 19 ont
\'{e}t\'{e} simplifi\'{e}s. Les temps sont divis\'{e}s de 4 \`{a} 8 fois, mais 
il est bon de savoir que ce sont autant de cas non test\'{e}s par
d\'{e}faut. Pour en am\'{e}liorer les performances de facon significative,
nous ne pouvons que conseiller aux porteurs d'impl\'{e}menter le
module KerN en assembleur sur leur(s) syst\`{e}me(s).

\end {itemize}

\item Intervention sur les codes C et assembleur :

\begin {itemize}

\item Pr\'{e}paration \`{a} l'int\'{e}gration du portage OS/2 (introduction de
code C en particulier). 

\item Traduction du noyau KerN en assembleur au format MOTOROLA sur
68K permettant l'introduction ais\'{e}e de KerN en assembleur sur la
majorit\'{e} des portages 68K. Nous gagnons ainsi un rapport 2 en vitesse
d'ex\'{e}cution des primitives de KerN sur ces machines.

\item Mise au point d'un certain nombre de flags s\'{e}mantiques de
compilation C sur les portages UNIX : 
LLPAGESIZE (taille des pages m\'{e}moire), LLTIMEUNIT
(fr\'{e}quence d'horloge), EXECORE (cores executables), CLOAD/NOCLOAD
(loader dynamique), ITINREAD (IT \`{a} la lecture), LLFOREGROUND (process
en foreground), LLVFORK (vfork ou non), LLRENAME (proc\'{e}dure rename
existe ou non). Et toujours: BSD4x et S5. Introduction du flag POSIX:
ce dernier devra prendre de plus en plus d'importance. \\
Cette manipulation des flags offre une plus grande lisibilit\'{e} du code
C,  permet une automatisation plus ais\'{e}e de la
g\'{e}n\'{e}ration de fichiers, mais surtout facilite l'introduction de
nouveaux portages.

\end {itemize}

\item Corrections propres \`{a} COMPLICE:

\begin {itemize}

\item
On change le nouveau syst\`{e}me de {\tt defvar} qui ne convenait pas
pour les {\tt (unless (boundp 'kop) (defvar kop ...))}. On red\'{e}fini
{\tt defvar} dans complice pour forcer les putprops. On parse quand m\^{e}me
pour pouvoir \'{e}tendre le {\bf cpenv}. Cela implique que si le
{\tt defvar} n'est pas en {\it ToplevelForm} alors le {\bf cpenv}
n'est pas \'{e}tendu. 

\item
Correction de la g\'{e}n\'{e}ration en "dur" de l'\'{e}tiquette 0 pour le
test des {\tt nsubrs} ({\tt :check-nsubr} dans {\tt cp2.ll}).
On ouvre les fonctions {\tt typevector} et {\tt typestring}.

\item
On traite les cas d\'{e}g\'{e}n\'{e}r\'{e}s de {\tt (and)} et {\tt (or)}.
On fait la sauvegarde les {\tt (cvalq x)} proteg\'{e}s pour chaque {\tt
call} ou {\tt jump}.
Correction d'un bug d'allocation de registres dans (arg <expr>).

\end{itemize}
\end{enumerate}

\Section {Les RATS}

Les changements de \LeLisp\ sont g\'{e}r\'{e}s avec des \|RATs|
(Requ\^{e}tes d'Action Technique).  Nous n'avons fourni que les sujets
des RATs par souci de place, mais vous pouvez avoir le texte complet
d'un rat d'ILOG en nous en communiquant le num\'{e}ro.  (Notez s'il vous
pla\^{\i}t que les num\'{e}ros des RATs sont partag\'{e}s par tous les produits
ILOG et pas uniquement \LeLisp.  Ne soyez pas effray\'{e} par les
num\'{e}ros \'{e}lev\'{e}s!). \\
Nous fournissons ici la liste de tous les rats corrig\'{e}s depuis
la derni\`{e}re version majeure de Le-Lisp (15.24).
\newpage
\begin{tabbing}
\= xxxxxxxxxxxxxxxxxxxxxxxxxxxxxx\= xxxxxxxxxxxxxxxxxxxxxxxxxxxxxx\= xxxxxxxxxxxxxxxxxxxxxxxxxxxxxx \= \kill
\verb_Rats for lelisp selected by status-fixed-or-declined as of 27/6/1991._\\ \verb_ _\\
\hspace{-5em}rat 512 \> area: {\it run-time
\ }\> aspect: {\it external-functions
\ }\\ 
\> status: {\it fixed
\ }\> origin: {\it kuczynsk@ilog
\ }\\ 
\verb_LISPCALL avec plusieurs flottants plante! (cf rat448)
_\\ 
\verb_ _\\ 
\hspace{-5em}rat 531 \> area: {\it language
\ }\> aspect: {\it arithmetic
\ }\\ 
\> status: {\it fixed
\ }\> origin: {\it eisenman@ilog
\ }\\ 
\verb_POWER sec omporte mal sur certaines machines avec des negatifs
_\\ 
\verb_ _\\ 
\hspace{-5em}rat 556 \> area: {\it compiler
\ }\> aspect: {\it complice
\ }\\ 
\> status: {\it fixed
\ }\> origin: {\it meller
\ }\\ 
\verb_compilation de maphash
_\\ 
\verb_ _\\ 
\hspace{-5em}rat 560 \> area: {\it compiler
\ }\> aspect: {\it complice
\ }\\ 
\> status: {\it fixed
\ }\> origin: {\it dufourd@elec.enst.fr
\ }\\ 
\verb_comportement diff. entre "complice foo" et "compilemodule foo"
_\\ 
\verb_ _\\ 
\hspace{-5em}rat 578 \> area: {\it run-time
\ }\> aspect: {\it external-functions
\ }\\ 
\> status: {\it fixed
\ }\> origin: {\it C. Julien [ACT]
\ }\\ 
\verb_encore un pb de ">" pour le compilo C sur ix386
_\\ 
\verb_ _\\ 
\hspace{-5em}rat 596 \> area: {\it run-time
\ }\> aspect: {\it external-functions
\ }\\ 
\> status: {\it fixed
\ }\> origin: {\it kuczynsk
\ }\\ 
\verb_(getglobal ">tcfloat") = (getglobal ">tcfloat3") sur RS6000
_\\ 
\verb_ _\\ 
\hspace{-5em}rat 625 \> area: {\it programming-environment
\ }\> aspect: {\it editors
\ }\\ 
\> status: {\it fixed
\ }\> origin: {\it kerjean
\ }\\ 
\verb_certaines commandes de edlin ne semblent pas repondre
_\\ 
\verb_ _\\ 
\hspace{-5em}rat 640 \\ 
\> status: {\it fixed
\ }\> origin: {\it le
\ }\\ 
\verb_erreur inadequate lors d'ouverture d'1 fichier inconnu sur RS6000
_\\ 
\verb_ _\\ 
\hspace{-5em}rat 644 \\ 
\> status: {\it fixed
\ }\> origin: {\it le
\ }\\ 
\verb_l'arg. numerique passe a lelisp reste sans effet sur la zone des CONS
_\\ 
\verb_ _\\ 
\hspace{-5em}rat 659 \> area: {\it programming-environment
\ }\> aspect: {\it trace
\ }\\ 
\> status: {\it fixed
\ }\> origin: {\it davis
\ }\\ 
\verb_Unilingual message in trace.
_\\ 
\verb_ _\\ 
\hspace{-5em}rat 666 \> area: {\it compiler
\ }\> aspect: {\it complice
\ }\\ 
\> status: {\it fixed
\ }\> origin: {\it J.F. Puget [Ilog]
\ }\\ 
\verb_mauvaise compilation des fonctions anonymes (LAMBDA)
_\\ 
\verb_ _\\ 
\hspace{-5em}rat 675 \> area: {\it compiler
\ }\> aspect: {\it complice
\ }\\ 
\> status: {\it fixed
\ }\> origin: {\it P. Kuczynski [Ilog]
\ }\\ 
\verb_le mode d'emploi des DEFVAR a chang\'{e} depuis la 15.24
_\\ 
\verb_ _\\ 
\hspace{-5em}rat 676 \> area: {\it language
\ }\> aspect: {\it other
\ }\\ 
\> status: {\it fixed
\ }\> origin: {\it J Chailloux [Ilog]
\ }\\ 
\verb_traduction systematique de toutes chaine francaise en anglais dans STARTUP.ll>>>>>>>>>>>>>>
_\\ 
\verb_ _\\ 
\hspace{-5em}rat 687 \> area: {\it i/o
\ }\> aspect: {\it virbitmap
\ }\\ 
\> status: {\it fixed
\ }\> origin: {\it Roland Ducournau  ducour@cerci2.uucp
\ }\\ 
\verb_Semantique de Clear-graph-env vis a vis de la couleur
_\\ 
\verb_ _\\ 
\hspace{-5em}rat 690 \\ 
\> status: {\it fixed
\ }\> origin: {\it le
\ }\\ 
\verb_il faut trait\'{e} le cas du "//" en debut de path pour Apollo
_\\ 
\verb_ _\\ 
\hspace{-5em}rat 698 \> area: {\it language
\ }\> aspect: {\it interpreter
\ }\\ 
\> status: {\it fixed
\ }\> origin: {\it grimm@medee.inria.fr
\ }\\ 
\verb_(CURRENT-DIRECTORY) sur la racine unix "/" rend un resultat inadequat
_\\ 
\verb_ _\\ 
\hspace{-5em}rat 699 \> area: {\it language
\ }\> aspect: {\it interpreter
\ }\\ 
\> status: {\it fixed
\ }\> origin: {\it grimm@medee.inria.fr
\ }\\ 
\verb_comportement non homogene de SEARCH-IN-PATH face a des types differents
_\\ 
\verb_ _\\ 
\hspace{-5em}rat 700 \> area: {\it i/o
\ }\> aspect: {\it virbitmap
\ }\\ 
\> status: {\it fixed
\ }\> origin: {\it ducour@cerci2.uucp
\ }\\ 
\verb_CURRENT-CLIP consomme 4 CONS
_\\ 
\verb_ _\\ 
\end{tabbing}

\newpage
\bigskip

\tableofcontents
\listoftables

\End
