\documentstyle{ilogmanuel}


\Begin
\Title {Diffusion Le-Lisp}
\SuperTitle {Le-Lisp  de  l'INRIA  version 15.23 \\
sous-version 2}

\Author {Septembre 1990}

                 
                            
\chapter {Version "15.23.2"}
Voici les corrections et les extensions de la nouvelle diffusion de
\LeLisp\ version 15.23 dat\'{e}e du 30 septembre 1990.

Il s'agit de la deuxi\^{e}me diffusion de maintenance de la version
15.23.  Cette diffusion comprend uniquement des 
contournements et corrections effectu\'{e}s en Lisp.  
Comme d\'{e}cid\'{e} au Club des Porteurs, aucune expansion LLM3 n'est n\'{e}cessaire.

\Section {Contenu de cette diffusion}

La bande de diffusion est explicit\'{e}e ci-dessous.

Les catalogues avec tous les fichiers (pour coh\'{e}rence):

\begin {itemize}
\item   benchmarks:
\item   bignum:
\item   common: 
\item   info:
\item   llib:           
\item   llmod:
\item   llobj:
\item   lltest:         
\item   llub:   
\item   manl:
\item   system:
\item   virbitmap:
\item   virtty: 
\end {itemize}

Les catalogues avec seulement les fichiers n\'{e}c\'{e}ssaires pour construire
des configurations :

\begin {itemize}
\item   68k:            
\item   ibmrt:
\item   lltool:
\item   sun:
\item   vax:
\end {itemize}

Les catalogues non inclus:

\begin {itemize}
\item   llm3:   aucun fichier
\item   manual: aucun fichier
\end {itemize}

\Section {Survol de 15.23 sous-version 2}

\begin {itemize}

\item Le nouveau BV dit \|BV 1+| (voir plus loin). \\
Ce nouveau BV est 
compatible ascendant au niveau du code source. La
d\'{e}finition de nouveaux champs dans certaines structures document\'{e}es,
oblige cependant la recompilation de tout code utilisant le BV, et cr\'{e}ant 
des sous-types des structures du BV;

\item Quelques nouveaut\'{e}s dans les modules 'messages' et 'path'
          (voir plus loin);

\item G\'{e}n\'{e}ralisation des {\tt images m\'{e}moire ex\'{e}cutables} 
sur un maximun de portages UNIX: \\
Les portages Sun et IBMRT offrent le choix \`{a} l'utilisateur 
de fabriquer des images ex\'{e}cutables ou non, le portage
Vax permet de fabriquer des images \LeLisp\ ex\'{e}cutables
en standard. Le code C aff\'{e}rant au mode \|EXECORE| est
d\'{e}sormais disponible dans un fichier \|execore.c|
rang\'{e} dans le r\'{e}pertoire de la machine, et non plus
dans \|lelisp.c|.
Voir plus loin pour plus de pr\'{e}cisions.

\item Mise \`{a} disposition d'une boucle d'\'{e}v\'{e}nements 
g\'{e}n\'{e}ralis\'{e}e (voir plus loin);

\item Disparition de l'instance X10.4 du BV. En effet toutes les
machines qui offraient X10, sont pass\'{e}es sur X11, et les nouvelles
implantations de X-Windows sont faites sous X11. Le BV X11 actuel supporte
X11r2, X11r3 et X11r4, ainsi que Dec-Windows, Open-Window et Motif.

\item Mise \`{a} niveau du {\tt multiple-getglobal} sur tous les portages.

\item Le nom de la fonction de plombage a chang\'{e}: elle se nomme
maintenant {\em tryaccess} et doit accepter un entier en argument. \\
rappel: l'\'{e}ventuel m\'{e}canisme de plombage est du ressort de chaque
porteur. La bande de distribution aux porteurs fournit une fonction
passante. 

\item Les {\tt Makefiles} ont \'{e}t\'{e} revus et homog\'{e}n\'{e}is\'{e}s entre les
diff\'{e}rents portages. Les tailles des images standard ont \'{e}t\'{e}
adapt\'{e}es; les syst\`{e}mes fournis en standard sont d\'{e}sormais \|lelispX11|, 
et \|cmplc|++; la commande \|complice| lance maintenant le 
core \|cmplc++| afin d'\'{e}viter les probl\`{e}mes de d\'{e}bordements de
zones lors de la compilation de modules \LeLisp\ cons\'{e}quents; dans la
m\^{e}me optique, la commande \|complice| accepte un argument
suppl\'{e}mentaire: {\tt -cons <n>} capable d'augmenter la zone des cellules.

\item Dans la m\^{e}me optique d'homog\'{e}n\'{e}isation des portages, les
recettes ont \'{e}t\'{e} revues, adapt\'{e}es et homog\'{e}n\'{e}is\'{e}es. Tous les
portages Le-Lisp fournis, poss\`{e}dent un r\'{e}pertoire {\tt recette/}
sous le r\'{e}pertoire de la machine, qui
contient des commandes de lancement de recettes. Parmi celles-ci on
notera la commande {\tt Suites} qui lance toutes les recettes (y
compris celle du BV). Les r\'{e}sultats peuvent \^{e}tre \'{e}tudi\'{e}s dans le
sous-r\'{e}pertoire {\tt recette/log/}. Toutes les erreurs sont
rep\'{e}rables par {\tt "**"} dans les fichiers de log.

\item Un nouveau r\'{e}pertoire sous lelisp: {\tt manl} qui contient les
pages des manuels Unix pour les commandes \|lelisp|, \|complice|,
\|config|, \|newdir|. Ces manuels sont au formal ``local{''}, c'est \`{a}
dire qu'ils peuvent \^{e}tre rang\'{e}s dans le r\'{e}pertoire /usr/man/manl
pour \^{e}tre utilis\'{e}s en standard avec la commande Unix {\tt man}.
\end {itemize}

\Section {Les RATS}

Les changements de \LeLisp\ sont g\'{e}r\'{e}s avec des \|RATs|
(Requ\^{e}tes d'Action Technique).  Nous n'avons fourni que les sujets
des RATs par souci de place, mais vous pouvez avoir le texte complet
d'un rat d'ILOG en nous en communiquant le num\'{e}ro.  (Notez s'il vous
pla\^{\i}t que les num\'{e}ros des RATs sont partag\'{e}s par tous les produits
ILOG et pas uniquement \LeLisp.  Ne soyez pas effray\'{e} par les
num\'{e}ros \'{e}lev\'{e}s!) \\

\newpage

\frenchspacing
\setlength{\parindent}{0in}
\setlength{\parskip}{3ex}
\raggedbottom
\begin{tabbing}
\= xxxxxxxxxxxxxxxxxxxxxxxxxxxxxx\= xxxxxxxxxxxxxxxxxxxxxxxxxxxxxx\= xxxxxxxxxxx
xxxxxxxxxxxxxxxxxxx \= \kill

\verb_Rats for le-lisp selected by status-fixed? as of 21/9/1990._\\ \verb_ _\\
\hspace{-5em}rat 42 \> area: {\it  i/o
\ }\> aspect: {\it  virbitmap
\ }\\ 
\> status: {\it  fixed
\ }\> origin: {\it  nuyens
\ }\\ 
\verb_les tests ne teste pas fill-rectangle, ni "character attributes"
_\\ 
\verb_ _\\ 
\hspace{-5em}rat 346 \> area: {\it  language
\ }\> aspect: {\it  other
\ }\\ 
\> status: {\it  fixed
\ }\> origin: {\it  P. Couronne [CISI]
\ }\\ 
\verb_lors d'un appel de C vers Lisp avec strings, pb de zone pleine.
_\\ 
\verb_ _\\ 
\hspace{-5em}rat 359 \> area: {\it  ports
\ }\> aspect: {\it  tests
\ }\\ 
\> status: {\it  fixed
\ }\> origin: {\it  Kuczynski[Ilog]
\ }\\ 
\verb_>Le breack nous ek nous ejecte en 2 coups sur HP9000!
_\\ 
\verb_ _\\ 
\hspace{-5em}rat 387 \> area: {\it  language
\ }\> aspect: {\it  other
\ }\\ 
\> status: {\it  fixed
\ }\> origin: {\it  A. Danan [Ilog]
\ }\\ 
\verb_Suppression des messages [REMOVE-LANGUAGE] encore difficile...
_\\ 
\verb_ _\\ 
\hspace{-5em}rat 394 \> area: {\it  language
\ }\> aspect: {\it  interpreter
\ }\\ 
\> status: {\it  fixed
\ }\> origin: {\it  Debiasy [Michelin]
\ }\\ 
\verb_EXPAND-PATHNAME ne fonctionne pas sur toutes les machines
_\\ 
\verb_ _\\ 
\hspace{-5em}rat 399 \> area: {\it  run-time
\ }\> aspect: {\it  o/s-interface
\ }\\ 
\> status: {\it  fixed
\ }\> origin: {\it  kuczynsk
\ }\\ 
\verb_CLOAD sur sun4
_\\ 
\verb_ _\\ 
\hspace{-5em}rat 418 \> area: {\it  i/o
\ }\> aspect: {\it  basic-i/o
\ }\\ 
\> status: {\it  fixed
\ }\> origin: {\it  Francis.Montagnac@mirsa.inria.fr
\ }\\ 
\verb_pb d'ouverture de fichier (OPENO) sur SUN4 et IBMRT
_\\ 
\verb_ _\\ 
\hspace{-5em}rat 432 \> area: {\it  run-time
\ }\> aspect: {\it  o/s-interface
\ }\\ 
\> status: {\it  fixed
\ }\> origin: {\it  E. Chailloux
\ }\\ 
\verb_le terminal est mal reinitialis\'{e} apres un save-core
_\\ 
\verb_ _\\ 
\hspace{-5em}rat 435 \> area: {\it  run-time
\ }\> aspect: {\it  external-functions
\ }\\ 
\> status: {\it  fixed
\ }\> origin: {\it  Macro NANNI [BULL]
\ }\\ 
\verb_la macro C LL>C>FLOAT marche-t-elle vraiment?
_\\ 
\verb_ _\\ 
\hspace{-5em}rat 440 \> area: {\it  i/o
\ }\> aspect: {\it  basic-i/o
\ }\\ 
\> status: {\it  fixed
\ }\> origin: {\it  berizzi
\ }\\ 
\verb_Mauvais message d'erreur dans path.
_\\ 
\verb_ _\\ 
\hspace{-5em}rat 446 \> area: {\it  compiler
\ }\> aspect: {\it  complice
\ }\\ 
\> status: {\it  fixed
\ }\> origin: {\it  berizzi
\ }\\ 
\verb_Des lignes trop longues dans la generation.
_\\ 
\verb_ _\\ 
\hspace{-5em}rat 448 \> area: {\it  run-time
\ }\> aspect: {\it  external-functions
\ }\\ 
\> status: {\it  fixed
\ }\> origin: {\it  kuczynsk
\ }\\ 
\verb_>>>LISPCALL avec plusieurs flottants plante sur Sun4
_\\ 
\verb_ _\\ 
\hspace{-5em}rat 460 \> area: {\it  i/o
\ }\> aspect: {\it  basic-i/o
\ }\\ 
\> status: {\it  fixed
\ }\> origin: {\it  berizzi
\ }\\ 
\verb_Bug dans renamefile sur macaux
_\\ 
\verb_ _\\ 
\hspace{-5em}rat 468 \> area: {\it  language
\ }\> aspect: {\it  other
\ }\\ 
\> status: {\it  fixed
\ }\> origin: {\it  Marco.Nanni@cediag.bull.fr
\ }\\ 
\verb_messages
_\\ 
\verb_ _\\ 
\hspace{-5em}rat 469 \> area: {\it  language
\ }\> aspect: {\it  other
\ }\\ 
\> status: {\it  fixed
\ }\> origin: {\it  Marco.Nanni@cediag.bull.fr
\ }\\ 
\verb_bug dans les benchmarks (do-check)
_\\ 
\verb_ _\\ 
\hspace{-5em}rat 474 \> area: {\it  language
\ }\> aspect: {\it  other
\ }\\ 
\> status: {\it  fixed
\ }\> origin: {\it  Marco.Nanni@cediag.bull.fr
\ }\\ 
\verb_paths virtuels
_\\ 
\verb_ _\\ 
\hspace{-5em}rat 477 \> area: {\it  language
\ }\> aspect: {\it  other
\ }\\ 
\> status: {\it  fixed
\ }\> origin: {\it  gallou@madeleine (Francoise Gallou)
\ }\\ 
\verb_option -v dans le lanceur
_\\ 
\verb_ _\\ 
\hspace{-5em}rat 481 \> area: {\it  language
\ }\> aspect: {\it  other
\ }\\ 
\> status: {\it  fixed
\ }\> origin: {\it  davis
\ }\\ 
\verb_read macro H for hash tables returns bad result
_\\ 
\verb_ _\\ 
\hspace{-5em}rat 485 \> area: {\it  programming-environment
\ }\> aspect: {\it  step
\ }\\ 
\> status: {\it  fixed
\ }\> origin: {\it  meller
\ }\\ 
\verb_trace du send
_\\ 
\verb_ _\\ 
\hspace{-5em}rat 487 \> area: {\it  language
\ }\> aspect: {\it  interpreter
\ }\\ 
\> status: {\it  fixed
\ }\> origin: {\it  meller
\ }\\ 
\verb_MACROEXPAND expanse trop (suite)...>
_\\ 
\verb_ _\\ 
\hspace{-5em}rat 491 \> area: {\it  i/o
\ }\> aspect: {\it  basic-i/o
\ }\\ 
\> status: {\it  fixed
\ }\> origin: {\it  berizzi
\ }\\ 
\verb_PROBEPATHM  et ses s\oe urs ne supportent pas les num\'{e}ros de version
_\\ 
\verb_ _\\ 
\hspace{-5em}rat 492 \> area: {\it  memory-management
\ }\\ 
\> status: {\it  fixed
\ }\> origin: {\it  berizzi
\ }\\ 
\verb_Perte de place dans les cores.
_\\ 
\verb_ _\\ 
\hspace{-5em}rat 494 \> area: {\it  ports
\ }\> aspect: {\it  loader
\ }\\ 
\> status: {\it  fixed
\ }\> origin: {\it  J. Grimm [INRIA]
\ }\\ 
\verb_confusion entre 0 et 0.0 par le loader du GOULD
_\\ 
\verb_ _\\ 
\hspace{-5em}rat 496 \> area: {\it  language
\ }\> aspect: {\it  interpreter
\ }\\ 
\> status: {\it  fixed
\ }\> origin: {\it  weinberg@ilog
\ }\\ 
\verb_QUOTE ou pas QUOTE en reprise errudv en mode debug
_\\ 
\verb_ _\\ 
\hspace{-5em}rat 497 \> area: {\it  other
\ }\\ 
\> status: {\it  fixed
\ }\> origin: {\it  berizzi@ilog
\ }\\ 
\verb_le nouveau MAKEMOD (15.23) n'est pas correct sur les path de llobj
_\\ 
\verb_ _\\ 
\hspace{-5em}rat 500 \> area: {\it  run-time
\ }\> aspect: {\it  external-functions
\ }\\ 
\> status: {\it  fixed
\ }\> origin: {\it  erhili@ilog
\ }\\ 
\verb_Doit-on tjrs mettre ">" devant les symboles de DEFEXTERN
_\\ 
\verb_ _\\ 
\hspace{-5em}rat 505 \> area: {\it  language
\ }\> aspect: {\it  extended-types
\ }\\ 
\> status: {\it  fixed
\ }\> origin: {\it  ducournau@semagroup
\ }\\ 
\verb_y a-t-il 2 SEND diff\'{e}rents selon que MICROCEYX soit charg\'{e} ou non
_\\ 
\verb_ _\\ 
\hspace{-5em}rat 507 \> area: {\it  ports
\ }\> aspect: {\it  loader
\ }\\ 
\> status: {\it  fixed
\ }\> origin: {\it  kuczynsk
\ }\\ 
\verb_Pourquoi ne peut-on pas faire une image sans loader compil\'{e}?
_\\ 
\verb_ _\\ 
\hspace{-5em}rat 509 \> area: {\it  language
\ }\> aspect: {\it  interpreter
\ }\\ 
\> status: {\it  fixed
\ }\> origin: {\it  chailloux@ilog
\ }\\ 
\verb_libell\'{e}s des dates en anglais
_\\ 
\verb_ _\\ 
\hspace{-5em}rat 515 \> area: {\it  run-time
\ }\> aspect: {\it  o/s-interface
\ }\\ 
\> status: {\it  fixed
\ }\> origin: {\it  ingenia
\ }\\ 
\verb_RENAMEFILE plante sur Sun4 OS4.1
_\\ 
\verb_ _\\ 
\end{tabbing}

\chapter {Le Bitmap Virtuel 1+}

Le {\em Bitmap Virtuel} 1+ est un sur-ensemble strict de l'ancienne version :
\begin {Itemize}
\item toutes les fonctions qui \'{e}taient document\'{e}es dans l'ancienne version
sont maintenues\,;
\item toutes les structures continuent \`{a} exister. Toutefois certaines
structures comme {\tt window}, {\tt display} et {\tt graph-env} ont \'{e}t\'{e}
\'{e}tendues d'o\`{u} la n\'{e}cessit\'{e} de recompiler 
toute application qui d\'{e}finit des sous-types de ces structures.
\end{Itemize}
Des changements ont \'{e}t\'{e} apport\'{e}s au niveau de :
\begin{Itemize}
\item la s\'{e}paration en plusieurs modules coh\'{e}rents correspondants
aux diff\'{e}rents types de fonctionnalit\'{e}s offertes par le bitmap virtuel
(dessin, gestion des \'{e}v\'{e}nements ...)\,;
\item l'ajout de nouvelles fonctionnalit\'{e}s\,.
\item l'incarnation du Bitmap Virtuel au dessus de X-window version 11 ;
\end{Itemize}
Ces changements ob\'{e}issent \`{a} deux motivations :
\begin {Itemize}
\item am\'{e}liorer les performances et la fiabilit\'{e} ;
\item pouvoir faire cohabiter le Bitmap Virtuel  avec des applications \'{e}crites
avec d'autres biblioth\`{e}ques graphiques et sp\'{e}cialement sur X11 : 
(biblioth\`{e}ques de {\tt widgets}, GKS, PHIGS, logiciels de cartographies, ...).
Cette cohabitation ne saurait \^{e}tre possible que si le mod\`{e}le du Bitmap
Virtuel est conforme au mod\`{e}le de ces biblioth\`{e}ques (Gestion des
\'{e}v\'{e}nements, fen\^{e}tres opaques,..).\\
\end{Itemize}





\Section  {Changements pour l'utilisateur}

Pour l'utilisateur, seules les nouvelles fonctionnalit\'{e}s sont
visibles.\\
Elles correspondent \`{a} de nouvelles fonctions et \`{a} l'ajout
de nouveaux champs aux structures.\\
A partir du BV1+, La version 10 de {\em X-window} n'est plus support\'{e}e.\\
La fonction {\tt bitprologue} \'{e}tait appel\'{e}e automatiquement lors du premier
appel de certaines fonctions de l'ancien BV. Ce n'est plus le cas dans le
BV1+. La fonction {\tt bitprologue} doit donc maintenant \^{e}tre ex\'{e}cut\'{e}e
avant toute op\'{e}ration graphique (dans le fichier {\tt .lelisp} par exemple).


\SSection {Structures}
Le fichier {\tt virstruct.ll} contient la d\'{e}finition de toutes les strutures
utilis\'{e}es par le BV.\\
Les structures qui ont \'{e}t\'{e} modifi\'{e}es sont les suivantes :
\BeginLL
(defstruct display
  name
  package
  device
  xmax
  ymax 
  eventmode
  prologuep
  keyboard-focus-window
  window
  graph-env
  root-window
  main-graph-env
  background
  foreground
  windows
  bitmaps
  menus
  colors         
  font-names     
  pattern-bitmaps
  cursor-bitmaps 
  extend
% New fields
  graph-envs        ;;; List of all graph env created by the user
                    ;;; except those associated to windows
                    ;;; Don't forget them in bitmap-save bitmap-restore
  current-selection ;;; To store current selection (a string) 
                    ;;; When the display is not an X11 display
                    ;;; Must be a globale variable for the multi-display
  resource          ;;; New field for resources (see Aida)
  named-cursors     ;;; list of the names of loaded named cursors.
  )

(defstruct graph-env 
  (font 0)
  (line-style 0)
  (pattern 1)
  (mode 3)
  foreground 
  background
  (clip-x 0)
  (clip-y 0)
  (clip-w 0)
  (clip-h 0)
  bitmap
  display
  extend
  font-y ;;; New field
  font-h ;;; New field
  )


;;; Information in field graphic-properties is not portable But is very
;;; useful for X11

(defstruct #:image:rectangle:window
  title
  hilited
  visible
  graph-env
  extend
  father
  properties
  (cursor 0)
  display
  subwindows
  ;; new fields :
  events-list     ;; (list of 'up-event 'down-event  'ascii-event  ...)
  window-type     ;; Window type : ('transparent or 'opaque) if () see display
  graphic-properties ;; list of symbols : backing-store, save-under,..
  state           ;; Symbol : 'iconify 'map 'unmap 'raise 'lower
  )


(defstruct font-info
  ascent
  descent
  angle
  weight
  minrbearing
  minlbearing
  minascent
  mindescent
  minwidth
  maxrbearing
  maxlbearing
  maxascent
  maxdescent
  maxwidth
  )
\EndLL




\SSection {Nouvelles fonctions}

Dans cette section, nous d\'{e}crivons les nouvelles fonctions ajout\'{e}es au Bitmap
Virtuel.\\


\SSSection {Module virinit}

Ce module contient les fonctions d'initialisation du Bitmap Virtuel ainsi que
des fonctions de bas niveau.

\Function {display-store-selection} {display buffer} {2}
Prend une cha\^{\i}ne de caract\`{e}res {\tt buffer} et la recopie dans un tampon
associ\'{e} au dispositif d'affichage {\tt display}.\\
Si le dispositif d'affichage le permet, ce {\em buffer} est partag\'{e} avec
toutes les applications autres que \LeLisp\ utilisant ce dispositif
d'affichage.\\


\Function {display-get-selection} {display} {1}
Cette fonction se charge de recup\'{e}rer une cha\^{\i}ne de
caract\`{e}res qui repr\'{e}sente la s\'{e}lection courante dans le dispositif
d'affichage {\tt display}.\\
Si le dispositif d'affichage offre cette caract\'{e}ristique (c'est le cas de
X11), cette fonction permet de r\'{e}cup\'{e}rer une cha\^{\i}ne de caract\`{e}res provenant
d'une autre application  utilisant le m\^{e}me dispositif d'affichage.


\Function {display-depth} {display} {1}
Cette fonction retourne le nombre de plans (bits/pixel) dans l'\'{e}cran
utilis\'{e} par le dispositif d'affichage {\tt display}.\\
Si le dispositif d'affichage est monochrome, elle retourne
donc 1.


\Function {bitmap-sync} {display} {0-1}
Cette fonction assure que toutes les requ\^{e}tes envoy\'{e}es au serveur ont \'{e}t\'{e}
ex\'{e}cut\'{e}es et que tous les \'{e}v\'{e}nements g\'{e}n\'{e}r\'{e}s par ces requ\^{e}tes
sont dans la file de \'{e}v\'{e}nements. \\
L'argument {\tt display} peut \^{e}tre omis, l'op\'{e}ration est effectu\'{e}e dans
le dispositif d'affichage courant.


\Function {display-synchronize} {display flag} {1-2}
Certains syst\`{e}mes comme X11 peuvent fonctionner en deux modes :
\begin {Itemize}
\item asynchrone : une action graphique peut ne pas \^{e}tre encore termin\'{e}e au
moment o\`{u} la fonction qui l'a d\'{e}clench\'{e}e rend la main (l'option par d\'{e}faut
sous X11),
\item synchrone : une action graphique est termin\'{e}e au moment o\`{u} la
fonction qui l'a d\'{e}clench\'{e}e rend la main.
\end {Itemize}
Cette fonction permet sur les syst\`{e}mes fonctionnant dans les deux modes
{\em synchrone} et {\em asynchrone}, de conna\^{\i}tre le mode courant ou de
passer de l'un \`{a} l'autre. Le mode synchrone correspond \`{a} une valeur vraie
de l'argument {\tt flag} et le mode asynchrone \`{a} une valeur fausse.\\
Cette fonction ne doit servir que pour le {\em debug}, et ne doit pas \^{e}tre
employ\'{e}e en utilisation normale.


\Function {standard-light-gray-pattern} {display} {0-1}
remplace la fonction {\tt standard-light-pattern} du manuel \LeLisp.

\Function {standard-medium-gray-pattern} {display} {0-1}
remplace la fonction {\tt standard-medium-pattern} du manuel \LeLisp.

\Function {standard-dark-gray-pattern} {display} {0-1}
remplace la fonction {\tt standard-dark-pattern} du manuel \LeLisp.



\SSSection {Module virdraw}

Ce module contient les op\'{e}rateurs de trac\'{e}s.


\Function {draw-arc} {x y w h angle1 angle2} {6}
Cette fonction dessine un arc de cercle ou d'ellipse dans la fen\^{e}tre
courante.\\
L'arc dessin\'{e} est d\'{e}fini par le rectangle centr\'{e} en {\tt x} et {\tt y}, 
de hauteur {\tt h} et de largeur {\tt w}. Seule la portion de l'ellipse
comprise entre les deux angles {\tt angle1} et {\tt angle2} (exprim\'{e}s en
degr\'{e}s dans le sens trigonom\'{e}trique avec l'origine \`{a} 3 heures) est
dessin\'{e}e.\\
Exemple, pour dessiner le demi-cercle de centre (10, 10) et de rayon 20 :
\BeginLL
(draw-arc 10 10 40 40 0 180)
\EndLL


\Function {fill-arc} {x y w h angle1 angle2} {6}
Cette fonction remplit une portion de camembert dans la fen\^{e}tre courante.\\
L'arc rempli est d\'{e}fini par le rectangle centr\'{e} en {\tt x} et {\tt y} 
de hauteur {\tt h} et de largeur {\tt w}. Seule la portion de l'ellipse
comprise entre les deux angles {\tt angle1} et {\tt angle2} (exprim\'{e}s en
degr\'{e}s dans le sens trigonom\'{e}trique avec l'origine \`{a} 3 heures).\\
Exemple, pour dessiner le demi-disque de centre (10, 10) et de rayon 20 :
\BeginLL
(fill-arc 10 10 20 20 0 180)
\EndLL


\Function {draw-segments} {n vx1 vy1 vbx2 vy2} {5}
Cette fonction dessine sur la fen\^{e}tre courante {\tt n} segments en un seul
appel.\\
Le {\tt i}-\`{e}me segment  est  d\'{e}fini par ses deux extr\'{e}mit\'{e}s
(vx[i], vy[i]) et (vx2[i], vy2[i]).\\
Les arguments {\tt vx1}, {\tt vy1}, {\tt vx2} et {\tt vy2} sont donc des
objets de type {\tt vector} comprenant au moins {\tt n} \'{e}l\'{e}ments.\\



\SSSection {Module virgraph}

Ce module contient les op\'{e}rations graphiques standards.


\Function {make-named-cursor} {name} {1}
Cette fonction charge un nouveau curseur dont le nom {\tt name} doit \^{e}tre
une cha\^{\i}ne de caract\`{e}res parmi la liste suivante :
\begin {Itemize}
\item "circle" : un cercle transparent,
\item "cross" : une intersection de routes,
\item "crosshair" : une croix fine,
\item "diamond-cross" : un losange constitu\'{e} de quatre triangles,
\item "dot" : un cercle plein,
\item "exchange" : une ic\^{o}ne symbolisant un \'{e}change entre deux entit\'{e}s,
\item "fleur" : une ic\^{o}ne symbolisant les quatre points cardinaux.
\item "left-hand" : une main pointant vers la gauche,
\item "heart" : un c\oe ur,
\item "iron-cross" : une croix de fer,
\item "left-ptr" : un pointeur en forme de fl\^{e}che,
\item "mouse" : une souris d'ordinateur,
\item "pencil" : un crayon \`{a} papier,
\item "pirate" : une t\^{e}te de mort,
\item "plus" : une croix \'{e}paisse,
\item "question" : un point d'interrogation,
\item "sizing" : une ic\^{o}ne symbolisant le redimensionnement,
\item "spray" : une bombe de peinture,
\item "watch" : une montre \`{a} aiguilles symbolisant l'attente,
\item "i-beam" : un curseur vertical pour le pointage dans du texte.
\end {Itemize}
Le nouveau curseur n'est ajout\'{e} \`{a} la liste des curseurs disponibles que s'il
n'existait pas d\'{e}j\`{a}.
Les curseurs nomm\'{e}s sont portables car similaires sur toutes les
incarnations du Bitmap Virtuel.


\Function {cursor-name} {num} {1}
Prend un num\'{e}ro de curseur (tel que le rendrait la fonction
{\tt current-cursor}) et retourne la cha\^{\i}ne de caract\`{e}res correspondant au
nom de ce curseur si c'est un curseur nomm\'{e}, () sinon.\\
Si ce num\'{e}ro ne correspond pas \`{a} un curseur charg\'{e}, une erreur est
d\'{e}clench\'{e}e.


\Function {make-line-style} {width style} {1}
Cr\'{e}e un nouveau style de dessin de ligne et retourne son num\'{e}ro.
Le nouveau style est d\'{e}fini par la largeur {\tt width} exprim\'{e}e en pixels
et le style {\tt  style} qui est un entier positif qui sera interpr\'{e}t\'{e}
par le type de dispositif d'affichage.\\
Par exemple, sur X11 il y a trois styles possibles\,: 0, 1 et 2.


\BeginHide

\Function {make-graph-env} {display font line-style pattern mode fg bg} {7}
Cette fonction cr\'{e}e un nouvel environnement graphique qui poss\`{e}de les
propri\'{e}t\'{e}s graphiques d\'{e}finies par\,:\\
\begin {Itemize}
\item {\tt font} : un num\'{e}ro d'une police de caract\`{e}res ;
\item {\tt line-style} : un num\'{e}ro de style de dessin de ligne ;
\item {\tt pattern} : un num\'{e}ro de motif de remplissage ;
\item {\tt mode} : un mode de combinaison valide ;
\item {\tt fg} et {\tt bg} : la couleur de fond et la couleur de dessin.
\end{Itemize}
Tous les arguments doivent \^{e}tres des objets d\'{e}j\`{a} cr\'{e}\'{e}s et valides.


\Function {kill-graph-env} {graph-env} {1}
Cette fonction lib\`{e}re  la structure {\tt graph-env}.\\
Plus aucune utilisation sur l'environnement graphique {\tt graph-env} n'est
possible.


\Function {current-graph-env} {graph-env} {0-1}
Si l'argument {\tt graph-env}  est fourni, cette fonction le positionne
comme environnement graphique courant pour toutes les op\'{e}rations de dessins.\\
S'il n'est pas fourni, la fonction retourne l'environnement  graphique
courant.\\
{\em Attention :} \\
Toute fen\^{e}tre \LeLisp\ poss\`{e}de son propre environnement graphique qu'elle
positionne chaque fois qu'elle devient la fen\^{e}tre courante.
Si l'on d\'{e}sire utiliser un autre environnement graphique, il faut le
positionner apr\`{e}s avoir positionn\'{e} la fen\^{e}tre courante.\\
Exemple :
\BeginLL
(current-window my-window)
(current-graph-env my-graph-env) ; effet limite a la fenetre my-window
\EndLL


\Function {graph-env-font} {graph-env font} {1-2}
Cette fonction permet de consulter ou de positionner la police de caract\`{e}res 
dans l'environnement graphique {\tt graph-env}.\\
Si l'argument {\tt font} est fourni, ce doit \^{e}tre le num\'{e}ro d'une police de
caract\`{e}res d\'{e}j\`{a} charg\'{e}e.


\Function {graph-env-mode} {graph-env mode} {1-2}
Cette fonction permet de consulter ou de positionner le mode de combinaison
utilis\'{e} par  l'environnement graphique {\tt graph-env}.\\
Si l'argument {\tt mode} est fourni, ce doit \^{e}tre le num\'{e}ro d'un mode de
combinaison valide.


\Function {graph-env-pattern} {graph-env pattern} {1-2}
Cette fonction permet de consulter ou de positionner le motif de remplissage
utilis\'{e} par  l'environnement graphique {\tt graph-env}.\\
Si l'argument {\tt pattern} est fourni, ce doit \^{e}tre le num\'{e}ro d'un motif de
remplissage d\'{e}j\`{a} cr\'{e}e.


\Function {graph-env-line-style} {graph-env line-style} {1-2}
Cette fonction permet de consulter ou de positionner le style de dessin de
ligne utilis\'{e} par  l'environnement graphique {\tt graph-env}.\\
Si l'argument {\tt line-style} est fourni, ce doit \^{e}tre le num\'{e}ro d'un style
de dessin de ligne d\'{e}j\`{a} d\'{e}fini.


\Function {graph-env-background} {graph-env color} {1-2}
Cette fonction permet de consulter ou de positionner la couleur de fond 
utilis\'{e} par  l'environnement graphique {\tt graph-env}.\\
Si l'argument {\tt color} est fourni, il doit \^{e}tre une instance valide de
la classe {\tt color}.


\Function {graph-env-foreground} {graph-env color} {1-2}
Cette fonction permet de consulter ou de positionner la couleur de dessin
utilis\'{e} par  l'environnement graphique {\tt graph-env}.\\
Si l'argument {\tt color} est fourni, il doit \^{e}tre une instance valide de
la classe {\tt color}.


\Function {graph-env-set-clip} {graph-env x y w h} {5}
Cette fonction permet de positionner la zone de {\em clipping}
utilis\'{e}e par  l'environnement graphique {\tt graph-env}.\\
Cette zone est d\'{e}finie par le rectangle ({\tt x}, {\tt y}, {\tt w} {\tt h}).


\Function {graph-env-set-clips} {graph-env n vx vy vw vh} {6}
Cette fonction permet de positionner les zones de {\em clipping}
utilis\'{e}es par  l'environnement graphique {\tt graph-env}.\\
Cette zone est d\'{e}finie par les {\tt n} rectangles ({\tt vx[i]}, {\tt vy[i]},
{\tt vw[i]}, {\tt vh[i]}).


\EndHide


\Function {display-get-font-names} {display maxnames pattern} {2-3}
Prend une cha\^{\i}ne de caract\`{e}res {\tt pattern} qui peut contenir les
caract\`{e}res {\tt *} (n'importe quelle suite de caract\`{e}res, m\^{e}me vide) et
{\tt ?} (n'importe quel caract\`{e}re) et retourne les noms des polices de
caract\`{e}res chargeables correspondant \`{a} cette expression.
L'argument {\tt pattern} est facultatif et vaut "*" par d\'{e}faut.
L'argument {\tt maxnames} permet de limiter le nombre des r\'{e}ponses,
car il peut y avoir un tr\`{e}s grand nombre de polices de caract\`{e}res dans le
syst\`{e}me.\\
Exemple d'utilisation :
\BeginLL
? (display-get-font-names (current-display) 100 "adobe*")
\EndLL
Recherche tous les noms de polices de caract\`{e}res
dont le nom commence par la cha\^{\i}ne {\tt "adobe"}.


\Function {display-get-font-info} {display font-name font-info} {2-3}
Cette fonction retourne une instance de la classe {\tt font-info} dont les
champs donnent des informations m\'{e}triques sur la police de caract\`{e}res de nom
{\tt font-name}.\\
Si l'argument {\tt font-info} est fourni, cette fonction n'alloue pas une
nouvelle instance de {\tt font-info} mais utilise l'argument {\tt font-info}
qui doit  \^{e}tre de type {\tt font-info}.\\
Cette fonction ne n\'{e}cessite pas que la police de caract\`{e}res de nom
{\tt font-name} soit d\'{e}j\`{a} charg\'{e}e.
La fonction peut par exemple \^{e}tre utile pour savoir si une police
de caract\`{e}res est de chasse fixe ou variable.\\
Cette fonction donne les informations m\'{e}triques :
\begin{Itemize}
\item g\'{e}n\'{e}rales sur la police (champs {\tt ascent} et {\tt descent})\,;
\item pour le plus petit caract\`{e}re (champs {\tt minlbearing},
{\tt minrbearing}, {\tt minascent}, {\tt mindescent}, et {\tt minwidth}\,;
\item pour le plus grand caract\`{e}re (champs {\tt maxlbearing},
{\tt maxrbearing}, {\tt maxascent}, {\tt maxdescent}, et {\tt maxwidth}\,;
\item un {\tt angle} pour l'inclinaison des polices italiques\,;
\item une \'{e}paisseur {\tt weight} pour les polices grasses\,;
\end{Itemize}
Le r\'{e}sultat d\'{e}pend du syst\`{e}me d'\'{e}cran et certains champs peuvent rester
nuls.\\
Si la police de caract\`{e}res de nom {\tt font-name} n'existe pas, la fonction
retourne {\tt ()}.



\SSSection {Module virwindow}

Ce module comprend toutes les fonctions ayant trait \`{a} la gestion des
fen\^{e}tres.


\Function {move-window} {window x y} {3}
Positionne la fen\^{e}tre {\tt window} au point de coordonn\'{e}es {\tt x, y}.


\Function {resize-window} {window w h} {3}
Redimensionne la fen\^{e}tre {\tt window} \`{a} largeur {\tt w} et la hauteur {\tt h}.


\Function {move-resize-window} {window x y w h} {5}
Cette fonction effectue en une seule op\'{e}ration un d\'{e}placement \`{a} la position
({\tt x}, {\tt y}) et un redimensionnement de la fen\^{e}tre {\tt window} avec
la largeur {\tt w} et la hauteur {\tt h}.


\Function {window-events-list} {window events} {1-2}
Cette fonction permet de consulter ou de positionner la liste des \'{e}v\'{e}nements
que la fen\^{e}tre {\tt window} d\'{e}sire traiter.\\
Quand l'argument {\tt events} est fourni, il doit \^{e}tre une liste
d'\'{e}v\'{e}nements corrects, c'est \`{a} dire de symboles
parmi\,: {\tt down-event, up-event, ascii-event,...}.


\Function {window-title} {window title} {1-2}
Cette fonction permet de consulter ou de positionner le titre 
de la fen\^{e}tre {\tt window}.\\
L'argument {\tt title}, s'il est fourni, est une cha\^{\i}ne de caract\`{e}res.\\
Seules les fen\^{e}tres principales peuvent avoir un titre, et non les
sous-fen\^{e}tres.
Cette fonction est sans effet avec certains dispositifs d'affichage.


\Function {window-state} {window state} {1-2}
Cette fonction permet de consulter ou de positionner l'\'{e}tat de la fen\^{e}tre 
{\tt window}.\\
L'argument {\tt state}, s'il est fourni est un symbole parmi :
{\tt map, unmap, lower, raise, iconify}.\\
Cette fonction permet par exemple d'iconifier une fen\^{e}tre et de la faire
r\'{e}appara\^{\i}tre.


\Function {window-background} {window color} {1-2}
Cette fonction permet de consulter ou de positionner la couleur de fond
de la fen\^{e}tre {\tt window}.\\
L'argument {\tt color}, s'il est fourni, doit \^{e}tre une instance valide du type
{\tt color}.\\
Cette fonction modifie aussi la couleur de fond de l'environnement graphique
associ\'{e} \`{a} la fen\^{e}tre.\\
{\em Remarque :}\\
Il n'est pas possible de changer la couleur de fond des sous-fen\^{e}tres dans
certains dispositifs d'affichage, mais ceci ne provoque pas d'erreur.


\Function {window-border} {window border} {1-2}
Cette fonction permet de positionner ou de consulter la taille 
(exprim\'{e}e en pixels) du bord de la fen\^{e}tre {\tt window}.


\Function {window-clear-region} {window x y w h} {5}
Cette fonction permet d'effacer le rectangle ({\tt x, y, w, h}) dans
la fen\^{e}tre {\tt window}.    


\BeginHide

\Function {window-graph-env} {window} {1}
Cette fonction retourne l'environnement graphique associ\'{e} \`{a} la fen\^{e}tre
{\tt window}.


\Function {graph-change-values} {ge font mode pattern line-style fg bg} {7}
Cette fonction permet de changer tous les attributs de l'environnement
graphique {\tt ge} en un seul appel.\\
Si l'on d\'{e}sire laisser un attribut inchang\'{e}, il faut lui passer
comme valeur {\tt ()}.


\Function {graph-env-subwindow-mode} {graph-env value} {2}
Cette fonction n'est utile que si le syst\`{e}me fournit des sous-fen\^{e}tres
qui disposent des m\^{e}me propri\'{e}t\'{e}s que les fen\^{e}tres principales comme
c'est le cas sur X11.\\
Cette fonction change l'attribut de l'environnement graphique {\tt graph-env}
qui sp\'{e}cifie si lorsqu'on dessine dans une fen\^{e}tre on d\'{e}sire dessiner 
aussi dans les sous-fen\^{e}tres de cette fen\^{e}tre.\\
L'argument {\tt value} est un entier qui peut prendre deux valeurs :
\begin{Itemize}
\item {\tt 0} (valeur par d\'{e}faut) : toute op\'{e}ration de dessin dans une 
fen\^{e}tre est limit\'{e}e \`{a} cette fen\^{e}tre et ne touche pas le contenu
des sous-fen\^{e}tres ;
\item {\tt 1} : dessiner dans une fen\^{e}tre peut alt\'{e}rer le contenu des
sous-fen\^{e}tres de celle-ci.
\end{Itemize}


\Function {window-change-attributes} {window props-set props-unset} {3}
Toute fen\^{e}tre \LeLisp\ poss\`{e}de un certain nombre de propri\'{e}t\'{e}s graphiques
dans son champ {\tt graphic-attributes}, sous forme d'une liste de symboles.\\
Ces propri\'{e}t\'{e}s ne sont pas garanties d'un syst\`{e}me \`{a} l'autre mais 
peuvent s'av\'{e}rer tres utiles dans certains syst\`{e}mes.\\
Par exemple, sur X11 on peut sp\'{e}cifier les propri\'{e}tes suivantes :
\begin {Itemize}
\item {\tt backing-store} : le syst\`{e}me de fen\^{e}trage sauvegarde dans ses
propres structures le contenu d'une fen\^{e}tre et en g\`{e}re lui-m\^{e}me le
r\'{e}affichage qui s'av\`{e}re plus rapide ;
\item {\tt save-under} : cette propri\'{e}t\'{e} est utile pour les fen\^{e}tres de
dialogue qui ont une dur\'{e}e de vie tres courte.
Elle permet de pr\'{e}server le contenu des fen\^{e}tres qui vont \^{e}tre
recouvertes par la fen\^{e}tre affich\'{e}e\,;
\item {\tt user-size} : cette propri\'{e}t\'{e} sert \`{a} sp\'{e}cifier que la taille
de la fen\^{e}tre sera indiqu\'{e}e interactivement par l'utilisateur lors de son
apparition ;
\item {\tt user-position} : cette propri\'{e}t\'{e} sert \`{a} sp\'{e}cifier que la
position de la fen\^{e}tre sera indiqu\'{e}e interactivement par l'utilisateur
lors de son apparition ;
\item {\tt override-redirect} : la fen\^{e}tre cr\'{e}\'{e}e sera totalement ignor\'{e}e
par le {\tt window-manager} dans X11 ;
\item {\tt iconic-state} : la fen\^{e}tre cr\'{e}\'{e}e va d\'{e}marrer son existence \`{a}
l'\'{e}tat d'ic\^{o}ne.
\end{Itemize}
La liste {\tt props-set} contient les propri\'{e}t\'{e}s \`{a} ajouter \`{a} la fen\^{e}tre
{\tt window} et la liste {\tt props-unset} les propri\'{e}t\'{e}s \`{a} enlever.\\
L'une ou l'autre de ces deux listes peut \^{e}tre vide.


\EndHide


\SSSection {Module virbit}

Ce module contient les fonctions de gestion des m\'{e}moires de points.


\Function {subst-colors} {bytetmap l} {2}
L'argument {\tt l} est une liste de la forme
{\tt ((old1 . new1) ... (oldn . newn))}
o\`{u} les {\tt oldi} et {\tt newi} sont des instances de la classe {\tt color}.
Cette fonction permet de substituer dans une m\'{e}moire de points {\tt bytemap}
tous les points de la couleur {\tt oldi} en la couleur {\tt newi}.\\
Cette fonction est sans effet sur une m\'{e}moire de points monochrome.


\SSSection {Module vircolor}

Ce module regroupe les fonctions de gestion de la couleur.


\Function {name-to-rgb} {name} {1}
Prend un nom de couleur {\tt name} et retourne les valeurs RGB
({\em red, green, blue}) correspondantes sous forme d'un vecteur de trois
entiers, ou () si cette couleur n'existe pas.


\Function {get-rgb-values} {pixel} {1}
Prend une valeur de pixel couleur {\tt pixel} (telle que le rendrait la
fonction {\tt byteref}) et retourne les valeurs RGB ({\em red, green, blue})
correspondantes sous forme d'un vecteur de trois entiers, ou () si cette
valeur de pixel ne correspond \`{a} aucune couleur.



\SSSection {Module virevent}

Ce module contient les fonctions de gestion des \'{e}v\'{e}nements.


\Function {allow-event} {display event} {2}
Cette fonction permet d'ajouter l'\'{e}v\'{e}nement {\tt event} \`{a} la liste des
\'{e}v\'{e}nements trait\'{e}s par le dispositif d'affichage {\tt display}.
Toutes les fen\^{e}tres qui seront cr\'{e}\'{e}es apr\`{e}s
l'appel \`{a} cette fonction vont demander au syst\`{e}me de recevoir
ce nouveau type d'\'{e}v\'{e}nement sauf si elles ont sp\'{e}cifi\'{e} leurs propres
listes d'\'{e}v\'{e}nements.
La fonction retourne l'\'{e}v\'{e}nement.\\
Exemple d'utilisation :
\BeginLL
(allow-event (current-display) 'functionkey-event)
\EndLL


\Function {disallow-event} {display event} {2}
Cette fonction permet d'enlever l'\'{e}v\'{e}nement {\tt event} de la liste des
\'{e}v\'{e}nements trait\'{e}s par le dispositif d'affichage {\tt display}.\\
Toutes les fen\^{e}tres qui seront cr\'{e}\'{e}es
apr\`{e}s l'appel de cette fonction vont demander au syst\`{e}me de ne pas recevoir
ce nouveau type d'\'{e}v\'{e}nements sauf si elles ont sp\'{e}cifi\'{e} leurs propres
listes d'\'{e}v\'{e}nement.
La fonction retourne l'\'{e}v\'{e}nement.\\
Exemple d'utilisation :
\BeginLL
(disallow-event (current-display) 'functionkey-event)
\EndLL


\Function {allowed-event-p} {display event} {2}
Ce pr\'{e}dicat indique si l'\'{e}v\'{e}nement {\tt event} fait partie des \'{e}v\'{e}nements
trait\'{e}s par le dispositif d'affichage {\tt display}, rend cet \'{e}v\'{e}nement
si c'est le cas, () sinon.



\SSection {Anciennes fonctions redocument\'{e}es}

\Function {inibitmap} {name} {0-1}
Charge le fichier de description du gestionnaire d'\'{e}cran de nom {\tt name}.
Si l'argument {\tt name} n'est pas  fourni, le nom est cherch\'{e} dans la
variable d'environnement BITMAP (voir la fonction GETENV).
Quand elle n'est pas d\'{e}finie, {\tt bvtty} est pris par d\'{e}faut.\\
\BeginLL
; Exemple : chargement du fichier de description pour X-window 11
? (inibitmap '|X11|)
= X11
\EndLL
Notez que l'on peut charger successivement plusieurs fichiers de description
diff\'{e}rents, ce qui peut \^{e}tre utile si l'on veut manipuler plusieurs
\'{e}crans.


\Function {current-clip} {x y w h} {fonction \`{a} 0 ou quatre arguments}
Le rectangle (x, y , w, h) devient le rectangle de d\'{e}coupe courant.
Sans argument, retourne les coordonn\'{e}es du rectangle de d\'{e}coupe courant
dans les variables \#:clip:x, \#:clip:y, \#:clip:w et \#:clip:h.
Toutes les op\'{e}rations d'affichage sont limit\'{e}es par le rectangle de
d\'{e}coupe : seules les op\'{e}rations graphiques comprises dans ce rectangle
seront donc r\'{e}alis\'{e}es.\\
Remarque : cette fonction n'est pas une variable-fonction. Elle ne doit donc
pas \^{e}tre employ\'{e}e \`{a} l'int\'{e}rieur d'un {\tt with}.




\Section {Manuel d'implantation}

En compl\'{e}ment de la documentation du chapitre pr\'{e}c\'{e}dent et de celle
du Manuel de r\'{e}f\'{e}rence \LeLisp\, voici la liste des m\'{e}thodes \`{a} impl\'{e}menter
pour disposer d'un portage du Bitmap Virtuel dans un nouveau syst\`{e}me
d'\'{e}cran.\\

Les fonctions sont s\'{e}par\'{e}es en modules interd\'{e}pendants.\\
Les m\'{e}thodes qui existaient d\'{e}j\`{a} dans l'ancien BV peuvent voir changer
le nombre et le type de leurs arguments.


\SSection {M\'{e}thodes du module virutil}
\BeginLL
(send 'current-display display)
\EndLL


\SSection {M\'{e}thodes du module virinit}

\BeginLL
(send 'bitprologue display)
(send 'bitmap-save display)
(send 'bitmap-restore display)
(send 'bitepilogue display)
(send 'bitmap-refresh display)
(send 'bitmap-sync display)
(send 'bitmap-flush display)
(send 'standard-roman-font display)
(send 'standard-bold-font display)
(send 'large-roman-font display)
(send 'small-roman-font display)
(send 'standard-background-pattern display)
(send 'standard-foreground-pattern display)
(send 'standard-light-gray-pattern display)
(send 'standard-medium-gray-pattern display)
(send 'standard-dark-gray-pattern display)
(send 'standard-lelisp-cursor display)
(send 'standard-gc-cursor display)
(send 'standard-busy-cursor display)
(send 'synchronize display flag)

;; Nouveau :
(send 'store-selection display buffer)
(send 'get-selection display)
(send 'display-depth display)
\EndLL


\SSection {M\'{e}thodes du module virevent}
\BeginLL
(send 'event-mode display event-mode)
(send 'eventp display)
(send 'read-event display event)
(send 'peek-event display event)
(send 'flush-event display)
(send 'add-event display event)
(send 'grab-event display window)
(send 'ungrab-event display window)
(send 'itsoft-event display)
(send 'read-mouse display event)

;; Nouveau :
(send 'allow-event display event)
(send 'disallow-event display event)
(send 'allowed-event-p display event)
\EndLL


\SSection {M\'{e}thodes du module virmenu}

\BeginLL
(send 'create-menu display menu)
(send 'kill-menu display menu)
(send 'activate-menu display menu x y)
(send 'menu-insert-item display menu choix index name active value)
(send 'menu-insert-item-list display menu choix name active)
(send 'menu-delete-item display menu choix index)
(send 'menu-delete-item-list display menu choix)
(send 'menu-modify-item display menu choix index name active value)
(send 'menu-modify-item-list display menu choix name active)
\EndLL


\SSection {M\'{e}thodes du module vircolor}

\BeginLL
(send 'make-color display color red green blue)
(send 'make-mutable-color display color red green blue)
(send 'make-named-color display color name)
(send 'kill-color display color)
(send 'red-component display color red)
(send 'blue-component display color blue)
(send 'green-component display color green)

;; Nouveau :
(send 'name-to-rgb display name rgb)
(send 'get-rgb-values display pixel rgb)
\EndLL


\SSection {M\'{e}thodes du module virbit}

\BeginLL
(send 'create-bitmap display bitmap)
(send 'create-window-bitmap display window bitmap)
(send 'kill-bitmap display bitmap)
(send 'bmref display bitmap x y)
(send 'bmset display bitmap x y bit)
(send 'byteref display bitmap x y)
(send 'byteset display bitmap x y byte)
(send 'bitblit display b1 b2 x1 y1 x2 y2 w h)
(send 'get-bit-line display bitmap y bitvector)
(send 'set-bit-line display bitmap y bitvector)
(send 'create-bytemap display bitmap)
(send 'get-byte-line display bitmap y bytevector)
(send 'set-byte-line display bitmap y bytevector)

;; Nouveau :
(send 'subst-colors display bitmap l ) ;;; l : (list of (oldcolor . newcolor))
\EndLL


\SSection {M\'{e}thodes du module virdraw}
\BeginLL

(send 'draw-cursor display graph-env x y state)
(send 'draw-substring display graph-env x y s start length)
(send 'draw-cn display graph-env x y cn)
(send 'draw-point display graph-env x y)
(send 'draw-polymarker display graph-env n vx vy)
(send 'draw-line display graph-env x0 y0 x1 y1)
(send 'draw-polyline display graph-env n vx vy)
(send 'draw-rectangle display graph-env x y w h)
(send 'fill-area display graph-env n vx vy)
(send 'fill-rectangle display graph-env x y w h)
(send 'draw-ellipse display graph-env x y rx ry)
(send 'draw-circle display graph-env x y r)
(send 'fill-ellipse display graph-env x y rx ry)
(send 'fill-circle display graph-env x y r)

;; Nouveau :
(send 'draw-segments display graph-env n vx1 vy1 vx2 vy2)
(send 'draw-arc display graph-env x y w h angle1 angle2)
(send 'fill-arc display graph-env x y w h angle1 angle2)
\EndLL


\SSection {M\'{e}thodes du module virgraph}

\BeginLL
(send 'current-font display graph-env font)
(send 'current-mode display graph-env mode)
(send 'current-pattern display graph-env pattern)
(send 'current-line-style display graph-env line-style)
(send 'current-foreground display graph-env color)
(send 'current-background display graph-env color)

(send 'clear-graph-env display graph-env)
(send 'font-height display graph-env)
(send 'font-ascent display graph-env)

(send 'font-max display)
(send 'line-style-max display)
(send 'pattern-max display)
(send 'cursor-max display)

(send 'width-substring  display ge string start length)
(send 'height-substring display ge string start length)
(send 'x-base-substring display ge string start length)
(send 'y-base-substring display ge string start length)
(send 'x-inc-substring  display ge string start length)
(send 'y-inc-substring  display ge string start length)

(send 'current-clip display graph-env x y w h)
(send 'load-font display font)
(send 'make-pattern display bitmap)
(send 'make-cursor display b1 b2 x y)
(send 'current-cursor display cursor)
(send 'move-cursor display x y)

;; Nouveau :
\EndLL
\BeginHide
(send 'graph-env-font display graph-env font)
(send 'graph-env-mode display graph-env mode)
(send 'graph-env-pattern display graph-env pattern)
(send 'graph-env-line-style display graph-env line-style)
(send 'graph-env-foreground display graph-env color)
(send 'graph-env-background display graph-env color)
(send 'make-graph-env display graph-env . window)
(send 'kill-graph-env display graph-env)
(send 'graph-env-set-clip display graph-env x y w h)
(send 'graph-env-set-clips display graph-env n vx vy vw vh)
(send 'current-graph-env display graph-env)
(send 'graph-env-change-values display graph-env font mode pattern ls fg bg)
(send 'graph-env-subwindow-mode display graph-env mode)
\EndHide
\BeginLL
(send 'make-named-cursor display key) ; key : index dans #:display:cursor-names
(send 'cursor-name display cursor)
(send 'make-line-style display width style)

(send 'get-font-names display maxnames pattern)
(send 'get-font-info display fontname font-info)

\EndLL


\SSection {M\'{e}thodes du module virwindow}
\BeginLL

(send 'current-window display window)
(send 'uncurrent-window display window)
(send 'find-window display x y)
(send 'create-window display window)
(send 'create-subwindow display window)
(send 'modify-window display window le to wi he ti hi vi)
(send 'update-window display window le to wi he)
(send 'kill-window display window)
(send 'pop-window display window)
(send 'move-behind-window display window1 window2)
(send 'current-keyboard-focus-window display window)
(send 'uncurrent-keyboard-focus-window display window)
(send 'map-window display window x y lx ly)

;; Nouveau :
(send 'window-clear-region display window x y w h)      
(send 'move-window display window le to)        
(send 'resize-window display window wi he)
(send 'move-resize-window display window le to wi he)
(send 'window-events-list display window events)
(send 'window-title display window title)
(send 'window-state display window state)
(send 'window-background display window color)
(send 'window-border display window border)
(send 'window-change-attributes display window props-set props-unset)
\EndLL




\SSection {Notes d'impl\'{e}mentation}

Ce paragraphe traite des nouveaut\'{e}s au niveau de l'impl\'{e}mentation du BV1+ et
de sa mise en \oe uvre sur X11.

\SSSection {Les fen\^{e}tres}

Le gestionnaire de fen\^{e}tres X11 propose deux types de sous-fen\^{e}tres :
\begin {Itemize}
\item {\tt InputOutput} : des sous-fen\^{e}tres qui poss\`{e}dent les m\^{e}mes
propri\'{e}t\'{e}s que les fen\^{e}tres principales ;
\item {\tt InputOnly} : des fen\^{e}tres qui ne servent que comme zones
rectangulaires sensibles aux \'{e}v\'{e}nements et dans lesquelles on ne peut
en aucun cas dessiner.\\
L'utilisation que l'on peut faire de ces fen\^{e}tres et de dessiner dans la
fen\^{e}tre  principale et d'utiliser la sous-fen\^{e}tre pour recevoir les
\'{e}v\'{e}nements.
\end {Itemize}
Nous d\'{e}signerons ces deux types de fen\^{e}tres sous les d\'{e}nominations
{\tt opaque} et {\tt transparent}.\\
Le nouveau BV supporte les deux types de fen\^{e}tres. La nouvelle structure de
fen\^{e}tre contient un champ {\tt window-type} qui doit toujours contenir
le type de la fen\^{e}tre.\\
En effet, comme ces deux types de fen\^{e}tres r\'{e}agissent diff\'{e}rement pour les
op\'{e}rations de dessins et r\'{e}affichage, l'utilisateur doit \^{e}tre inform\'{e} sur
le type de fen\^{e}tre qu'il utilise.\\
La diff\'{e}rence principale est que les fen\^{e}tres opaques re\c{c}oivent les
\'{e}v\'{e}nements de type r\'{e}affichage alors que les fen\^{e}tres transparentes ne
peuvent pas, et donc chaque fen\^{e}tre ne doit \^{e}tre responsable que du dessin
explicitement fait dans cette fen\^{e}tre.
Le dessin dans les sous-fen\^{e}tres opaques ne peut pas alt\'{e}rer le contenu de
la fen\^{e}tre p\`{e}re.
L'implantion du BV doit obligatoirement remplir la valeur du champ
{\tt window-type} de chaque nouvelle instance de la classe {\tt window}.


\SSSection {\'{e}v\'{e}nements}

La gestion des \'{e}v\'{e}nements est maintenant partag\'{e}e entre la structure
{\tt display} et la structure {\tt window}.\\
Ces deux structures poss\`{e}dent un champ {\tt events-list} qui d\'{e}signe
l'ensemble des \'{e}v\'{e}nements que l'on d\'{e}sire traiter.\\
De nouveaux types d'\'{e}v\'{e}nements ont \'{e}t\'{e} rajout\'{e}s :
\begin {Itemize}
\item {\tt map-window} : la fen\^{e}tre qui re\c{c}oit cet \'{e}v\'{e}nement est maintenant
visible et est plac\'{e}e au dessus de toutes les autres fen\^{e}tres\,;
\item {\tt unmap-window} : la fen\^{e}tre dispara\^{\i}t de l'\'{e}cran (exemple :
iconifier une fen\^{e}tre)\,;
\item {\tt visibility-change} : la visibilit\'{e} de la fen\^{e}tre est alt\'{e}r\'{e}e,
le champ {\tt detail} de l'\'{e}v\'{e}nement est un entier positif ayant
comme valeurs possibles :
\begin {Itemize}
\item 0 : totalemement visible,
\item 1 : partiellement visible,
\item 2 : totalemement invisible,
\end {Itemize}
Cet \'{e}v\'{e}nement est utile si l'on veut par exemple maintenir une fen\^{e}tre au
premier plan en permanence.
\item {\tt client-message} : \'{e}v\'{e}nement de type communication dont le champ
{\tt detail} contient le message envoy\'{e} si c'est une cha\^{\i}ne de caract\`{e}res.
\end {Itemize}


\SSSection {Communication avec le Window Manager}

L'impl\'{e}mentation du BV1+ respecte le protocole X11 : {\em Inter Client
Communication Control Protocol} (ICCCM). Cela permet \`{a}
\LeLisp\ de cohabiter avec les autres applications graphiques pr\'{e}sentes,
comme le {\em Window Manager} qui sous X11 est un client du serveur.\\
D'autre part, il est possible de  modifier l'\'{e}tat d'une fen\^{e}tre \,:
iconifier, faire dispara\^{\i}tre la fen\^{e}tre par le biais de la communication
avec le {\em window-manager}.\\
Ainsi le champ {\tt state} d'une fen\^{e}tre indiquera toujours l'\'{e}tat d'une
fen\^{e}tre : iconify, map ou unmap.\\
Sur X11, le Bitmap Virtuel ne fonctionne correctement en pr\'{e}sence d'un
{\em window-manager} que si celui-ci respecte le protocole.



\SSection {Les tests automatiques}

\SSSection {Principe}

Le Bitmap Virtuel poss\`{e}de maintenant une routine de tests automatiques comme
les autres modules de \LeLisp.\\
Ces tests permettent de v\'{e}rifier que les fonctions ont bien un comportement
correspondant \`{a}  ce qu'on attend d'elles.\\
Ils permettent aussi de tester certains aspects visuels automatiquement en
{\em photographiant} une fen\^{e}tre \`{a} un moment donn\'{e} et en comparant son
contenu avec un mod\`{e}le.\\
La m\'{e}thode effectivement retenue pour ces tests est plus \'{e}conomique car elle
permet d'\'{e}viter de charger de tels mod\`{e}les, qui seraient des {\em bitmaps}
parfois volumineux.
On prend deux fen\^{e}tres, et on effectue dans chacune de ces
fen\^{e}tres des trac\'{e}s qui doivent aboutir par des voies diff\'{e}rentes \`{a} un
r\'{e}sultat identique.\\
Par exemple, on tracera un rectangle avec {\tt draw-rectangle} dans la
premi\`{e}re, et le rectangle identique avec quatre {\tt draw-line} dans la
seconde, et finalement on s'assure que les contenus des deux fen\^{e}tres sont
bien identiques en comparant point par point leurs {\em bitmaps} respectifs.\\

\SSSection {Organisation des fichiers sous Unix}

Les fichiers de tests du Bitmap Virtuel sont dans le r\'{e}pertoire {\tt lltest}
de \LeLisp.\\
Pour lancer le test, lancer un \LeLisp\ graphique et taper :
\BeginLL
? ^Ltestbv
\EndLL
Le test s'ex\'{e}cute alors automatiquement et prend quelques minutes (pendant
lesquelles il ne faut toucher \`{a} rien, afin de ne pas influer
sur le d\'{e}roulement des tests).\\
Le fichier {\tt testbv} provoque le chargement de plusieurs fichiers de tests
(appel\'{e}s avec la fonction {\tt testfn}) :
\begin {Itemize}
\item un fichier de tests g\'{e}n\'{e}ral : {\tt tbvdata.ll},
\item un fichier sp\'{e}cifique couleur/monochrome :
\begin {Itemize}
\item sur un \'{e}cran monochrome : le fichier {\tt tbvnb.ll}
\item sur un \'{e}cran couleur ou niveaux de gris : le fichier {\tt tbvcolor.ll}
\end {Itemize}
\item un fichier "non portable" pour des tests sp\'{e}cifiques \`{a} un syst\`{e}me
d'\'{e}cran :
\begin {Itemize}
\item pour X11 : le fichier tbvX11.ll,
\item pour un syst\`{e}me d'\'{e}cran qui s'appellerait {\tt zz69} : le fichier
serait {\tt tbvzz69.ll}.
\end {Itemize}
Le nom de ce fichier est une concat\'{e}nation de la cha\^{\i}ne "tbv" et du nom du
{\tt bitmap} (la variable {\tt \#:bitmap:name}).
\end {Itemize}
Le seul fichier de test sp\'{e}cifique \`{a} l'heure actuelle concerne X11.


\SSSection {Les tests sur X11}
L'utilisation des tests sous Unix est int\'{e}ressante \`{a} conna\^{\i}tre pour qui
d\'{e}sire porter la routine de test sur son propre syst\`{e}me.\\
Sur l'incarnation X11 sous Unix, les tests peuvent \^{e}tre lanc\'{e}s gr\^{a}ce \`{a} une
commande {\tt Bv.rec} dans le r\'{e}pertoire {\tt recette} de la machine.\\
Cette commande est un {\em shell-script} qui :
\begin {Itemize}
\item Cr\'{e}e des fichiers d'environnement sp\'{e}cifiques (.Xdefaults, .lelisp)
afin d'isoler l'application des options propres \`{a} la personne
qui effectue le test,
\item Lance un \LeLisp\ graphique {\tt lelispX11},
\item charge {\tt testbv.ll}
\end {Itemize}

Ce test est lanc\'{e} avec l'ensemple de la recette \LeLisp. Les fichiers
r\'{e}sultats \`{a} consulter sont dans le r\'{e}pertoire {tt recette/log/}.
Cela permet de lancer le test sur plusieurs machines, donc dans des contextes
diff\'{e}rents, et de pouvoir conserver trace de tous les r\'{e}sultats.\\
Le fichier r\'{e}sultat concernant le BV
contient en t\^{e}te quelques informations quant \`{a} la
personne qui a lanc\'{e} le test, le contenu de la variable d'environnement
{\tt DISPLAY}, etc.\\


\SSSection {\'{e}volution des tests}
Ces tests du BV peuvent et doivent \^{e}tre affin\'{e}s et compl\'{e}t\'{e}s.\\
Si un des tests ne fonctionne pas, il est possible qu'il faille le raffiner.
Si c'est le test qui est faux dans le contexte du nouveau portage, on peut
l'\'{e}liminer \`{a} condition d'apposer un commentaire pour expliquer pourquoi
ce test a \'{e}t\'{e} \'{e}limin\'{e}.\\
Les modifications et ajouts fournis par les porteurs seront incorpor\'{e}s dans
les prochaines versions des tests du BV, et devront permettre d'apporter une
solution fiable et syst\'{e}matique \`{a} la validation du Bitmap Virtuel.



\SSSection {La mesure des performances}

Le fichier {\tt bvbench.ll} contient un programme de mesure des performances
du bitmap virtuel.\\
Il suffit de charger ce fichier et de taper :
\BeginLL
? (bench "/tmp/monbench")
\EndLL
L'ex\'{e}cution de cette routine prend plusieurs minutes et le r\'{e}sultat est
sauvegard\'{e} dans le fichier {\tt /tmp/monbench}.


\chapter {Les messages multilingues}

\Section {Anciennes fonctions modifi\'{e}es dans le paragraphe 1.1}

\Function {remove-language} {lang} {SUBR \`{a} 1 argument}
ajouter:
Si la langue \LT lang\GT\ est une des langues d\'{e}finies d'origine
({\em french} et {\em english}), {\tt remove-language} efface tous les
messages associ\'{e}s, mais conserve le nom de la langue dans la base des
langages autoris\'{e}s. Cela permet de continuer de pouvoir charger des
librairies standard o\`{u} des messages sont d\'{e}finis dans ces langues
d'origine, m\^{e}me apr\`{e}s un {\tt remove-language}. 


\Section {Nouvelles fonctions pour le paragraphe 1.1}

\Function {default-language} {lang} {SUBR \`{a} 0 ou 1 argument}
Une nouvelle notion est introduite par cette fonction: le langage par
d\'{e}faut. L'id\'{e}e est que l'utilisateur qui d\'{e}sire se d\'{e}finir une
nouvelle langue, ne soit pas contraint de red\'{e}finir tous les messages
pr\'{e}-existants du syst\`{e}me dans cette nouvelle langue. {\tt
default-language} lui permet de choisir la langue qui sera utilis\'{e}e
pour tous les messages n'ayant pas de valeur dans la langue courante
(cf {\tt current-language}). \\
Sans argument, cette fonction retourne le nom symbolique de la langue
par d\'{e}faut. Avec un argument (de type langue qui a \'{e}t\'{e} enregistr\'{e}
au moyen de la fonction {\tt record-language}), elle change la valeur
de la langue par d\'{e}faut.\\
{\tt default-language} peut d\'{e}clencher les erreurs {\tt
error-wrong-language} et  {\tt error-not-recorded-language}.\\
A l'initialisation, le langage par d\'{e}faut est le m\^{e}me que le langage
courant. \\

Exemples:\\
\? {(current-language)}
\= {french}
\? {(default-language)}
\= {french}
\? {(get-message 'errwna)}
\= {mauvais nombre d'arguments}
\? {(record-language 'german)}
\= {german}
\? {(defmessage exemple (german "das ist ein example"))}
\= {exemple}
\? {(current-language 'german)}
\= {german}
\? {(get-message 'exemple)}
\= {das ist ein example}
\? {(get-message 'errwna)}
\= {mauvais nombre d'arguments}
\? {(default-language 'english)}
\= {english}
\? {(get-message 'exemple)}
\= {das ist ein example}
\? {(get-message 'errwna)}
\= {wrong number of arguments}
\? {(get-message-p 'example) ; GET-MESSAGE-P suit aussi de DEFAULT-LANGUAGE}
\= {das ist ein example}


\chapter {Les pathnames virtuels}

\Section {Anciennes fonctions modifi\'{e}es dans le paragraphe 6.10.1}

\Function {namestring} {path} {SUBR \`{a} 1 argument}
Ajouter : l'argument peut \^{e}tre de type {\tt pathname}, {\tt string} 
ou {\tt symbol}.

\Function {pathname} {x} {SUBR \`{a} 1 argument}
convertit son argument \LT\ x\GT\ (de type {\tt string}, {\tt symbol} ou
{\tt pathname}) correspondant
\`{a} un nom de fichier dans la syntaxe du syst\`{e}me d'exploitation courant en
l'objet de type {\tt pathname} correspondant.

Ajouter l'exemple :\\
\? {(pathname \#u"/usr/ilog/")}
\= {\#:pathname:\#[ () () ("usr" "ilog") () () () ]}




\Section {Nouvelles fonctions pour le paragraphe 6.10.1}

\Function {equal-pathname} {path1 path2} {SUBR \`{a} 2 arguments}
Cette fonction rend {\tt T} si ses deux arguments sont des pathnames \'{e}quivalents.\\
Exemples sur Unix :\\
\? {(equal-pathname \#p"/a/./b" \#p"/a/b")}
\= {t}
\? {(equal-pathname \#p"/a/../b" \#p"/b")}
\= {()}
Cette fonction consid\`{e}re que {\tt "/a/../b"} et {\tt "/b"} sont des pathnames
Unix diff\'{e}rents \`{a} cause de l'\'{e}ventuelle pr\'{e}sence de liens
symboliques.



\Section {Nouvelles fonctions pour le paragraphe 6.10.3}

Tous les champs des objets de type {\tt pathname} peuvent \^{e}tre modifi\'{e}s au moyen
des fonctions suivantes.

\Function {set-pathname-host} {path host} {SUBR \`{a} 2 arguments}
Cette fonction permet de modifier le champs {\tt host} de son 
argument \LT path\GT\ de
type {\tt pathname}, et renvoie la nouvelle valeur du champs.
L'argument \LT host\GT\ doit \^{e}tre de type {\tt string} ou \^{e}tre \'{e}gal \`{a} ().

\Function {set-pathname-device} {path device} {SUBR \`{a} 2 arguments}
Cette fonction permet de modifier le champs {\tt device} de son argument \LT path\GT\ de
type {\tt pathname}, et renvoie la nouvelle valeur du champs.
L'argument \LT device\GT\ doit \^{e}tre de type {\tt string} ou \^{e}tre \'{e}gal \`{a} ().

\Function {set-pathname-directory} {path dir} {SUBR \`{a} 2 arguments}
Cette fonction permet de modifier le champs {\tt directory} de son argument \LT path\GT\ de
type {\tt pathname}, et renvoie la nouvelle valeur du champs.
L'argument \LT dir\GT\ doit \^{e}tre une liste, dont les \'{e}l\'{e}ments sont des cha\^{\i}nes
de caract\`{e}res ou l'un des symboles {\tt \#:pathname:up},
{\tt \#:pathname:current} et {\tt \#:pathname:wild}.

\Function {set-pathname-name} {path name} {SUBR \`{a} 2 arguments}
Cette fonction permet de modifier le champs {\tt name} de son argument \LT path\GT\ de
type {\tt pathname}, et renvoie la nouvelle valeur du champs.
L'argument \LT name\GT\ doit \^{e}tre de type {\tt string}, \^{e}tre \'{e}gal \`{a} () ou au
symbole {\tt \#:pathname:wild}.

\Function {set-pathname-type} {path type} {SUBR \`{a} 2 arguments}
Cette fonction permet de modifier le champs {\tt type} de son argument \LT path\GT\ de
type {\tt pathname}, et renvoie la nouvelle valeur du champs.
L'argument \LT type\GT\ doit \^{e}tre de type {\tt string} ou \^{e}tre \'{e}gal \`{a} ().

\Function {set-pathname-version} {path version} {SUBR \`{a} 2 arguments}
Cette fonction permet de modifier le champs {\tt version} de son argument \LT path\GT\ de
type {\tt pathname}, et renvoie la nouvelle valeur du champ.

\Function {combine-pathnames} {path1 path2} {SUBR \`{a} 2 arguments}
Cette fonction prend comme arguments deux pathnames \LT path1\GT\ et \LT
path2\GT\ et rend le pathname r\'{e}sultant de leur composition.
La composition de 2 pathnames suit les r\`{e}gles suivantes: \\
 - on choisit le champs {\tt host} de \LT path2\GT\, sauf s'il est
vide, auquel cas on choisit le champs {\tt host} de \LT path1\GT\ ; \\
 - on choisit le champs {\tt device} de \LT path2\GT\, sauf s'il est
vide, auquel cas on choisit le champs {\tt device} de \LT path1\GT\ ; \\
 - concat\'{e}nation des champs {\tt directory} des 2 pathnames 
si le champs {\tt directory} de \LT path2\GT\ d\'{e}crit un chemin
relatif. 
Si le champs {\tt directory} de \LT path2\GT\ n'est pas relatif, alors
il est pris tel quel comme r\'{e}sultat. Le champs {\tt directory}
r\'{e}sultat est \'{e}ventuellement simplifi\'{e} dans tous les cas; \\
 - les champs {\tt name}, {\tt type} et {\tt version} de \LT path2\GT\
deuxi\`{e}me pathname sont seuls consid\'{e}r\'{e}s au d\'{e}triment de 
ces champs dans \LT path1\GT\ .

Exemples :\\
\? {(combine-pathnames \#u"/home/" \#u"foo.sh")}
\= {\#p"/home/foo.sh"}
\? {(combine-pathnames \#u"/usr/lelisp/llib/" \#u"../llobj/pretty.lo")}
\= {\#p"/usr/lelisp/llobj/pretty.lo"}
\? {(combine-pathnames \#u"/usr/lelisp/llib/path.ll"
                       \#u"../llmod/path.lm")}
\= {\#p"/usr/lelisp/llmod/path.lm"}
\? {(combine-pathnames \#u"/usr/lelisp/llib/path.ll"
                       \#u"/usr/lelisp/llobj/path.lo")}
\= {\#p"/usr/lelisp/llobj/path.lo"}

Cette fonction est diff\'{e}rente de {\tt merge-pathnames} dans la mesure o\`{u}
cette derni\`{e}re sert \`{a} compl\'{e}ter les champs manquants d'un pathname, alors
que {\tt combine-pathnames} est une composition des champs d\'{e}crivant
le ``contenant{''} (les champs d\'{e}crivant l\`{a} o\`{u} est le fichier: {\tt
host, device, directory}) de deux 
pathnames, avec utilisation de ``l'objet contenu{''} (les champs
d\'{e}crivant le fichier: {\tt name, type, version}) du deuxi\`{e}me pathname.

On peut noter que quels que soient les pathnames {\tt p1} et {\tt p2},
l'\'{e}valuation suivante :\\ 
\? {(probefile (combine-pathnames p1 p2))}

rendra toujours le m\^{e}me r\'{e}sultat que :\\
\? {(with ((current-directory p1)) (probefile p2))}

\chapter {Le mode EXECORE sous UNIX}

Le mode {\tt EXECORE} existait d\'{e}j\`{a} sur certains portages. Il a
\'{e}t\'{e} g\'{e}n\'{e}ralis\'{e} autant que faire se peut, et sa mise en \oe uvre
pour l'utilisateur a \'{e}t\'{e} simplifi\'{e}e.

\Section {Description}
Certains portages ({\em Sun, IBMRT}... ) laissent le choix \`{a}
l'utilisateur de fabriquer des syst\`{e}mes \LeLisp\ directement
ex\'{e}cutables, sans passer par le m\'{e}canisme des {\tt cores}
d\'{e}port\'{e}s. Le portage sur {\em Vax} offre le m\'{e}canisme des {\tt execore}
en standard.

Dans une configuration {\tt execore} la commande {\tt lelisp} n'est plus un
script-shell de lancement du {\em binaire + core}, mais directement l'image
m\'{e}moire \LeLisp\ ex\'{e}cutable. \\
Pour obtenir un tel effet, il suffit \`{a}
l'utilisateur de jouer avec la variable de {\em Makefile} {\tt EXE}. Lorsque
cette variable n'a pas de valeur (cas par d\'{e}faut sur Sun et IBMRT),
le m\'{e}canisme {\tt execore} n'est pas utilis\'{e}. Si l'utilisateur
lui donne la valeur {\tt exe}, les images obtenues seront directement
{\tt ex\'{e}cutables}. \\
L'un des avantages au mode {\tt execore} sous certains Unix,
est qu'une image Le-Lisp se charge plus rapidement, puisqu'on ne
charge que le code dont on a besoin en m\'{e}moire. \\
Un autre avantage du mode {\tt execore} est que le message d'erreur
{\em no compatible core image} n'apparait plus.

\Section {Exemple}
\BeginLL
sun% make lelisp
/bin/cc -O -sun3 -DNBSYST=22 -DBSD4x -DBSD42 -DTIMEUNIT=60. -DINRIA -DSUNOS40 -B
static  -DSUN    -DFOREIGN  -DFILEINI=\"../llib/startup.ll\" -I../common -DSYSNA
ME=\"Le-Lisp\" -DSTAMP=\"$$\" -c ../common/lelisp.c
mv lelisp.o o/lelisp.o
cc -O -sun3 -DNBSYST=22 -DBSD4x -DBSD42 -DTIMEUNIT=60. -DINRIA -DSUNOS40 -Bstati
c  -DSUN    -DFOREIGN  -DFILEINI=\"../llib/startup.ll\" -I../common -DSYSNAME=\"
Le-Lisp\" -z -x -o lefpu31bin \
o/llmain.o o/llstdio.o o/llfloat.o  o/lelisp.o  lefpu31bin.o \
-lm -lc
ln lefpu31bin lelispbin
../system/config lelisp lelispbin lelispconf.ll "sun" -stack 6 -code 600 -heap 2
56 -number 0  -vector 4 -string 5 -symbol 5 -cons 4 -float 0
; Le-Lisp (by INRIA) version 15.23  (14/Fev/90)  [sun]
= (Version:  15.23)
= subversion
= herald
= defvar
= syste`me unix
 (load-std sav min pepe env ld llcp)  pour charger l'environnement std,
 (load-stm sav min pepe env ld llcp)  pour l'environnement std modulaire,
 (load-cpl sav min meme env ld cmpl)  pour l'environnement complice.
= /nfs/work/v15.2/llib/startup.ll
? (setq #:system:name (quote |lelisp|))
= lelisp
? ; configuration du syste`me Le_Lisp standard modulaire
? (defvar #:ld68k:mc68881 t)
= #:ld68k:mc68881
? (progn
?   (load-stm #:system:name t t t t t)
?   (add-feature 'MC68881)
?   (add-feature (if (eq 0.0 0.0)
?                  '31BITFLOATS
?                  '64BITFLOATS))
?   )
Je charge /nfs/work/v15.2/llib/lap68k.ll
Je charge loader.lm
Je charge /nfs/work/v15.2/llmod/llpatch.lm
Je charge /nfs/work/v15.2/llmod/module.lm
Je charge /nfs/work/v15.2/llmod/defs.lm
Je charge /nfs/work/v15.2/llmod/toplevel.lm
Je charge /nfs/work/v15.2/llmod/cpmac.lm
Je charge /nfs/work/v15.2/llmod/llcp.lm
Je charge /nfs/work/v15.2/llmod/peephole.lm
Je charge /nfs/work/v15.2/llmod/virtty.lm
Je charge /nfs/work/v15.2/llmod/virbitmap.lm
Je charge /nfs/work/v15.2/llmod/pepe.lm
Je charge /nfs/work/v15.2/llmod/setf.lm
Je charge /nfs/work/v15.2/llmod/defstruct.lm
Je charge /nfs/work/v15.2/llmod/sort.lm
Je charge /nfs/work/v15.2/llmod/array.lm
Je charge /nfs/work/v15.2/llmod/callext.lm
Je charge /nfs/work/v15.2/llmod/trace.lm
Je charge /nfs/work/v15.2/llmod/pretty.lm
Je charge /nfs/work/v15.2/llmod/debug.lm
Je charge /nfs/work/v15.2/llmod/ttywindow.lm
Je charge /nfs/work/v15.2/llmod/abbrev.lm
Je charge /nfs/work/v15.2/llmod/microceyx.lm
Attendez, je sauve : Syste`me standard modulaire
; Le-Lisp (by INRIA) version 15.23 (30/Sept/90)   [sun]
; Tu utilises /nfs/work/v15.2/
; Syste`me standard modulaire : lun 24 sept 90 20:37:27
= (31bitfloats mc68881 edlin date microceyx debug setf pepe virbitmap virtty
compiler pretty abbrev loader callext defstruct pathname messages)
? (end)
Que Le-Lisp soit avec vous.
sun% ls -l lelisp
-rwxrwxr-x  1 kuczynsk  184 Sep 24 20:36 lelisp
sun% cat lelisp
exec /udd/chatelet/work/v15.2/sun/lelispbin -stack 6 -code 600 -heap \
256 -number 0 -vector 4 -string 5 -symbol 5 -cons 4 -float 0 \
-r /udd/chatelet/work/v15.2/sun/llcore/lelisp.core $*
sun% make lelisp EXE=exe
/bin/cc -O -sun3 -DNBSYST=22 -DBSD4x -DBSD42 -DTIMEUNIT=60. -DINRIA -DSUNOS40 -B
static  -DSUN    -DFOREIGN  -DFILEINI=\"../llib/startup.ll\" -I../common -DSYSNA
ME=\"Le-Lisp\" -DSTAMP=\"$$\" -DEXECORE -c ../common/lelisp.c
mv lelisp.o o/lelispexe.o
cc -O -sun3 -DNBSYST=22 -DBSD4x -DBSD42 -DTIMEUNIT=60. -DINRIA -DSUNOS40 -Bstati
c  -DSUN    -DFOREIGN  -DFILEINI=\"../llib/startup.ll\" -I../common -DSYSNAME=\"
Le-Lisp\" -z -x -o lefpu31exebin \
o/llmain.o o/llstdio.o o/llfloat.o  o/lelispexe.o execore.o  lefpu31bin.o \
-lm -lc
ln lefpu31exebin lelispexebin
../system/config lelisp lelispexebin lelispconf.ll "sun" -stack 6 -code 600 -hea
p 256 -number 0  -vector 4 -string 5 -symbol 5 -cons 4 -float 0
; Le-Lisp (by INRIA) version 15.23  (14/Fev/90)  [sun]
= (Version:  15.23)
= subversion
= herald
= defvar
= syste`me unix
 (load-std sav min pepe env ld llcp)  pour charger l'environnement std,
 (load-stm sav min pepe env ld llcp)  pour l'environnement std modulaire,
 (load-cpl sav min meme env ld cmpl)  pour l'environnement complice.
= /nfs/work/v15.2/llib/startup.ll
? (setq #:system:name (quote |lelisp|))
= lelisp
? ; configuration du syste`me Le_Lisp standard modulaire
? (defvar #:ld68k:mc68881 t)
= #:ld68k:mc68881
? (progn
?   (load-stm #:system:name t t t t t)
?   (add-feature 'MC68881)
?   (add-feature (if (eq 0.0 0.0)
?                  '31BITFLOATS
?                  '64BITFLOATS))
?   )
Je charge /nfs/work/v15.2/llib/lap68k.ll
Je charge loader.lm
Je charge /nfs/work/v15.2/llmod/llpatch.lm
Je charge /nfs/work/v15.2/llmod/module.lm
Je charge /nfs/work/v15.2/llmod/defs.lm
Je charge /nfs/work/v15.2/llmod/toplevel.lm
Je charge /nfs/work/v15.2/llmod/cpmac.lm
Je charge /nfs/work/v15.2/llmod/llcp.lm
Je charge /nfs/work/v15.2/llmod/peephole.lm
Je charge /nfs/work/v15.2/llmod/virtty.lm
Je charge /nfs/work/v15.2/llmod/virbitmap.lm
Je charge /nfs/work/v15.2/llmod/pepe.lm
Je charge /nfs/work/v15.2/llmod/setf.lm
Je charge /nfs/work/v15.2/llmod/defstruct.lm
Je charge /nfs/work/v15.2/llmod/sort.lm
Je charge /nfs/work/v15.2/llmod/array.lm
Je charge /nfs/work/v15.2/llmod/callext.lm
Je charge /nfs/work/v15.2/llmod/trace.lm
Je charge /nfs/work/v15.2/llmod/pretty.lm
Je charge /nfs/work/v15.2/llmod/debug.lm
Je charge /nfs/work/v15.2/llmod/ttywindow.lm
Je charge /nfs/work/v15.2/llmod/abbrev.lm
Je charge /nfs/work/v15.2/llmod/microceyx.lm
Attendez, je sauve : Syste`me standard modulaire
; Le-Lisp (by INRIA) version 15.23 (30/Sept/90)   [sun]
; Tu utilises /nfs/work/v15.2/
; Syste`me standard modulaire : lun 24 sept 90 20:44:05
= (31bitfloats mc68881 edlin date microceyx debug setf pepe virbitmap virtty
compiler pretty abbrev loader callext defstruct pathname messages)
? (end)
Que Le-Lisp soit avec vous.
sun% ls -l lelisp
-rwxrwxr-x  1 kuczynsk 1754132 Sep 24 20:44 lelisp
sun% ./lelisp
; Le-Lisp (by INRIA) version 15.23 (30/Sept/90)   [sun]
; Tu utilises /nfs/work/v15.2/
; Syste`me standard modulaire : lun 24 sept 90 20:44:05
= (31bitfloats mc68881 edlin date microceyx debug setf pepe virbitmap virtty
compiler pretty abbrev loader callext defstruct pathname messages)
? (end)
Que Le-Lisp soit avec vous.
sun%
\EndLL

\Section {Notation}
On notera qu'il n'est pas n\'{e}cessaire d'ajouter l'option {\tt -s}
comme avec d'anciennes versions apr\`{e}s la commande {\tt lelisp} en
mode {\tt execore}. \\
Seules les noms des commandes finales (\|lelisp|, \|cmplc|, etc) sont
partag\'{e}es
entre le mode execore et le mode normal. Aussi, il existe un binaire
diff\'{e}rent selon les deux modes. On trouvera par exemple, un {\tt
lelispbin} et un {\tt lelispexebin} dans les entr\'{e}es du {\em Makefile}.

\chapter {La boucle d'\'{e}v\'{e}nements}

Dans ce chapitre nous d\'{e}crivons le moyen de disposer d'une seule boucle
d'\'{e}v\'{e}nements dans un programme \LeLisp\ utilisant plusieurs types
d'entr\'{e}es\,:\\
\begin {itemize}
\item graphiques : clavier,souris,...\,;
\item fichiers au sens UNIX ({\tt socket}, {\tt pipe}, ...)\,.
\end{itemize}
Le module d\'{e}crit dans ce papier, illustre une interface fonctionnelle
accessible \`{a} l'utilisateur qui permet d'associer des fonctions
de traitements \`{a} des entr\'{e}es particuli\`{e}res.\\
Cette boucle d'\'{e}v\'{e}nements permet aussi de r\'{e}gler le probl\'{e}me des
interruptions\,: dans l'attente d'un ou de plusieurs types d'\'{e}v\'{e}nements
toutes les interruptions classiques sont trait\'{e}es.\\
L'impl\'{e}mentation de cette boucle repose aujourd'hui sur le
 {\em terminal virtuel} de \LeLisp\ (voir le chapitre  15 du Manuel de 
r\'{e}f\'{e}rence) et sur la fonctionnalit\'{e} UNIX {\tt select}. Le code
pr\'{e}sent\'{e} ici ne tourne aujourd'hui qu'avec UNIX. D'autres
implantations de la boucle d'\'{e}v\'{e}nements sur d'autres syst\`{e}mes sont
faisables et seront r\'{e}alis\'{e}s.\\
La boucle d'\'{e}v\'{e}nements offerte marche en standard avec le terminal 
virtuel de type {\tt \#:tty:evloop}.\\
Cela interdit toute utilisation de la boucle d'\'{e}v\'{e}nements avec
un autre type de terminal virtuel qui n'est pas un sous-type de
{\tt \#:tty:evloop}.\\
Toutefois une exception est faite pour le terminal virtuel utilis\'{e} par
Aida o\`{u} la boucle d'\'{e}v\'{e}nements est int\'{e}gr\'{e}e d'office.\\

\Section {Fonctions}
\Function {evloop-init} {} {SUBR sans argument}
Cette fonction initialise la boucle d'\'{e}v\'{e}nements et positionne
le terminal virtuel {\tt \#:tty:evloop} comme terminal courant.\\
Cette fonction ne devrait \^{e}tre appel\'{e}e qu'une seule fois.

\Function {evloop-stop} {} {SUBR sans argument}
Cette fonction inhibe l'utilisation de la boucle d'ev\`{e}nements et positionne
comme terminal virtuel le terminal par d\'{e}faut du syst\`{e}me {\tt tty}.\\

\Function {evloop-restart} {} {SUBR sans argument}
Cette fonction red\'{e}marre la boucle d'\'{e}v\'{e}nements et repositionne
le terminal virtuel {\tt \#:tty:evloop} comme terminal courant.

\Function {evloop-disallow-tty-input} {} {SUBR sans argument}
Cette fonction inhibe le traitement des entr\'{e}es de l'utilisateur sur
l'entr\'{e}e standard par la boucle d'\'{e}v\'{e}nements.\\
Cette fonction n'a d'effets que si la boucle d'\'{e}v\'{e}nements est 
initialis\'{e}e et est active.


\Function {evloop-allow-tty-input} {} {SUBR sans argument}
Cette fonction autorise le traitement des entr\'{e}es de l'utilisateur sur
l'entr\'{e}e standard par la boucle d'\'{e}v\'{e}nements.\\
Cette fonction n'a d'effets que si la boucle d'\'{e}v\'{e}nements est 
initialis\'{e}e et est active.

\Function {evloop-add-input} {fd manage-function arg-to-use} {SUBR \`{a} 3 arguments}
Cette fonction ajoute le descripteur de fichier ({\em unix}) \`{a} la boucle
d'\'{e}v\'{e}nements.\\
Si la boucle d'\'{e}v\'{e}nements est active d\`{e}s qu'il y a quelque chose \`{a}
lire sur le descripteur de fichier {\tt fd} la fonction
{\tt manage-function} est ex\'{e}cut\'{e}e avec comme argument {\tt arg-to-use}.\\
Cette fonction n'a aucun effet si la boucle d'\'{e}v\'{e}nements n'est pas 
initialis\'{e}e.

\Function {evloop-remove-input} {fd} {SUBR \`{a} 1 argument}
Cette fonction enl\^{e}ve  le descripteur de fichier {\tt fd} de la liste
des descripteurs de fichiers trait\'{e}s par la boucle d'\'{e}v\'{e}nements.\\

\Function {evloop-change-manage-function} {fd new-manage-function new-arg-to-use} {SUBR \`{a} 3 arguments}
Si le descripteur de fichier {\tt fd} est d\'{e}j\`{a} trait\'{e} dans la boucle
d'\'{e}v\'{e}nements cette fonction remplace la fonction \`{a} ex\'{e}cuter 
lorsqu'il y a quelques choses \`{a} lire dans {\tt fd} par la fonction
{\tt new-manage-function} avec comme argument {\tt new-arg-to-use}.

\Function {evloop-select} {} {SUBR sans arguments}
Cette fonction n'est pas utile pour l'utlisation standard de ce module
 elle ne sert qu'a l'utilisateur qui d\'{e}sire impl\'{e}menter un nouveau type
de terminal virtuel.\\
Cette fonction repr\'{e}sente le moteur de la boucle d'\'{e}v\'{e}nements\,:\\
Elle est tout le temps appel\'{e} par le syst\`{e}me  et c'est elle 
qui bloque le programme dans l'attente d'un ou de plusieurs descripteurs
de fichiers pr\^{e}ts en lecture pour ensuite ex\'{e}cuter les
fonctions  qui leurs sont associ\'{e}s.

\Function {evloop-readp} {fd} {SUBR sans arguments}
Cette fonction retourne la valeur {\tt t} si une entr\'{e}e est pr\^{e}te en
lecture. C'est la g\'{e}n\'{e}ralisation de la fonction {\tt eventp} du Bitmap 
Virtuel. La fonction retourne {\tt ()}  sinon.

\Function {evloop-input-managedp} {} {SUBR \`{a} 1 argument}
Cette fonction retourne la valeur {\tt t} si l'entr\'{e}e {\tt fd} est trait\'{e} par
la boucle d'\'{e}v\'{e}nements. L'argument {\tt fd} est typiquement un entier 
designant un descripteur de fichiers sous {\tt UNIX}.


\Function {evloop-initialized-p} {} {SUBR sans  arguments}
Cette fonction retourne  la valeur {\tt t} si la boucle d'\'{e}v\'{e}nements est
d\'{e}j\`{a} initialis\'{e}e.


\Function {evloop-wait} {} {fonction sans arguments}
Cette fonction met le processus en attente d'une entr\'{e}e.\\
Lorsque l'attente est termin\'{e}e la fonction retoune le descripteur
 (s'il y en a un seul) sous la forme d'une paire point\'{e} LeLisp 
({\em (fd . (function . arg))}) o\`{u}  {\tt fd} est le descripteur de fichier,
 {\tt function} la fonction de traitement qui lui est associ\'{e}e et 
{\tt arg} l'argument enregistr\'{e}.
S'il y plus d'un descripteur pr\^{e}t en lecture la fonction retourne une
liste de paires point\'{e}es.

\Function {evloop-set-timeout} {secs millisecs} {fonction \`{a} 2 arguments}
Cette fonction permet de limiter le temps d'attente du processus durant les
appels aux fonctions {\tt evloo-select} et {\tt evloop-wait}.\\
Lorsque le temps d'attente sp\'{e}cifi\'{e} par {\tt secs} -- en secondes -- et (ou)
{\tt millisecs} -- en milli\'{e}mes de secondes -- est ecoul\'{e} durant une attente
le deux fonctions cit\'{e}es peuvent \'{e}ventuellement ex\'{e}cuter une fonction
(voir la fonction {\tt evloop-set-timeout-handler}.\\
Les arguments {\tt secs} et {\tt millisecs} doivent \^{e}tre des entiers 
positifs. S'ils valent 0 tous les deux les attentes sont infinies.
Cette fonction permet donc de d\'{e}clencher un traitement p\'{e}riodique lorsque
le processus ne re\c{c}oit plus d'entr\'{e}es pendant une p\'{e}riode sup\'{e}rieure 
\`{a} la frequence d\'{e}finie par {\tt secs } et {\tt millisecs}.\\
Un exemple d'utilisation dans \Aida de cette fonction serait d'afficher l'heurer\'{e}guli\`{e}rement.\\
A la difference de la fonction {\tt clockalarm} de \LeLisp le traitement
n'est pas declench\'{e} pendant l'ex\'{e}cution d'un traitement autre que
l'appel de des fonction  {\tt evloo-select} et {\tt evloop-wait}.

\Function {evloop-set-timeout-handler} {handler} {fonction \`{a} 1 argument}
L'argument {\tt  handler} est soit {\tt ()} soit une fonction LeLisp \`{a}
appeler lorsque le temps specifi\'{e} par la fonction {\tt evloop-set-timeout}
est ecoul\'{e}.

\SSection {Traitement des sorties}
Cette section presente les fonctions suppl\'{e}mentaires pour ajouter des 
sorties (fichiers, descripteurs de fichiers et autres) dans la boucle
d'\'{e}v\'{e}nements.\\
\Function {evloop-add-output} {fd manage-function arg-to-use} {fonction \`{a} 3 arguments}
Cette fonction rajoute le descripteur de fichier {\tt fd} ({\em unix}) \`{a} la boucle d'\'{e}v\'{e}nements.\\
Si la boucle d'\'{e}v\'{e}nements est active d\`{e}s que le descripteur de fichier
{\tt fd} est pr\^{e}t en lecture la fonction 
{\tt manage-function} est ex\'{e}cut\'{e}e avec comme argument {\tt arg-to-use}.\\
Cette fonction n'a aucun effet si la boucle d'\'{e}v\'{e}nement n'est pas 
initialis\'{e}e.

\Function {evloop-remove-output} {fd} {fonction \`{a} 1 argument}
Cette fonction enl\`{e}ve  le descripteur de fichier {\tt fd} de la liste
des descripteurs de fichiers trait\'{e}s par la boucle d'\'{e}v\`{e}nements en \'{e}criture.\\

\Function {evloop-change-output-manage-function} {fd new-manage-function new-arg-to-use} {fonction \`{a} 3 arguments}
Si le descripteur de fichier {\tt fd} est d\'{e}j\`{a} trait\'{e} dans la boucle
d'\'{e}v\`{e}nements cette fonction remplace la fonction \`{a} ex\'{e}cuter 
lorsqu'il y a quelques choses \`{a} lire dans {\tt fd} par la fonction
{\tt new-manage-function} avec comme argument {\tt new-arg-to-use}.

\Section {Exemple}
Nous pr\'{e}sentons dans ce chapitre un exemple d'utilisation de la
boucle d'\'{e}v\'{e}nements avec le {\tt bitmap virtuel} de \LeLisp\ sur
l'incarnation {\tt X11 windows}.\\
\BeginLL
;;; Definition de fonctions utilitaires
;;;
(defun display-manage-events (display)
;;; Cette fonction lit tous les evenements en attente dans 
;;; le dipositif d'affichage DISPLAY et les imprime a l'ecran
        (current-display display)
        (while (eventp) (print (read-event))))
;;;
;;; Lancement du systeme Le-Lisp
% lelispX11
; Le-Lisp (by INRIA) version 15.23 (14/Fev/90)   [sony]
; Systeme standard sur X11-windows : ven  6 juil 90 14:25:52 
= (31bitfloats abbrev callext compiler date debug defstruct edlin loader 
mc68881 messages microceyx pathname pepe pretty setf virbitmap virtty)
? (loadmodule 'evloop)
= evloop
? ;;; Chargement du module qui definit la boucle d'evenements
?  (setq display (bitprologue))
= #<#:display:x11 X11 trocadero:0.0>
? ;;; Initialisation du dispositif d'affichage
? (evloop-init)
= t
? (setq fd (send 'file-descriptor display))
= 3
? ;;; On recupere le descripteur de fichier (socket) associe a DISPLAY
? (evloop-add-file-descriptor fd 'display-manage-events display)
= t
? ;;; On le rajoute a la boucle d'evenements standard
? (setq w (create-window 'window 0 0 200 200 "lisp window1" 1 1))
= #<#:image:rectangle:window lisp window1> 
? ;;; On cree une fenetre
?  (setq display1 (bitprologue '|X11| "host1:0"))
= #<#:display:x11 X11 trocadero:0.0>
? ;;; Initialisation du dispositif d'affichage sur un deuxieme ecran
= t
? (setq fd1 (send 'file-descriptor display))
= 4
? ;;; On recupere le descripteur de fichier (socket) associe a DISPLAY1
? (evloop-add-file-descriptor fd1 'display-manage-events display)
= t
? ;;; On le rajoute a la boucle d'evenements standard
? (setq w1 (create-window 'window 0 0 200 200 "lisp window2" 1 1))
= #<#:image:rectangle:window lisp window2> 
? ;;; On cree une deuxieme fenetre dans l'ecran de la machine "host"
? ;;; Et tous les evenements clavier ou souris associees qux
? ;;; deux fenetres w et w1  seront affiches
? ;;; de me^me que tout ce que l'utilisateur tape au clavier
? ;;; dans la fenetre de lancement de Le-Lisp
#:event:#[enterwindow-event #<#:image:rectangle:window lisp window1> () 175 
194 167 169 () ()]
#:event:#[down-event #<#:image:rectangle:window lisp window1> 0 143 154 135 
129 () ()]
#:event:#[up-event #<#:image:rectangle:window lisp window1> 0 143 154 135 129 
() ()]
#:event:#[leavewindow-event #<#:image:rectangle:window lisp window1> () 279 
254 271 229 () ()]
#:event:#[enterwindow-event #<#:image:rectangle:window lisp window2> () 174 
177 166 152 () ()]
#:event:#[ascii-event #<#:image:rectangle:window lisp window2> 97 193 147 185 
122 () ()]
#:event:#[leavewindow-event #<#:image:rectangle:window lisp window2> () 225 
180 217 155 () ()]
? (end)
Que Le-Lisp soit avec vous.
\EndLL

\newpage
\bigskip

\tableofcontents
\listoftables

\End
