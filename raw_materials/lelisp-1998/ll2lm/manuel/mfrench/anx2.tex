%===================================================================
\Chapter {2}{Limitations et difficult\'{e}s connues}
%===================================================================

%---------------------------------------------------------------------------
\Section {Restrictions}
%---------------------------------------------------------------------------

Certains points ne sont pas parfaitement trait\'{e}s\,:

\begin{itemize}
\item les modules utilis\'{e}s par l'{\em Analyseur de Modules} lui-m\^{e}me
(\|path|, \|sets|, \|stringio|,...) peuvent \^{e}tre oubli\'{e}s dans ses
propositions d'importations\,;

\item les m\'{e}thodes et les fonctions appel\'{e}es par leur nom (avec {\tt
send}, {\tt map}, ...), si le nom de fonction n'apparait pas
explicitement de mani\`{e}re {\em quot\'{e}e}: {\tt '<function>}.
\end{itemize}

Pour ces raisons, les diagnostics de l'{\em Analyseur de Modules} peuvent ne
pas \^{e}tre complets,
mais ils aident \`{a} affiner la description d'un module.

%---------------------------------------------------------------------------
\Section {Probl\`{e}mes connus}
%---------------------------------------------------------------------------

\SSection{Coupure des symboles trop longs}

Lors de l'\'{e}criture des fichiers \|.lm|
ou des bases de r\'{e}f\'{e}rences (.ref),
il peut arriver qu'un symbole trop long soit coup\'{e} incorrectement.
Par exemple, le nom de fonction \Masai2d\ suivant:

\begin{Longcode*}
#:(#:ge-object:point:bbox:set:lego:ge-plain:polyline . #:ge-object:point):ge-contains?
\end{Longcode*}

est mal coup\'{e} et devient:

\begin{Longcode*}
#:(#:ge-object:point:bbox:set:lego:ge-plain:polyline . #:ge-object:point)
:ge-contains?
\end{Longcode*}

Ceci sera relu comme deux symboles diff\'{e}rents lors du chargement
ult\'{e}rieur du fichier\,:

\begin{Longcode*}
#:(#:ge-object:point:bbox:set:lego:ge-plain:polyline . #:ge-object:point)
#:user:ge-contains?
\end{Longcode*}


Cette mauvaise \'{e}criture provoquera des erreurs lors de l'analyse du
module concern\'{e} et de tout module utilisant la fonction en question.

Il est bien s\^{u}r possible de modifier \`{a} la main le fichier \|.lm|. 
Par contre, le fichier de r\'{e}f\'{e}rences (.ref) est relu et r\'{e}\'{e}crit \`{a}
l'analyse de tout module du projet, la modification manuelle de ce
fichier est donc inutile.

La seule solution s\'{e}rieuse
consiste \`{a} changer la valeur de la marge lors de
l'analyse en modifiant la description de projet comme suit:

\begin{Code*}
(defun proj-activate-func ()
  ...
  (rmargin 255))

(define-rt-project proj
    required-projects (lisp)    
    ...
    activate-function proj-activate-func
    ...
    ))
\end{Code*}

Le changement de la marge peut, \'{e}ventuellement, provoquer la m\^{e}me
erreur sur d'autres noms de fonctions du projet. Dans ce cas, essayer
une autre valeur de la marge, jusqu'\`{a} la valeur maximale\,:

\begin{Code*}
(rmargin (1+ (slen (outbuf))))
\end{Code*}


\SSection{H\'{e}ritage de structure}

Si votre projet d\'{e}finit des objets \|microceyx|, vous avez d\^{u}
choisir entre l'h\'{e}ritage et la red\'{e}finition des accesseurs aux
champs. Ce m\'{e}canisme est contr\^{o}l\'{e} par la variable globale
\|#:system:defstruct-all-access-flag|. Les projets \Aida\ et \Masai2d\
utilisent l'h\'{e}ritage des accesseurs et positionnent donc la variable
\|#:system:defstruct-all-access-flag| \`{a} \|()| lors de l'activation du
projet.

Si vous souhaitez que la variable
\|#:system:defstruct-all-access-flag| soit \`{a} \|t| dans votre projet
alors que celui-ci requiert les projets \Aida\ ou \Masai2d\, il faut que
vous definissiez une fonction d'activation comme suit \,:

\begin{Code*}
(defun proj-activate-func ()
  (setq #:system:defstruct-all-access-flag t))               

(define-rt-project proj
  required-projects (aida)      
  ...
  activate-function maida2d-activate-func
  )
\end{Code*}

Dans ce cas, l'avertissement suivant peut se produire lors de
l'analyse de vos modules\,:

\begin{Longcode*}
** W.114 :
structure image must be defined inside an EVAL-WHEN(COMPILE) in module:
(image)
\end{Longcode*}

Vous pouvez ignorer ce message.


\SSection{Abr\'{e}viations}

Il existe des cas o\`{u} une abr\'{e}viation n'est pas trouv\'{e}e lors de
l'analyse.
En effet, il peut arriver que le champ \|cpenv| du module soit alt\'{e}r\'{e} par
l'ajout d'une abr\'{e}viation erron\'{e}e: \|{unknown-abbrev}:<abrev>|.
Dans ce cas, les analyses suivantes provoqueront le message d'erreur
ci-dessous\,:

\begin{Code*}
** W.104 : parent structure package unknown-abbrev not found for : {
unknown-abbrev}:<abrev>
\end{Code*}

Il est alors n\'{e}cessaire de corriger manuellement le fichier \|.lm|, 
quite \`{a} supprimer la forme \LeLisp\ contenant l'abr\'{e}viation erron\'{e}e.

\SSection{Conflits de noms de modules}
Il se peut qu'un fichier d'une application en cours d'analyse porte le
m\^{e}me nom qu'un des projets utilis\'{e}s.
La priorit\'{e} d'utilisation est g\'{e}r\'{e}e avec la variable {\tt
\#:system:path}. Par d\'{e}faut, priorit\'{e} est donn\'{e}e aux projets import\'{e}s.
Il conviendra de r\'{e}soudre de tels conflits
\`{a} la main, l'id\'{e}al \'{e}tant d'\'{e}viter tout conflit de nom de
fichier.


\SSection{G\'{e}n\'{e}ration du {\tt Makefile} de compilation}

La g\'{e}n\'{e}ration du \|Makefile| de compilation se fait \`{a} partir de la
base de r\'{e}f\'{e}rences (fichier .ref) et non \`{a} partir des descriptions
modulaires (fichiers \|.lm|). Il est donc n\'{e}cessaire de fabriquer la
base de r\'{e}f\'{e}rences -- soit par l'option \|-build|, soit en analysant
tous les modules du projet -- avant de construire le \|Makefile|
de compilation. \\
On prendra soin lors de l'utilisation d'options de compilation \`{a}
l'aide de \|complice-options|, que les cha\^{\i}nes de caract\`{e}res
utilis\'{e}es ne soient pas plus longues qu'une ligne de {\tt Makefile}.
Si c'etait le cas, il conviendrait de supprimer les {\tt \^M}
perturbateurs.



