%%%%%%%%%%%%%%%%%%%%%%%%%%%%%%%%%%%%%%%%%%%%%%%%%%%%%%%%%%%%%%%%%%%%%%%%%%%%
\Chapter {3} {The Module Compiler}
%%%%%%%%%%%%%%%%%%%%%%%%%%%%%%%%%%%%%%%%%%%%%%%%%%%%%%%%%%%%%%%%%%%%%%%%%%%%

In this chapter, you will learn how to use the results of the {\em Module Analyzer} with the {\em compiler}.
You will also briefly review how to use the modular compiler delivered
with \LeLisp\,:  {\tt complice}.

%---------------------------------------------------------------------------
\Section {After the analysis:  The compilation}
%---------------------------------------------------------------------------

We saw in the preceding chapter how to create \LeLisp\ modular description files.  In reality, these files only exist for the {\tt complice} compiler that uses information contained in these modular description files to compile efficiently. 

%---------------------------------------------------------------------------
\Section {The compilation:  Analyzer specifics}
%---------------------------------------------------------------------------

A certain number of keys of \|define-rt-project| and an option of the analysis command {\tt ll2lm} deal directly with the compilation of the modules.  The following sections detail these options:

\SSection {The Analysis Option dedicated to the compilation}
%-----------------------------------------------------
The {\tt ll2lm} analyzer option dedicated to the compilation allows you to create a compilation {\tt Makefile}:
\begin{itemize}

\item {\Large \|-makefile|}: this principal option allows you to create a modular compilation {\tt Makefile} ({\tt see complice}) of all the concerned project modules (see the \|-project| option). 
By default, the resulting file of a modular compilation by {\tt complice} ({\tt
.lo}) is saved in the specific directory by the {\tt
ll-object-directory} key, if it exists.  If the {\tt
ll-object-directory} key does not exist, the resulting file of a modular compilation is saved next to the modular description file ({\tt .lm}).
This {\tt Makefile} offers certain predefined entries:
\begin {enumerate}
\item \|all| (default)\ :\ 
allows you to update the compilations of all the project modules.
\item \|clean|\ :\ 
allows you erase the object files ({\tt .lo}) of all the project modules.
\BeginLL
	% make -f myproject.mk clean
\EndLL
\item \|i|\ :\ 
allows you to enter in the interactive {\tt toplevel} loop of the compiler with all the compilation flags loaded in the environment.
\BeginLL
	% make -f myproject.mk i
\EndLL
\end{enumerate}
See all the \|-dependency| option to work on the dependency levels of the 
{\tt Makefile} entries.  At the same time as the {\tt Makefile}, a {\bf
project.pth} file is generated containing a definition of the {\tt
\#:system:path} variable with all the necessary access paths for the execution of the {\tt <project>} project modules.

\end{itemize}

We advise you to use this {\tt Makefile} in all of your project management to compile the ensemble of the modules.  This {\tt Makefile} will be useful to create the first series of compilations (the first time for the ensemble of modules) but equally in further instances for the maintenance of the modules.
In general, for now, we will start the {\tt Makefile} to update the ensemble of modified modules at their dependencies.  The dependencies vary in function of the \|-dependency| analysis option used at the time of the creation of the {\tt Makefile} (see Chapter 6).

\SSection {Project definition keys dedicated to the compilation}
%--------------------------------------------------------------------
The \|define-rt-project| keys concerning the compilation are:
\begin{itemize}

\item {\Large \|ll-object-directory| {\em path}}: this key, if it exists, must designate the directory where the resulting files of a ({\tt .lo}) modular compilation will be organized by {\tt
complice}.
This value can be of type {\tt string} or {\tt pathname}.  By default, these files will be organized in the directory designated by the \|ll-module-directory| key or, if this key no longer exists, it will be organized in the same directory as the module descriptor file ({\tt .lm}).  This key is used by the \|-makefile| option of the {\em Module Analyzer}.

\item {\Large \|make-file| {\em path}}: this key, if it exists, must designate the name of the compilation {\tt Makefile} generated by the module analyzer with the option \|-makefile|.  This value can be of the type {\tt string} or {\tt pathname}.
By default, the name of this {\tt Makefile} will be calculated from the project name that has the extension suffix {\tt ".mk"}.  It will also be organized in the directory designated by the \|system-directory| key (or its default value if it doesn't exist).
This key is used by the \|-makefile| option of the {\em Module Analyzer}.

\item {\Large \|complice-options| {\em A-list}}: this key, if it exists, contains an A-list where the first element of each sub-list is a module file descriptor name and the rest of the sub-list is a list of character strings that represent the options used at the time of the designated module compilation.  For example, if MYMOD must be compiled with the \|-parano T|\ compilation flag, you would write:
\BeginLL
(define-rt-project myproject
  ...
  complice-options ((mymod "-parano T"))
  ...
)
\EndLL
It is possible to require compilation options for all the project modules by using the {\em "all"} keyword in the form of {\it
string} in the place of the module name:
\BeginLL
(define-rt-project myproject
  ...
  complice-options ((mymod1 "-parano T")
                    ("all" "-o /tmp/"))
  ...
)
\EndLL
The behavior of "all" in the \|complice-options| key is identical to that of "all" in the \|analyzer-options| key.

This key is used by the \|-makefile| option of the {\em Module Analyzer}.

\end{itemize}


%---------------------------------------------------------------------------
\Section {Compiling a module}
%---------------------------------------------------------------------------
There are a certain number of options of the compiler that are usuable with the {\tt complice} command.  These options can be used module by module (or globally) in the {\tt Makefile} created by the {\em Module Analyzer} in the \|complice-options| field of {\tt define-rt-project}.

{\it Don't forget: Under Unix, these options are detailed in the Unix manual furnished with the distribution band:  {\tt lelisp/manl/complice.l}}

\begin{itemize}
\item {\Large \|-callext flagp|} : 
allows you to decide if you will create the accessors to the external symbols in function of \|flagp|.
If \|flagp| is NIL (default), the accessors are redefined so that they have no effect during the compilation:  the functions {\tt getglobal} and {\tt defextern-cache} are redefined to stay inoperational.  If \|flagp| is not null, these functions continue to function normally during the compilation:  the external symbols must be present in the binary image of the \|complice| compiler. 

\item {\Large \|-cons n|} : allows you to extend the {\tt cons} zone 
to a value equal to {\tt \|n| * 8 Kcons}.

\item {\Large \|-cmplc cmd|} :  allows you to specify an other start-up command that the \|cmplc++| memory image used by default 

\item {\Large \|-e S-expr|} : allows you to execute the \LeLisp\ form \|S-expr| in the compilation environment before peforming the modular compilation. 

\item {\Large \|-g flagp|} : allows you to require posting of the state of the lisp memory with the help of the (GC T) command after the compilation.

\item {\Large \|-i|} : allows you to enter under the interactive {\tt toplevel} of \|complice|.  This feature is especially useful in the ``debug'' phase to interactively observe the behavior of the compiler.  This option is activated with the \|i| entry of the compilation {\tt Makefile} created by the {\em Module Analyzer}.  

\item {\Large \|-lldir path|} : allows you to look for a compilation core (by default:  \|cmplc++|) in the directory described by \|path|.

\item {\Large \|-o path|} :  allows you to specify the directory where the resulting file from the modular compilation ({\tt .lo}) will be organized.
By default, these files are organized in the current directory.

\item {\Large \|-p path|} : allows you to add the \|path| path in the \|#:system:path| variable before the modular compilation.

\item {\Large \|-parano flagp|} : allows you to set the variable of the compiler \|#:complice:parano-flag| to the value of \|flagp|.  The value by default is NIL.  (see {\em Chapter 13} of the \LeLisp\ {\em Reference Manual} for more details on the use of this variable.)

\item {\Large \|-r|} : allows you to set the \|#:system:read-case-flag| to \|T|.  This option is mandatory if you want to influence the value of \|#:system:read-case-flag|; do not use the \|-e| option that will have no effect in this case.

\item {\Large \|-v|} : allows you to post the \LeLisp\ forms evaluated in the compilation environment before the modular compilation

\item {\Large \|-w flagp|} : allows you to set the \|#:complice:warning-flag| variable to the \|flagp| value.  (see {\em Chapter 13} of the \LeLisp\ {\em Reference Manual} for more details on the use of this variable.)
\end{itemize}
