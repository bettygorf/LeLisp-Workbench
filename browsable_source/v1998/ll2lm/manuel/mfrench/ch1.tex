%==========================================================================
\Chapter {1} {Introduction \`{a} la gestion modulaire de l'application}
%==========================================================================


Ce manuel pr\'{e}sente comment s'emploient les outils de g\'{e}n\'{e}ration
de modules de \LeLisp. La fonction et l'int\'{e}r\^{e}t de ces outils
sont pr\'{e}sent\'{e}s progressivement dans le pr\'{e}sent chapitre.

Les lecteurs non familiers avec les notions de {\em programmation
modulaire} ou de {\em modules} trouveront dans la premi\`{e}re section
une introduction \`{a} ces notions.


%---------------------------------------------------------------------------
\Section {Application Le-Lisp et modules}
%---------------------------------------------------------------------------


\SSection{A propos de l'application}


\SSSection{Inadaptation des fichiers interpr\'{e}t\'{e}s}

Pour assurer dans de bonnes conditions la gestion du code source
d'une application \LeLisp\ \`{a} des fins de d\'{e}veloppement, de
maintenance, et surtout de livraison \`{a} l'utilisateur final, 
l'usage direct des fichiers source s'av\`{e}re 
incommode et peu robuste\,:
\begin{itemize}
\item cela impose d'op\'{e}rer des regroupements de fichiers
pour lesquels le langage n'offre pas de support \'{e}vident\,;
\item charg\'{e}s par une forme comme {\tt loadfile} -- et m\^{e}me
incorpor\'{e}s dans une image m\'{e}moire -- les fichiers sources donnent lieu \`{a}
des codes interpr\'{e}t\'{e}s qui ne sont pas ex\'{e}cut\'{e}s de mani\`{e}re
optimale\,;
\item livrer une application sous forme de source la fragilise
-- dans la mesure ou la moindre fonction interne de l'application
est accessible en modification ou peut-\^{e}tre \'{e}cras\'{e}e -- et
est contraire \`{a} tout objectif de confidentialit\'{e}.
\end{itemize}


\SSSection{Disponibilit\'{e} du compilateur modulaire}

\LeLisp\ offre la possibilit\'{e} de regrouper les fichiers
sources sous formes de {\em modules} et {\bf de les compiler}.
Le compilateur modulaire \LeLisp\ produit des fichiers objets
contenant un code interm\'{e}diaire, qui, une fois 
charg\'{e}\footnote{Le chargement du code interm\'{e}diaire est effectu\'{e}
par ce que nous appellerons dans la suite le {\em chargeur}
ou {\em loader}.}, s'ex\'{e}cute
de 3 \`{a} 30 fois plus rapidement que le m\^{e}me code interpr\'{e}t\'{e}.
Cette compilation, tout en assurant la compatibilit\'{e} avec 
l'interpr\`{e}te, introduit un certain nombre d'exigences -- 
telles que l'interdiction de recourir
\`{a} des liaisons dynamiques de variables -- et r\'{e}alise un
ensemble de v\'{e}rifications statiques\footnote{Et notamment un
contr\^{o}le syntaxique sur {\bf tout} le code, alors qu'un
code interpr\'{e}t\'{e} peut difficilement, dans la pratique,
\^{e}tre \'{e}valu\'{e} exhaustivement.} qui concourent \`{a}
une plus grande solidit\'{e} des codes.


\SSSection{Principe de modularit\'{e}}

La compilation modulaire suppose essentiellement de d\'{e}finir
pour chaque module l'{\tt interface}
avec les autres modules, c'est \`{a} dire l'ensemble des fonctions
qui pourront \^{e}tre appel\'{e}es depuis l'ext\'{e}rieur du module.

Si l'on consid\`{e}re un module comme un ensemble de fonctions de 
l'application regroup\'{e}es logiquement
(voir plus loin comment constituer cela),
d\'{e}crire l'interface du module consiste \`{a}
sp\'{e}cifier au compilateur un sous-ensemble des fonctions
d\'{e}finies dans le module. Ces fonctions seront dites
{\em export\'{e}es}.
Un module utilisant l'une des fonctions export\'{e}es par un
autre module devra alors {\em importer} le module la fournissant pour 
pouvoir \^{e}tre ex\'{e}cut\'{e}.

La mise en \oe uvre des modules et de leurs relations client-fournisseur
est l'objet principal de ce manuel\footnote{Voir \'{e}galement
le Chapitre 13 du {\em Manuel de R\'{e}f\'{e}rence} de Le-Lisp, pour
une d\'{e}finition exhaustive des modules.}.
L'automatisation presque compl\`{e}te
de cette activit\'{e} est r\'{e}alis\'{e}e par l'{\em Analyseur de Modules}
de \LeLisp.

Mais avant d'\'{e}tudier comment s'en servir nous donnons les
informations \'{e}l\'{e}mentaires sur les modules qui permettent
de poursuivre la lecture du manuel.



\SSection{A propos des modules}

La d\'{e}finition intuitive d'un module a d\'{e}j\`{a} \'{e}t\'{e} donn\'{e}e\,:
il s'agit d'un regroupement de codes \LeLisp\ en vue
de leur compilation modulaire.
Ce regroupement s`accompagne de la d\'{e}finition de
l'interface du module.

Pour sp\'{e}cifier les fichiers du module, l'interface en question et
d'autres informations utiles au compilateur, comme les modules
import\'{e}s, il faut employer une {\em description modulaire}.
Il s'agit d'un fichier d'extension {\tt .lm}, qui comporte
un certain nombre de cl\'{e}s conventionnelles.



\SSSection {Modules interpr\'{e}t\'{e}s}

Commen\c{c}ons par consid\'{e}rer le module
comme un simple regroupement de fichiers source \LeLisp.

Par exemple, un d\'{e}veloppeur a \'{e}crit un programme \LeLisp\ assez
cons\'{e}quent scind\'{e} en trois fichiers {\tt a.ll}, {\tt b.ll} et {\tt
c.ll}. Pour utiliser cette application, il doit explicitement charger
ces trois fichiers (par {\tt loadfile} ou {\tt libloadfile}).

Il peut structurer cette application en un module pour simplifier son
chargement.

Appelons ce module {\tt mymod}. Il suffit de cr\'{e}er le fichier {\tt
mymod.lm} ainsi:

\begin{Code*}
defmodule mymod
files (a.ll b.ll c.ll)
\end{Code*}

Ce fichier {\tt mymod.lm} est le descripteur du module {\tt
mymod}.

Il suffit ensuite d'\'{e}valuer simplement {\tt (loadmodule 'mymod)} pour
charger en une seule fois les trois fichiers mentionn\'{e}s dans la
rubrique {\tt files}: {\tt a.ll}, {\tt b.ll} et {\tt c.ll}.

Le fichiers ainsi charg\'{e}s dans l'environnement constituent une
version interpr\'{e}t\'{e}e de l'application


\SSSection{Imports}

Admettons que l'application en question mette en \oe uvre une interface
minimale comprenant un bouton \Aida.
Le chargement qui vient d'\^{e}tre d\'{e}crit a du sens tel quel si on le
r\'{e}alise apr\`{e}s avoir lanc\'{e} \Aida\ et s'\^{e}tre assur\'{e} que l'ensemble
des fonctionnalit\'{e}s d'\Aida\ que requiert le module {\tt mymod}
est pr\'{e}sent dans l'image m\'{e}moire utilis\'{e}e.

Par contre, l'application ainsi charg\'{e}e
ne pourrait pas fonctionner sans \Aida.

Pour \'{e}viter ce genre d'\'{e}cueil il faut
affiner la d\'{e}finition du module {\tt mymod} en ajoutant des
imports au descripteur du module\,:

\begin{Code*}
defmodule mymod
files (a.ll b.ll c.ll)
import (application button)
\end{Code*}

Si l'on \'{e}value ensuite sous \Aida\ la forme
{\tt (loadmodule 'mymod)}, le
chargeur va d'abord s'assurer que sont pr\'{e}sents en m\'{e}moire les
modules mentionn\'{e}s dans la rubrique {\tt import}\footnote{Et cela
r\'{e}cursivement.} (dans notre exemple
les modules {\tt application} et {\tt button} d'\Aida).

Si l'un de ces deux modules n'est pas pr\'{e}sent en m\'{e}moire, il sera
\'{e}galement charg\'{e} (par un {\tt loadmodule}) ainsi que ses
sous-modules s'ils ne sont pas pr\'{e}sents non plus.

La rubrique {\tt import}, qui est une simple s\'{e}curit\'{e} dans le
cas de cet exemple, devient obligatoire si aucun module
n'est pr\'{e}sent a priori, ce qui est le bon postulat lorsque l'on
fabrique un ex\'{e}cutif.


\SSSection {Modules compil\'{e}s}


L'organisation modulaire des fichiers rev\^{e}t bien entendu
son principal int\'{e}r\^{e}t 
lorsqu'il s'agit de les compiler avec le 
compilateur modulaire {\tt Complice}.

Cette compilation 
(cf. le chapitre 13 du {\em Manuel de r\'{e}f\'{e}rence} \LeLisp\  et le guide de
programmation d'\Aida) g\'{e}n\`{e}re un fichier {\tt mymod.lo} contenant
une suite d'instructions \|LAP| (cf. le Chapitre 12 du {\em Manuel
de R\'{e}f\'{e}rence} de \LeLisp).

Nous avons vu que, quand un module est interpr\'{e}t\'{e}, la fonction {\tt
loadmodule} charge les fichiers indiqu\'{e}s dans la rubrique {\tt
files}.

Quand le module est compil\'{e}, ce ne sont plus les fichiers source
({\tt a.ll}, {\tt b.ll} et {\tt c.ll}) qui sont charg\'{e}s, mais
seulement le fichier {\tt mymod.lo} r\'{e}sultant de leur compilation.

Ce fichier {\tt mymod.lo} contient toutes les fonctions du module
sous forme compil\'{e}e. Ces fonctions ne sont plus charg\'{e}es en m\'{e}moire
sous forme de listes (dans la zone {\tt cons}) mais sous forme de
suites d'instructions en langage machine (dans la zone {\tt code}).

Leur ex\'{e}cution est alors plus rapide et il est possible de cacher
aux autres modules les fonctions strictement internes de ce
module, en ne les mentionnant pas dans l'interface externe du module.

Par exemple, si dans le module {\tt mymod} on ne d\'{e}sire rendre
visible -- on dit g\'{e}n\'{e}ralement {\em exporter} --
que deux fonctions {\tt f} et {\tt g}, il suffit d'ajouter la
ligne suivante au descripteur de module\,:

\begin{Code*}
defmodule mymod
files (a.ll b.ll c.ll)
import (application button)
export (f g)
\end{Code*}



%---------------------------------------------------------------------------
\Section {Introduction \`{a} l'analyse de modules}
%---------------------------------------------------------------------------

\LeLisp\ propose un jeu d'outils qui permettent\,:
\begin{itemize}
\item d'automatiser la mise au point des 
descriptions modulaires\,;
\item de pr\'{e}parer
une compilation de nature incr\'{e}mentale\,;
\item de compiler effectivement, phase \`{a} l'issue
de laquelle les modules objets sont produits.
\end{itemize}
L'utilisation in fine du compilateur est donc l'objectif
principal de ce guide, et de la m\'{e}thodologie qu'il pr\'{e}sente.
Et il s'int\'{e}ressera particuli\`{e}rement
\`{a} la phase permettant de pr\'{e}parer la 
compilation. Nous d\'{e}signerons le plus souvent cette phase
sous le vocable de phase d'{\em analyse} des modules.


Au lieu de r\'{e}aliser manuellement 
la sp\'{e}cification compl\`{e}te des diverses cl\'{e}s des 
descriptions modulaires, en pr\'{e}alable
\`{a} une compilation "manuelle", 
l'emploi de l'{\em Analyseur de Modules} va permettre\,:
\begin{itemize}
\item de constituer int\'{e}gralement les descriptions modulaires, \`{a}
partir d'un d\'{e}coupage des sources de l'application fourni par
l'utilisateur\,;
\item de pr\'{e}parer des scripts de commande de type {\tt Makefile}
pour r\'{e}aliser l'analyse sur l'ensemble des fichiers composant
l'application \,;
\item de r\'{e}aliser une analyse des codes source, lors
de la mise au point des descriptions modulaires, qui permette
de d\'{e}tecter des erreurs dans le d\'{e}coupage en modules (existence
de cycles dans les d\'{e}pendances inter-modules)
ou des lacunes dans les informations donn\'{e}es sur l'application
pour la compiler modulairement 
(manque d'informations syst\`{e}me sur l'acc\`{e}s
aux produits utilis\'{e}s par l'application, manque d'informations
sur l'environnement devant \^{e}tre pr\'{e}sent pendant la compilation,
etc.) \,;
\item de pr\'{e}parer la compilation modulaire en produisant un
script de compilation de type {\tt Makefile}
offrant la possibilit\'{e} de compiler l'ensemble de l'application
incr\'{e}mentalement, c'est \`{a} dire en permettant de
corriger \'{e}ventuellement un \`{a} un les
modules produisant des erreurs de compilation, et de poursuivre
la compilation.
\end{itemize}


\SSection{L'Analyseur de Modules\,: premi\`{e}res notions}

L'{\em Analyseur de Modules} est l'outil
qui permet de produire les
descriptions modulaires et le {\tt Makefile} de compilation \`{a} partir
d'un certain nombre d'informations de base sur l'application.
L'analyseur fonctionne en effectuant plusieurs passes sur les fichiers
sources et les descriptions modulaires de l'application,
et cela jusqu'\`{a} \'{e}limination par
le programmeur, des messages d'erreur et de {\em warning} appelant une
correction.


\SSSection{Usage concret de l'Analyseur de Modules}

L'{\em Analyseur de Modules} se pr\'{e}sente sous la forme d'une commande
du syst\`{e}me d'exploitation, {\tt ll2lm}, 
dot\'{e}e d'un ensemble d'options
permettant de param\'{e}trer de mani\`{e}re d\'{e}taill\'{e}e
la phase d'analyse courante.

L'appel de cette commande permet soit d'analyser directement
un module, soit de produire 
des scripts de commande de type {\tt Makefile}, utilisant
eux-m\^{e}mes la commande {\tt ll2lm}, et 
qui permettront de lancer l'analyse sur tout le projet.

Utiliser l'analyseur revient le plus souvent
\`{a} g\'{e}n\'{e}rer des scripts de commande et \`{a} les lancer.

\begin{Side}{Note\,:}
La derni\`{e}re section du chapitre pr\'{e}cise o\`{u} trouver
l'{\em Analyseur de Modules}, comment l'installer et le
configurer.
\end{Side}


\SSSection{D\'{e}marrer avec l'analyseur}

Dans cette section nous abordons l'utilisation de l'analyseur en 
\'{e}tudiant quelles sont les informations minimales \`{a} fournir pour
pouvoir g\'{e}n\'{e}rer les modules de l'application.

L'application dans sa forme non modulaire est constitu\'{e}e d'un
ensemble de fichiers source qui sont r\'{e}partis dans les
r\'{e}pertoires de d\'{e}veloppement.
En premier lieu, il faut donc sp\'{e}cifier \`{a} l'analyseur quels
sont les fichiers qui constituent l'application et o\`{u} ils
se trouvent.

Mais nous avons d\'{e}j\`{a} expos\'{e} le fait qu'un ensemble de fichiers
source "en vrac" ne constituait pas n\'{e}cessairement un d\'{e}coupage
logique de l'application appropri\'{e} \`{a} sa compilation
et \`{a} sa distribution. 
Un d\'{e}coupage coh\'{e}rent en modules permettra
en effet de limiter l'interface externe des modules
aux fonctions qui sont strictement n\'{e}cessaires \`{a} l'usage externe.
Ce qui est valable pour chaque module est en fait
valable pour l'application elle-m\^{e}me\,: c'est en
d\'{e}coupant l'application en modules que l'on rationalisera
l'interface externe de l'application.


Si l'analyseur peut travailler par d\'{e}faut avec pour seule
base de d\'{e}coupage celle des fichiers source de l'application -- il
g\'{e}n\'{e}rera alors un fichier de description modulaire par
fichier source --, il est beaucoup plus raisonnable de consid\'{e}rer
que la base de travail doit \^{e}tre donn\'{e}e sous
forme d'un ensemble de descriptions modulaires regroupant
les fichiers source.
Ces descriptions modulaires de d\'{e}marrage ne doivent en fait
comporter que les cl\'{e}s {\tt defmodule} et {\tt files}, 
sachant que les autres rubriques ({\tt import} et {\tt export}
notamment) seront remplies par l'analyseur.


\SSection{Projets}


\SSSection{Notion de projet}

Une fois l'application d\'{e}coup\'{e}e (soit en fichiers source si l'on
s'en contente, soit par le biais de descripteurs de modules
de d\'{e}marrage qui sp\'{e}cifient les composantes de l'application)
il faut indiquer \`{a} l'analyseur\,:
\begin{itemize}
\item le d\'{e}coupage lui-m\^{e}me\,;
\item les r\'{e}pertoires o\`{u} trouver les fichiers source et les
descriptions modulaires\,;
\item  les r\'{e}pertoires o\`{u} doivent \^{e}tre produits
les modules, les scripts d'analyse, etc.
\end{itemize}

L'ensemble de ces sp\'{e}cifications, d\'{e}crivant le contexte
d'analyse sont regroup\'{e}es dans ce que l'on appelle
un {\em projet}.

Il s'agit d'un fichier d'extension {\tt .prj} dans lequel sera
\'{e}crite une forme de d\'{e}finition de projet o\`{u} l'on sp\'{e}cifiera
les informations utiles pour analyser l'application.
En voici un exemple extr\^{e}mement simple\,:
\BeginLL
(define-rt-project myappli
   root-directory #u"/udd/local/myappli/" 
   directories (#u"sources/")
   modules-lists ("myappli.lst") 
\EndLL

La forme ci-dessus d\'{e}finit le projet {\tt myappli}
associ\'{e} \`{a} "mon" application. Le projet est d\'{e}fini dans
le fichier {\tt myappli.prj}.
L'ensemble des r\'{e}pertoires li\'{e}s \`{a} l'application
se trouvent sous la racine \|/udd/local/myappli|. Les fichiers
sources sont dans le r\'{e}pertoire {\tt sources} (sous le r\'{e}pertoire
racine). 
Le d\'{e}coupage en modules de l'application est donn\'{e} dans le fichier
{\tt myappli.lst} qui se trouve dans le r\'{e}pertoire mentionn\'{e} dans la
cl\'{e} {\tt directories}\footnote{Si plusieurs r\'{e}pertoires de sources
existent et sont mentionn\'{e}s dans cette cl\'{e}, il faut en fait
pr\'{e}voir dans chacun d'eux un fichier du type de {\tt myappli.lst}.}.

Le projet rassemble donc en premier lieu
toutes les informations ayant trait\,:
\begin{itemize}
\item \`{a} l'application elle-m\^{e}me, en tant qu'un ensemble
de modules (ou de fichiers, \`{a} d\'{e}faut). L'option concern\'{e}e
est {\tt modules-lists}\,;
\item \`{a} la localisation des ressources constituant l'application
-- syst\`{e}me de r\'{e}pertoires relatifs \`{a} un r\'{e}pertoire racine --.
Les principales options concern\'{e}es sont {\tt root-directory}
et {\tt directories}.
\end{itemize}

Bien entendu un certain nombre d'autres informations utiles
\`{a} l'analyseur peuvent ou doivent \^{e}tre sp\'{e}cifi\'{e}es via le projet\,:
o\`{u} sont g\'{e}n\'{e}r\'{e}s les modules et leur descriptions, quels sont
les noms que l'on souhaite donner aux fichiers de script g\'{e}n\'{e}r\'{e}s,
etc.

L'objectif n'est pas de d\'{e}tailler ici toutes ces options\,:
les plus courantes d'entre-elles sont pr\'{e}sent\'{e}es dans le
Chapitre 2 et l'ensemble des options est document\'{e}
dans le Chapitre 5.



\SSSection{Projets et produits}

Dans ce premier contact avec la notion de projet,
nous avons pass\'{e} sous silence la fa\c{c}on dont
l'analyseur pouvait g\'{e}rer les imports dans le cas o\`{u} des
modules externes \`{a} l'application {\tt myappli} \'{e}taient
n\'{e}cessaires pour qu'elle s'ex\'{e}cute.
Supposons en effet que l'application {\tt myappli} comporte
le module {\tt mymod} -- pr\'{e}sent\'{e} ci-dessus dans l'exemple
sur les modules -- qui utilise des fonctions appartenant
\`{a} \Aida\ -- les imports devant \^{e}tre produits \'{e}tant
{\tt application} et {\tt button} --.
Avec le projet {\tt myappli} d\'{e}fini
au paragraphe pr\'{e}c\'{e}dent, l'analyseur n'aura pas les
moyens de fabriquer la liste des imports n\'{e}cessaires pour
compiler et charger le module {\tt mymod}.

En d'autres termes, si l'application \`{a} analyser utilise
une autre application ou un produit tel qu'\Aida, il faut le
sp\'{e}cifier \`{a} l'analyseur.

Pour ce faire, il faut employer la cl\'{e} {\tt required-projects}
de la fa\c{c}on suivante\,:
\BeginLL
(define-rt-project myappli
   required-projects (aida)
   root-directory #u"/udd/local/myappli/" 
   directories (#u"sources/")
   modules-lists ("myappli.lst") 
\EndLL

La forme ci-dessus sp\'{e}cifie que le projet {\tt myappli} 
requiert la pr\'{e}sence du projet {\tt aida} lors de l'analyse. 
Exprimer le fait qu'une application est cliente d'une autre
application devient ainsi particuli\`{e}rement ais\'{e}.

L'ensemble des produits d'\Ilog\ bas\'{e}s sur \LeLisp\ sont dot\'{e}s
d'une description de projet et du fichier correspondant\,: 
{\tt lelisp.prj},
{\tt aida.prj},
{\tt smeci.prj},
etc.,
ceux-ci pouvant \^{e}tre directement r\'{e}f\'{e}renc\'{e}s
dans les projets des applications.



\SSection{Organisation informatique de l'application}

Avant de conclure ce chapitre d'introduction par le suivi
des op\'{e}rations d'analyse et de compilation de A \`{a} Z, nous
pr\'{e}cisons l'organisation informatique de l'application qui
apparait la plus souhaitable pour mener \`{a} bien toutes
ces \'{e}tapes.

\SSSection{R\'{e}partition des sources et des modules}

Il est souhaitable de s\'{e}parer les sources de l'application
des modules (y compris les descriptions modulaires 
mise \`{a} jour par l'{\em Analyseur de Modules})\,:
\begin{itemize}
\item la livraison de l'application \`{a} ses utilisateurs en
sera facilit\'{e}e\,: seuls les modules devront \^{e}tre livr\'{e}s,
\`{a} moins qu'on ne livre qu'un ex\'{e}cutif \,;
\item la s\'{e}paration entre fichiers sources contr\^{o}l\'{e}s par
le programmeur et fichiers de travail modifi\'{e}s par l'analyseur
est g\'{e}n\'{e}ralement appr\'{e}ciable.
On notera d'ailleurs qu'il existe un moyen de maintenir
des descriptions modulaires d'extension diff\'{e}rente
de {\tt .lm} (on utilise alors souvent {\tt .lc}).
Ces descriptions modulaires peuvent \^{e}tre conserv\'{e}es avec les
fichiers sources comme sp\'{e}cification du d\'{e}coupage de
l'application en modules.
\end{itemize}


\SSSection{R\'{e}pertoire d'analyse}

Les fichiers de projets et les scripts g\'{e}n\'{e}r\'{e}es par l'analyseur
sont en g\'{e}n\'{e}ral group\'{e}s dans un r\'{e}pertoire sp\'{e}cifique,
distinct des sources et des modules.
Ce r\'{e}pertoire est appel\'{e} {\tt modana} dans la distribution des
produits \Ilog. Il sera r\'{e}f\'{e}renc\'{e} dans la d\'{e}finition de projet
en utilisant la cl\'{e} {\tt system-directory}, cf. Chapitre 5.

Il faut noter que pour ses besoins propres l'analyseur produit des
fichiers correspondant \`{a} une {\em table de r\'{e}f\'{e}rence} interne,
dont l'usage et la mise \`{a} jour ne concernent pas l'utilisateur
en premi\`{e}re instance.
Par contre, l'emplacement de ces fichiers peut-\^{e}tre param\'{e}tr\'{e}
(cf. l'option {\tt crunch-directory} de la d\'{e}finition de projet)
et se trouve \^{e}tre conventionnellement le r\'{e}pertoire 
{\tt modana}.

\begin{Side}{\bf Remarque}
Les fichiers d'extension {\tt .lst} associ\'{e}s \`{a} un projet et
donnant la liste des modules de l'application
(cf. {\tt myappli.lst} dans l'exemple des pages pr\'{e}c\'{e}dentes) sont 
obligatoirement plac\'{e}s avec les fichiers sources,
rendus accessibles via la cl\'{e} {\tt directories}.
\end{Side}



\SSSection{R\'{e}sum\'{e}}

L'organisation conventionnelle que nous proposons pour le
d\'{e}veloppement et la compilation des applications est
-- \|<root-dir>| correspondant au r\'{e}pertoire racine de 
l'application -- \,:
\begin{itemize}
\item \|<root-dir>/sources|\,: fichiers source (avec \'{e}ventuellement
les descriptions modulaires "source" {\tt .lc}) de l'application\,;
\item \|<root-dir>/modules|\,: descriptions modulaires produites
par l'analyseur {\tt .lm} et fichiers objets {\tt .lo} produits
par le compilateur\,;
\item \|<root-dir>/modana|\,: r\'{e}pertoire o\`{u} sont plac\'{e}s
les projets et les scripts (de type {\tt Makefile})
produits pour l'analyse ou la compilation.
\end{itemize}


%---------------------------------------------------------------------------
\Section {M\'{e}thodologie\,: de l'analyse \`{a} la compilation, en 6 \'{e}tapes}
%---------------------------------------------------------------------------

Cette section pr\'{e}sente le d\'{e}roulement de la g\'{e}n\'{e}ration
des modules de l'application {\tt myappli}.

On peut r\'{e}sumer le cycle de compilation de l'ensemble des modules
d'un projet, \`{a} 3 phases principales\,: 
\begin{enumerate}
\item la pr\'{e}paration du projet\,;
\item l'analyse\,;
\item la compilation.
\end{enumerate}

Chacune de ces phases peut se d\'{e}composer. 
La pr\'{e}paration du projet se fait elle m\^{e}me en deux ou trois
\'{e}tapes et la phase de compilation se fait en deux temps.
Cela donne en fait 6 \'{e}tapes\,:
\begin{itemize}
\item {\em description du projet} (en utilisant la forme
{\tt define-rt-project} d\'{e}j\`{a} introduite)\,;

\item si besoin, c'est \`{a} dire quand il manque des descriptions
modulaires ou quand il n'y en a pas,
r\'{e}alisation d'une {\em analyse de d\'{e}marrage}\,;
\begin{itemize}
\item fabrication du {\tt Makefile} d'{\em analyse de d\'{e}marrage}\,;
\item r\'{e}alisation des deux passes correspondant \`{a}
l'{\em analyse de d\'{e}marrage}\,;
\end{itemize}

\item fabrication du {\tt Makefile} d'{\em analyse courante}\,;

\item {\em analyse} proprement dite\,: elle correspond
\`{a} l'emploi incr\'{e}mental du script d'analyse courante,
avec correction du contenu des modules pour lesquels
sont d\'{e}tect\'{e}es des erreurs\,;

\item fabrication du {\tt Makefile} de {\em compilation}\,;

\item {\em compilation} effective des modules. 
\end{itemize}

Le d\'{e}roulement de A \`{a} Z d'une session de g\'{e}n\'{e}ration de modules
comporte au maximum les 6 \'{e}tapes importantes list\'{e}es ci-dessus,
sachant que l'\'{e}tape 2 (r\'{e}alisation d'une analyse de d\'{e}marrage)
est optionnelle et d\'{e}pend des exigences que l'on se donne
en mati\`{e}re de gestion modulaire pour une application donn\'{e}e.



%---------------------------------------------------------------------------
\SSection{1\`{e}re \'{e}tape: description du projet}
%---------------------------------------------------------------------------

Pour d\'{e}crire le projet (ou \'{e}ventuellement les projets)
associ\'{e}(s) \`{a} l'application,
il faut utiliser une (ou \'{e}ventuellement plusieurs)
forme(s) de d\'{e}finition de projet {\tt define-rt-project}. 

La d\'{e}finition d'un projet sera stock\'{e}e dans un fichier de
description de nom conventionnel\,: \|<nom de projet>.prj|. 
Ce fichier rassemblera en fait toutes les informations n\'{e}cessaires
au projet et notamment les instructions de chargement 
d'autres projets quand ils sont requis.
On \'{e}crira alors directement une forme de chargement,
comme dans l'exemple ci-dessous o\`{u} le projet {\tt aida} est requis
et doit \^{e}tre pr\'{e}alablement charg\'{e}\,:
\begin{Code*}
(loadfile "/usr/ilog/aida/modana/aida.prj")

(define-rt-project myappli
   required-projects (aida)
   root-directory #u"/udd/local/myappli/" 
   directories (#u"sources/")
   modules-lists ("myappli.lst")
   system-directory #u"modana/"
   crunch-directory #u"modana/")
\end{Code*}


Par d\'{e}faut, c'est \`{a} dire si l'option {\tt modules-lists}
n'est pas employ\'{e}e, chaque fichier source (de la forme {\tt
<file>.ll}) de chaque r\'{e}pertoire pr\'{e}sent sous la clef \|directories|
va donner lieu \`{a} un fichier de description modulaire
(de la forme {\tt <file>.lm}).

Dans l'exemple ci-dessus, l'on pr\'{e}f\`{e}re r\'{e}f\'{e}rencer 
explicitement une liste de modules d\'{e}j\`{a} sp\'{e}cifi\'{e}s par
des descriptions modulaires indiquant les fichiers source
correspondant \`{a} chaque module -- cf. la r\'{e}f\'{e}rence au
fichier \|myappli.lst| --.

\medskip 

{\em Pour le d\'{e}tail, cf. Chapitre 2 pour d\'{e}marrer et Chapitre 5
comme Manuel de R\'{e}f\'{e}rence}.

%------------------------------------------------------------------
\SSection{2\`{e}me \'{e}tape: analyse de d\'{e}marrage}
%------------------------------------------------------------------


Dans le cas o\`{u} il manque des descriptions modulaires ou quand il 
n'y en a pas, c'est \`{a} dire quand on veut les fabriquer \`{a} partir
des fichiers source, l'analyse ne peut pas r\'{e}ellement commencer
avant de les avoir constitu\'{e}s.
Il faut donc effectuer une sorte d'analyse \`{a} blanc, appel\'{e}e
analyse de d\'{e}marrage et qui, comme toute analyse, peut
s'effectuer \`{a} partir d'un fichier de commande de type
{\tt Makefile}.


\SSSection{Fabrication du Makefile d'analyse de d\'{e}marrage}

Pour fabriquer le script de commande de l'analyse de d\'{e}marrage
il faut utiliser la commande {\tt ll2lm} avec l'option
de g\'{e}n\'{e}ration du makefile d'analyse (cette option est \|-init|) 
et avec la valeur par d\'{e}faut de l'option
de gestion du niveau des d\'{e}pendances du makefile
(il s'agit de l'option \|-dependency| dont la valeur
doit imp\'{e}rativement rester \`{a} 1).
Il faut donc lancer la commande, une fois plac\'{e} dans le r\'{e}pertoire
racine mentionn\'{e} dans le projet\,:
\begin{Code*}
% cd /udd/local/myappli
% ll2lm -load modana/myappli.prj -p myappli -init
\end{Code*}
L'option \|-load| indique quel fichier de projet charger et
l'option \|-project| (que l'on peut raccourcir en \|-p|)
indique quel projet doit \^{e}tre analys\'{e}.

L'on obtient ainsi un script d'analyse de nom {\tt myappli.mki}
dans le r\'{e}pertoire d'analyse (mentionn\'{e} dans le projet via la
cl\'{e} {\tt system-directory}).


\SSSection{Analyse de d\'{e}marrage}

L'analyse de d\'{e}marrage se fait en lan\c{c}ant le script 
{\tt myappli.mki} g\'{e}n\'{e}r\'{e} \`{a} l'\'{e}tape pr\'{e}c\'{e}dente, et
ceci deux fois\,:

\begin{itemize}

\item la premi\`{e}re fois avec l'entr\'{e}e {\tt init1}\,:
\begin{Code*}
% make -f modana/myappli.mki init1
\end{Code*}
Cette premi\`{e}re passe permet de fabriquer les tables de r\'{e}f\'{e}rence
internes de l'analyseur (cf. \|modana/*.ref|),
\`{a} partir d'une premi\`{e}re version incompl\`{e}te
des descriptions modulaires. 

La plupart des messages et avertissements de cette \'{e}tape ne sont dus
qu'au fait que l'{\em Analyseur de Modules} ne conna\^{\i}t pas encore tous
les modules\,: de nombreuses fonctions et autres d\'{e}finitions sont
inconnues et introuvables. C'est pourquoi on ne tiendra pas compte des
messages de cette \'{e}tape. 

Si toutefois des erreurs bloquantes surviennent,
alors on devra revoir l'ordre d'analyse des fichiers, pour
pouvoir poursuivre.


\item la seconde fois avec l'entr\'{e}e {\tt init2}\,:
\begin{Code*}
% make -f modana/myappli.mki init2
\end{Code*}

Cette seconde passe se charge de refabriquer tous les modules, 
comme s'ils n'existaient pas, mais en ayant cette fois connaissance
de l'environnement complet d'analyse.
Les messages (de type {\tt warning} ou {\tt error}) 
affich\'{e}s sont d\'{e}sormais tout \`{a} fait significatifs et doivent \^{e}tre
exploit\'{e}s par le programmeur pour corriger le contenu des modules
ou pour modifier le d\'{e}coupage de l'application.
Ce travail de mise au point correspond aux phases suivantes dites
d'{\em analyse courante}.
\end{itemize}


%------------------------------------------------------------------
\SSection{3\`{e}me \'{e}tape: fabrication du script d'analyse courante}
%------------------------------------------------------------------

Une fois effectu\'{e}e l'analyse de d\'{e}marrage, ou dans le cas o\`{u}
l'ensemble des descriptions modulaires de l'application
avait \'{e}t\'{e} d\^{u}ment d\'{e}fini \`{a} la main, 
l'on fabrique un script d'analyse
courante qui servira pour r\'{e}aliser toutes les analyses ult\'{e}rieures.

On lance pour ce faire la commande {\tt ll2lm} avec l'option
d'analyse \|-init| et avec l'option de niveau de d\'{e}pendance
{\tt -dep 2} afin d'\'{e}tablir une d\'{e}pendance sur les modules 
import\'{e}s\,:

\begin{Code*}
% ll2lm -load myappli.prj -p myappli -init -dep 2
\end{Code*}


%-------------------------------------------------------------------
\SSection{4\`{e}me \'{e}tape: analyse courante}
%-------------------------------------------------------------------

L'analyse courante se lance en utilisant le {\tt Makefile}
g\'{e}n\'{e}r\'{e} \`{a} l'\'{e}tape pr\'{e}c\'{e}dente sans entr\'{e}e de type
{\tt init1} ou {\tt init2}\,:
\begin{Code*}
% make -f modana/myappli.mki
\end{Code*}

D\'{e}bute alors une s\'{e}rie de cycles analyse / corrections jusqu'\`{a}
ce qu'aucun message grave, c'est \`{a} dire pr\'{e}c\'{e}d\'{e} de la marque
{\tt **}, ne soit plus d\'{e}livr\'{e} par l'analyseur.

Si certaines corrections imposent de modifier la d\'{e}finition du projet
on reprendra les \'{e}tapes 1 \`{a} 4. Il suffit par contre de relancer l'\'{e}tape 4
quand les listes d'{\tt import} sont modifi\'{e}es, chaque fois qu'une
d\'{e}pendance circulaire est signal\'{e}e ou qu'une redondance d'{\tt
import} est mentionn\'{e}e.

Au cas o\`{u} l'on exploite les fichiers {\tt .lc} comme description
modulaire source, on devra \'{e}diter directement ces fichiers pour 
corriger les causes des messages d'erreur.

\medskip 

{\em Pour le d\'{e}tail, cf. Chapitre 2 pour d\'{e}marrer et Chapitre 5
comme Manuel de R\'{e}f\'{e}rence. Le Chapitre 4 documente l'ensemble des
messages d'analyse.}


%----------------------------------------------------------------
\SSection{5\`{e}me \'{e}tape: fabrication du script de compilation}
%----------------------------------------------------------------

Quand l'{\em Analyseur de Modules} n'\'{e}met plus de message supposant
une correction, la phase de pr\'{e}paration de la compilation peut \^{e}tre
achev\'{e}e en g\'{e}n\'{e}rant un {\tt Makefile} de compilation.

L'option \|-makefile| de la commande d'analyse permet d'engendrer
le {\tt Makefile} \`{a} partir de la table de r\'{e}f\'{e}rences du projet \,:

\begin{Code*}
% ll2lm -load modana/myappli.prj -p myappli -makefile -dep 2
\end{Code*}

Le fichier g\'{e}n\'{e}r\'{e} est nomm\'{e} par d\'{e}faut {\tt myappli.mk}
et se trouve dans le r\'{e}pertoire d'analyse 
(mentionn\'{e} dans le projet via la cl\'{e} {\tt system-directory}).


Ce {\tt Makefile} devra 
\^{e}tre reg\'{e}n\'{e}r\'{e} si de nouveaux modules sont ajout\'{e}s au projet, ou
si la d\'{e}finition du projet a \'{e}t\'{e} modifi\'{e}e.
Un certain nombre de clefs de la forme de d\'{e}finition de projets,
telles que \|directories|, \|ll-object-directory| (ainsi que
\|complice-options| qui concerne directement le compilateur,
cf. Chapitre 3) influent sur le contenu du {\tt Makefile}
de compilation.
Il en va de m\^{e}me pour la cl\'{e} \|-dependency| de la commande d'analyse
qui sp\'{e}cifie \'{e}galement le niveau de d\'{e}pendance utilis\'{e}
pour op\'{e}rer les {\tt make}.

\medskip 

{\em Pour le d\'{e}tail, cf. Chapitres 3 et 5.}


%-----------------------------------------------------------------
\SSection{6\`{e}me \'{e}tape: compilation}
%-----------------------------------------------------------------

La derni\`{e}re \'{e}tape consiste \`{a} r\'{e}aliser la compilation de l'ensemble
des modules de l'application, puis, par la suite, la mise \`{a} jour
des modules compil\'{e}s.

Il suffit en g\'{e}n\'{e}ral de lancer le {\tt Makefile} de compilation,
sans pr\'{e}ciser d'entr\'{e}e  particuli\`{e}re\,:
\begin{Code*}
% make -f modana/myappli.mk
\end{Code*}

On utilisera l'entr\'{e}e pr\'{e}d\'{e}finie {\tt clean} pour effacer tous les
fichiers objets ({\tt .lo}) et l'entr\'{e}e {\tt i} pour \^{e}tre sous la
boucle interactive du compilateur (environnement de compilation
charg\'{e}).

Il se peut que {\tt complice} d\'{e}c\`{e}le de nouvelles erreurs lors de la
compilation des modules. On devra alors corriger ces erreurs et sans
doute recommencer les {\em \'{e}tapes 4} \`{a} {\em 6}.


{\em Pour le d\'{e}tail, cf. Chapitre 3 du pr\'{e}sent manuel et le Manuel
de Ref\'{e}rence de} \LeLisp, {\em Chapitre 13.}


\bigskip

\begin{Side}{Note}
{\em L'annexe A pr\'{e}sente de mani\`{e}re compl\`{e}te l'ensemble des
op\'{e}rations d'analyse et de compilation sur un exemple tr\`{e}s simple,
\'{e}quivalent \`{a} celui propos\'{e} dans cette section,
en reproduisant l'int\'{e}gralit\'{e} des messages obtenus pendant 
les diverses sessions.}
\end{Side}



%---------------------------------------------------------------------------
\Section{Installation de l'analyseur et configuration}
%---------------------------------------------------------------------------

L'installation de l'analyseur n'est pas obligatoire pour pouvoir
l'utiliser. En effet la commande {\tt ll2lm} est livr\'{e}e 
en standard avec le produit \LeLisp. 
Elle se trouve dans le r\'{e}pertoire {\tt <machine>} au m\^{e}me
titre que les autres commandes de \LeLisp\ et est directement
utilisable. La configuration standard des tailles des zones de {\tt
ll2lm} est normalement suffisante dans la majorit\'{e} des cas. 

Toutefois, si durant une phase d'analyse vous obtenez des messages
libell\'{e}s de la mani\`{e}re suivante\,:

\begin{Code*}
***** Fatal error : no room for symbols.
***** Fatal error : no room for code.
\end{Code*}

il devient n\'{e}cessaire de reconfigurer l'image m\'{e}moire 
{\tt cll2lm}\footnote{Il ne s'agit pas
d'une erreur de documentation\,: le nom de l'image m\'{e}moire est
effectivement {\tt cll2lm} et non {\tt ll2lm}.}
utilis\'{e}e par {\tt ll2lm} en augmentant la zone concern\'{e}e -- celle 
des symboles et du code pour le message pr\'{e}c\'{e}dent.

Voici la d\'{e}marche \`{a} suivre\,:

\begin{itemize}
\item aller dans le r\'{e}pertoire machine de la distribution \LeLisp\,:
\begin{Code*}
% cd /usr/ilog/lelisp/<machine>
\end{Code*}
\item d\'{e}truire la commande {\tt cll2lm}\,:
\begin{Code*}
% /bin/rm cll2lm
\end{Code*}
\item reconfigurer une image m\'{e}moire {\tt cll2lm}. Dans
l'exemple suivant, l'image est reconfigur\'{e}e avec une zone des
symboles \`{a} 25\,:
\begin{Code*}
% make cll2lm SYMBOL_A=25 CODE_A=2000
\end{Code*}
La commande {\tt make} ci-dessus fait appel \`{a} la variable {\tt
\|SYMBOL_A|} pour reconfigurer la zone des {\em symbol}, d'une
mani\`{e}re  g\'{e}n\'{e}rale, on reconfigurera une zone {\em xxx} en utilisant
la variable {\tt \|XXX_A|}.
\end{itemize}

\begin{Side}{\bf Remarque} 
Pour conna\^{\i}tre la taille des zones standard de la commande {\tt
cll2lm}, consulter le fichier {\tt cll2lm}.
\begin{Code*}
% more /usr/ilog/lelisp/<machine>/cll2lm
\end{Code*}
\end{Side}
