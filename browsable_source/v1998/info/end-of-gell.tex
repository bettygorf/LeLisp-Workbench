\documentstyle[11pt,fr]{iarticle}

\begin{document}
\frenchspacing
\setlength{\parindent}{0in}
\setlength{\parskip}{3ex}
\raggedbottom

\Title {Gell}
\SuperTitle {portage Gell/Portage Natif \\ La transition sur HP9700}

\Author {janvier 1994}
\bigskip

Cher client, \\
Vous \^{e}tes possesseur d'une HP9700, et 
vous avez eu \`{a} travailler jusqu'\`{a} ce jour avec les produits Ilog
dont le portage ce nommait {\tt Chp9700}, correspondant \`{a}
l'utilisation de la technologie {\it GELL}. Nous vous proposons
d'\'{e}voluer vers notre nouveau portage des m\^{e}mes produits, de
technologie {\it NATIVE}, nomm\'{e} {\tt hp9700} {\bf plus simple}
d'utilisation et {\bf plus performant}. 
\par

\Section {Rappel: Technologie GELL}
La technologie {\it GELL} consistait \`{a} prendre du code
\LeLisp\ compil\'{e} ({\tt file.lo}), \`{a} le traduire en C ({\tt file.c,
file.h}), puis \`{a} compiler ce code C ({\tt file.o}), \`{a} linker ce code
C compil\'{e} avec les librairies des produits Ilog ({\tt LlibO.o,
AidaO.o, ...}) afin d'optenir un binaire \LeLisp\ ({\tt leAIX11bin,
...}). Un outil nomm\'{e} {\tt configurator} vous aidait dans cette d\'{e}marche.
Il convenait ensuite d'\'{e}xecuter ce binaire, puis d'\'{e}laborer
les fonctions \LeLisp\ correspondant au code C compil\'{e} ({\tt
file.le}), et de sauver le tout dans une image m\'{e}moire. Le document
intitul\'{e} {\tt
gen\'{e}rateur de runtime sur HP9700} vous aidait dans cette d\'{e}marche.
\par
L'avantage de cette technique \'{e}tait essentiellement la
portabilit\'{e}. Les inconv\'{e}nients \'{e}taient principalement la
complexit\'{e} de la mise en \oe uvre (cf ci-dessus), le cycle de
d\'{e}veloppement (pas de chargement dynamique), et les performances
d\'{e}grad\'{e}es par rapport \`{a} un portage natif.

\Section{Technologie NATIVE}
Peut-\^{e}tre disposiez-vous d'autres portages des produits Ilog (sun4,
decstation, ...) auquel cas la simplicit\'{e} d'utilisation d'un portage
NATIF en regard d'un portage GELL vous parait \'{e}vidente. 
Si vous ne disposez que du portage des produits Ilog sur Chp9700
(GELL), vous 
d\'{e}couvrirez un monde plus simple avec le portage ds m\^{e}mes produits
sur HP9700 (NATIF): la phase strictement lisp reste la m\^{e}me, jusqu'\`{a}
la compilation (optention des fichiers {\tt file.lo}). On passe alors
directement au lancement d'un binaire minimal fourni avec la bande de
distribution de \LeLisp , dans lequel on charge les fichiers lisp compil\'{e}s
({\tt <Crtl>A file.lo}). Il existe toujours un {\tt g\'{e}n\'{e}rateur de runtime}
pour vous aider dans cette phase.
L'outil {\tt configurator} a disparu, ainsi que la plupart des
{\tt makefile}. Il ne subsite en effet qu'un seul makefile ({\tt
lelisp/hp9700/Makefile}) permettant de refabriquer des binaires lisp,
et les fabricateurs d'images m\'{e}moire {\tt makeaida, makesmeci, ...}.
\par
Comme nous venons de le voir, le code \LeLisp\ peut maintenant \^{e}tre
charg\'{e} dynamiquement en compil\'{e} ({\tt file.lo}) \`{a} l'aide de la
fonction {\tt loadmodule} (ou du caract\`{e}re sp\'{e}cial {\tt <Crtl>A}). Il est
possible \'{e}galement de compiler ``\`{a} 
la vol\'{e}e {''} avec les fonctions {\tt compile, compilemodule,
compile-all-in-core}. Si vous d\'{e}veloppez du code C en plus du code
\LeLisp\ , vous n'aurez qu'\`{a} compiler ce code C, puis \`{a} le charger
dynamiquement \`{a} l'aide de la fonction {\tt cload} (il est \'{e}galement
possible de linker statiquement ces fichiers C avec le binaire
\LeLisp\,: cf document de configuration de \LeLisp ).
\par
En plus de la simplicit\'{e} d'emploi, il vous sera facile de constater
une am\'{e}lioration des performances d'un facteur 2 \`{a} 2,5.

\Section {Conclusion}
L'introduction de la technologie NATIVE sur HP9700 a pour but de
simplifier le travail des d\'{e}veloppeurs, et de leur permettre de
produire plus rapidement encore leurs applications, et d'augmenter la
rapidit\'{e} de celles-ci.
\par
Ilog, et toute son \'{e}quipe de developpeurs vous demande
de les excuser pour les inconv\'{e}nients que pourrait entrainer cette
\'{e}volution.

\tableofcontents

\end{document}




