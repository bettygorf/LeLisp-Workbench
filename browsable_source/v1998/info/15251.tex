\documentstyle{ilogmanuel}


\Begin
\Title {Diffusion Le-Lisp}
\SuperTitle {Le-Lisp  de  l'INRIA  version 15.25}

\Author {novembre 1991}

                 
                            
\chapter {Version "15.25"}
Voici les corrections et les extensions de la nouvelle diffusion de
\LeLisp\ version 15.25 dat\'{e}e du 17 novembre 1991.

Une fois par an, l'INRIA et ILOG livrent aux porteurs une nouvelle
version de Le-Lisp comprenant des am\'{e}liorations, extensions et
corrections.  Seule cette diffusion annuelle contient des changements
\`{a} la souche de Le-Lisp \'{e}crite en LLM3, et au manuel de l'utilisateur.

\Section {Contenu de cette diffusion}

La bande de diffusion se d\'{e}compose selon les r\'{e}pertoires suivants:

\begin {itemize}
\item   68k : l'expanseur commun \`{a} tous les 680x0;
\item   benchmarks : pour r\'{e}aliser des mesures de performances des syst\`{e}mes Le-Lisp;
\item   benchmarks/Results : les r\'{e}sultats de ces mesures de performances;
\item   bignum : le module BIGNUM de DEC/INRIA;
\item   bignum/l : l'interface des bignum avec Le-Lisp;
\item   common : les fichiers C communs \`{a} tous les portages utilisant C;
\item   info : ce fichier;
\item   llib : la librairie Le-Lisp;
\item   llm3 : le noyau de l'interpr\`{e}te Le-Lisp;
\item   llmod : la description modulaire de tous les fichiers des
librairies Le-Lisp;
\item   llobj : le r\'{e}sultat de la compilation modulaire de tous les
fichiers des librairies Le-Lisp;
\item   lltest : les jeux de tests pour tester, verifier, recetter Le-Lisp;
\item   lltool : quelques outils utiles aux porteurs;
\item   llub : une librairie suppl\'{e}mentaire, remplie par les
utilisateurs, non maintenue;
\item   manl : un manuel UNIX des principales commandes pour Le-Lisp;
\item   manuel : le manuel de r\'{e}f\'{e}rence en francais, au format {\tt tcomp};
\item   sun : le syst\`{e}me Sun3;
\item   sun/C : les fichiers C sp\'{e}cifiques \`{a} Sun3;
\item   sun/conf : les configurateurs d'images Le-Lisp sp\'{e}cifiques \`{a} Sun3;
\item   sun/recette : les commandes permettant de lancer les recettes
sur Sun3;
\item   vaxunix : le syst\`{e}me Vax Ultrix;
\item   vaxunix/C : les fichiers C sp\'{e}cifiques \`{a} Vax;
\item   vaxunix/conf : les configurateurs d'images Le-Lisp sp\'{e}cifiques \`{a} Vax;
\item   vaxunix/recette : les commandes permettant de lancer les recettes
sur Vax; 
\item   virbitmap : les incarnations de BV et leurs chargeurs;
\item   virbitmap/bvtty : l'incarnation du BV sur un {\tt tty}
alphanum\'{e}rique simple;
\item   virbitmap/X11 : l'incarnation du BV sur X11;
\item   virtty : les descriptions de {\tt tty} d\'{e}j\`{a} rencontr\'{e}s par
le-Lisp. 
\end {itemize}

Le contenu exact de la bande de diffusion est explicit\'{e}e dans les
fichiers {\tt LLUSERFILES}, {\tt LLPORTFILES} et {\tt LLPORTMJFILES}
ainsi que dans les fichiers {\tt UserFILES} et {\tt PortFILES} dans
les r\'{e}pertoires des machines.

\Section{Sp\'{e}cifications de la version 15.25}

On rappelle dans cette section, tous les travaux d\'{e}j\`{a} effectu\'{e}s
dans la version 15.24.2. Cela offre une meilleur id\'{e}e des
diff\'{e}rences entre les deux versions majeures. La version concern\'{e}e est
signal\'{e}e en t\^{e}te de chaque paragraphe.

\begin {enumerate}

\item Intervention d'ordre g\'{e}n\'{e}ral sur le code \LeLisp\ :

\begin {itemize}

\item {\bf 15.24.2} \\
Un gros travail a \'{e}t\'{e} accompli sur le module PATHNAME,
aussi bien en correction d'anomalies qu'en ajout de s\'{e}mantique,
en particulier le probl\`{e}me de la longueur des noms de fichiers (sur
Sys5 et sur OS/2). On notera la cr\'{e}ation d'un outil de
d\'{e}tection de conflit de noms: {\tt lelisp/llub/verifpath.ll}.
On notera \'{e}galement certaines corrections telles que:

\begin {enumerate}

\item {\bf 15.24.2} \\
``//{''} sur machine {\tt apollo} affecte le champ DEVICE de la structure
PATHNAME. 

\item {\bf 15.25} \\
EXPAND-PATHNAME peut rendre  
un r\'{e}sultat aussi long que n\'{e}cessaire (il n'y a plus de
limitation de buffer). Introduction par la m\^{e}me occasion de la
fonction MAP-EXPAND-PATHNAME, \'{e}vitant ainsi de co\^{u}teux r\'{e}sultats
interm\'{e}diaires inutiles. Cela n\'{e}cessite pas mal de modification des
codes C et lisp plus de nouveaux tests .

\item {\bf 15.25} \\
le module PATH est compl\`{e}tement impl\'{e}ment\'{e} en OS/2.

\item {\bf 15.25} \\
la fonction CONTROL-FILE-PATHNAME est revue et corrig\'{e}e afin de
rendre les valeurs ad\'{e}quates, selon le produit nomm\'{e} et le syst\`{e}me
utilis\'{e}. 
L'algorithme de nommage du fichier d'initialisation d'un produit {\tt
toto} est le suivant, selon le syst\`{e}me d'exploitation:
\begin{itemize}
\item os2: {\tt \$HOME\|\|toto.ini}
\item unix: {\tt \$HOME/.toto}
\item vaxvms: {\tt sys\$login:.toto}
\end {itemize}
Dans tous les cas, le deuxi\`{e}me argument \'{e}ventuellement fourni \`{a} la
fonction CONTROL-FILE-PATHNAME permet de remplacer ce nom par d\'{e}faut. 

\end{enumerate}

\item {\bf 15.24.2} \\
Changement de l'ordre des r\'{e}pertoires dans la variable
\#:SYSTEM:PATH.
Le r\'{e}pertoire {\tt system/} est plac\'{e} devant les librairies.
Cela permet de mieux g\'{e}rer l'introduction de nouveaux portages en
cours d'ann\'{e}e, lesquels n\'{e}cessitent parfois une intervention locale dans
un fichier de la librairie. 

\item {\bf 15.24.2} \\
Reprise de certaines bizareries autour des longueurs des buffers
d'entr\'{e}e sortie (harmonisation LL/C, mise \`{a} niveau de stringio.ll).

\item {\bf 15.24.2} \\
Utilisation de DEFMACRO-OPEN sur MAPHASH dans le module {\tt hash}.
Cela impose quelques modifications au niveaux des importations et
d\'{e}clarations avec EVAL-WHEN notamment danc le module {\tt cpmac}.
Cela peut provoquer des imcompatibilit\'{e}s entre code compil\'{e} 15.24
et 15.25. Pour plus de suret\'{e}: {\bf un module important le module
{\tt hash} (m\^{e}me r\'{e}cursivement) devra \^{e}tre recompil\'{e} s'il est
lui-m\^{e}me import\'{e} par un module utilisant MAPHASH}.

\item {\bf 15.25} \\
La longueur de la table de hashage utilis\'{e}e par la {\it oblist}
Le-Lisp est pass\'{e}e de {\bf 263} \`{a} {\bf 997}. Ca n'a pas d'autre
cons\'{e}quence que d'optimiser les acc\`{e}s aux symboles dans les
grosses configurations sans vraiment p\'{e}naliser les petites, et de
modifier le r\'{e}sultat de la fonction 
HASH.

\item {\bf 15.25} \\
Modification du protocole de recherche des symboles C dans la table
des symboles: on d\'{e}conseille dor\'{e}navant l'utilisation du '\_' devant
les symboles pass\'{e}s \`{a} GETGLOBAL. Nous conservons la
possibilit\'{e} de le faire, mais au d\'{e}triment des performances de
recherche du symbole. L'option optimale est celle o\`{u} le nom C exact
apparait comme argument du GETGLOBAL.
Le code C est modifi\'{e} en cons\'{e}quence.

\item {\bf 15.25} \\
Possibilit\'{e} de suivre (STEP) les fonctions appel\'{e}es par {\tt SEND}
(sauf en mode graphique).

\item {\bf 15.25} \\
Quelques interventions ont \'{e}t\'{e} faites sur le code du Bitmap Virtuel
pour corriger certaines erreurs, mais aussi pour permettre
les incarnations du BV sur Motif et sur PostScript.

\end {itemize}

\item Intervention sur les codes C et assembleur :

\begin {itemize}

\item {\bf 15.24.2} \\
Traduction du noyau KerN en assembleur au format MOTOROLA sur
68K permettant l'introduction ais\'{e}e de KerN en assembleur sur la
majorit\'{e} des portages 68K. Nous gagnons ainsi un rapport 2 en vitesse
d'ex\'{e}cution des primitives de KerN sur ces machines.

\item {\bf 15.24.2} \\
Mise au point d'un certain nombre de flags s\'{e}mantiques de
compilation C sur les portages UNIX : 
LLPAGESIZE (taille des pages m\'{e}moire), LLTIMEUNIT
(fr\'{e}quence d'horloge), EXECORE (cores executables), CLOAD/NOCLOAD
(loader dynamique), ITINREAD (IT \`{a} la lecture), LLFOREGROUND (process
en foreground), LLVFORK (vfork ou non), LLRENAME (proc\'{e}dure rename
existe ou non), LLGHOSTPROC (processus fant\^{o}me). Et toujours: BSD4x
et S5. \\
Cette manipulation des flags offre une plus grande lisibilit\'{e} du code
C,  une plus grande maintenabilit\'{e}, permet une automatisation plus
ais\'{e}e de la 
g\'{e}n\'{e}ration de fichiers facilitant ainsi l'introduction de
nouveaux portages.


\item {\bf 15.25} \\
Int\'{e}gration du portage OS/2 (introduction de
code C en particulier). 

\item {\bf 15.25} \\
R\'{e}arrangement du fichier {\tt lelisp.c} en vue de facilit\'{e} sa
maintenance
et son \'{e}volution. L'id\'{e}e est d'introduire des fichiers C d\'{e}pendants
machine ({\it  exemple: lelisp/sun/C/Execore.c, lelisp/sun/C/Machine.h}) . 
Le fichier {\tt .c} contient tout le code tr\`{e}s sp\'{e}cifique au portage
(mode execore, getglobal, ...), le fichier {\tt .h} contient tous les
{\tt include}s propres \`{a} ce portage. Tous ces fichiers sont engendr\'{e}s
automatiquement. 

\item {\bf 15.25} \\
Introduction d'un processus fant\^{o}me \`{a} la {\tt lelispgo} permettant
de r\'{e}aliser des appels syst\`{e}mes sans exploser n encombrement
m\'{e}moire, sur les machines ne disposant pas de VFORK. Ce processus
est lanc\'{e} d\`{e}s le d\'{e}marrage de \LeLisp\ , et c'est lui qui est
utiliser pour {\it forker} lors d'appels syst\`{e}me tels que
COMLINE. Cette possibilit\'{e}e d\'{e}pend du flag de compilation {\tt
LLGHOSTPROC} et n'est pas op\'{e}rationnelle sur tous les portages.

\item {\bf 15.25} \\
Ajout de la fonction FCOPY en C, pour l'impl\'{e}mentation de la fonction
lisp COPYFILE.
\end {itemize}

\item Interventions sur le code LLM3:

\begin {itemize}

\item {\bf 15.25} \\
Ajout du code n\'{e}cessaire \`{a} la fonction {\tt COPY-FILE}.

\item {\bf 15.25} \\
Intervention sur les code des fonction GC et GCINFO. Toutes les zones
doivent exploiter le m\^{e}me protocole d'affichage: {\tt n} signifie
{\tt n} objets, {\tt (n)} signifie {\tt n} kilo-objets. 

\item {\bf 15.25} \\
Harmonisation de toutes les fonctions sur les chaines de caract\`{e}res.
Leur comportement est maintenant le m\^{e}me face \`{a} des arguments hors
limite (exemple: le caract\`{e}re de code ASCII 255, soit -1).

\item {\bf 15.25} \\
Correction d'une anomalie de comportement du GC sur les symboles
plac\'{e}s en t\^{e}te de {\it bucket} dans la {\it table de hashage} de la
{\it oblist}. 

\item {\bf 15.25} \\
Le code LLM3 a \'{e}t\'{e} localement optimis\'{e} en plusieurs endroits.

\end {itemize}

\item Corrections propres \`{a} COMPLICE:

\begin {itemize}

\item {\bf 15.24.2} \\
On change le nouveau syst\`{e}me de DEFVAR qui ne convenait pas
pour les {\tt (unless (boundp 'kop) (defvar kop ...))}. On red\'{e}fini
DEFVAR dans complice pour forcer les putprops. On parse quand m\^{e}me
pour pouvoir \'{e}tendre le {\bf cpenv}. Cela implique que si le
DEFVAR n'est pas en {\it ToplevelForm} alors le {\bf cpenv}
n'est pas \'{e}tendu. 

\item {\bf 15.24.2} \\
Correction de la g\'{e}n\'{e}ration en "dur" de l'\'{e}tiquette 0 pour le
test des {\tt nsubrs} ({\tt :check-nsubr} dans {\tt cp2.ll}).
On ouvre les fonctions {\tt typevector} et {\tt typestring}.

\item {\bf 15.24.2} \\
On traite les cas d\'{e}g\'{e}n\'{e}r\'{e}s de (AND) et (OR).
On fait la sauvegarde les {\tt (cvalq x)} proteg\'{e}s pour chaque {\tt
call} ou {\tt jump}.
Correction d'un bug d'allocation de registres dans (arg <expr>).

\item {\bf 15.25} \\
Les fonctions d\'{e}finies avec SYNONYM ont le m\^{e}me status que les
fonctions d\'{e}finies avec DEFUN. 

\item {\bf 15.25} \\
COMPILE-ALL-IN-CORE qui ne tenait pas compte de DONT-COMPILE
le fait maintenant.

\end{itemize}

\item Intervention sur la technologie de portage:

\begin {itemize}

\item {\bf 15.24.2} \\
Pr\'{e}sence unique du LOADER dans le r\'{e}pertoire de la machine.
Le nombre et la complexit\'{e} des portages allant en augmentant, il
apparait que chaque machine exige une ou plusieurs sp\'{e}cificit\'{e}s 
qui lui est
propre, qui la diff\'{e}rencie des autres machines utilisant le m\^{e}me
type de processeur.
exemples: big endian/little indian sur lapmips, introduction du 68040
sur lap68k, etc  

\item {\bf 15.24.2} \\
Am\'{e}nagement de la cha\^{\i}ne de production du binaire \LeLisp\ 
sur UNIX:
introduction du fichier {\tt lelispllm3.o} ne contenant strictement que
l'expansion LLM3, et garantissant ainsi la non-expansion LLM3.
Avec le temps, la complexit\'{e} des portages augmentant, 
un certains nombres de
choses sont venues par dessus l'expansion LLM3 (kern.o, plombage,
sp\'{e}cificit\'{e}s C de certains portages etc) et sont int\'{e}gr\'{e}es dans le
fichier {\tt lelispbin.o} qui lui est fourni aux clients commes base
de recompilation du syst\`{e}me. C'est toujours le makefile {\tt
Makeport} qui fabrique {\tt lelispbin.o}, mais c'est maintenant un
makefile {\tt makellm3} qui, en amont du {\tt Makeport}, 
fabrique le fichier {\tt lelispllm3.o}. Ce
makefile n'est lanc\'{e} qu'une fois par an \`{a} l'occasion des versions
majeures. 

\item {\bf 15.24.2, 15.25} \\
Homog\'{e}n\'{e}it\'{e} plus pouss\'{e}e sur les traits de portages tels que
EXECORE, CLOAD, 31BITFLOATS et 64BITFLOATS, scheduleur.

\item {\bf 15.24.2} \\
Encore plus de g\'{e}n\'{e}ration automatique: {\tt complice} et les
{\it configurateurs} d'images m\'{e}moire ({\tt conf/*conf.ll}) sont
engendr\'{e}s automatiquement. C'est pourquoi les noms des configurateurs
commencent maintenant par une majuscule. Le {\tt Makefile} int\`{e}gre
cela. 

\item {\bf 15.24.2} \\
Introduction de la recette interpr\'{e}t\'{e}e.

\item {\bf 15.24.2} \\
Etant donn\'{e} le temps que mettait la recette du module {\it
KerN} \`{a} s'ex\'{e}cuter sur cetaines machines, les tests 13, 18 et 19 ont
\'{e}t\'{e} simplifi\'{e}s. Les temps sont divis\'{e}s de 4 \`{a} 8 fois, mais 
il est bon de savoir que ce sont autant de cas non test\'{e}s par
d\'{e}faut. Pour en am\'{e}liorer les performances de facon significative,
nous ne pouvons que conseiller aux porteurs d'impl\'{e}menter le
module KerN en assembleur sur leur(s) syst\`{e}me(s).

\end {itemize}

\end{enumerate}

\Section {Corrections d'anomalies}

Les changements et corrections de \LeLisp\ sont g\'{e}r\'{e}s avec des \|RATs|
(Requ\^{e}tes d'Action Technique).  Nous n'avons fourni que les sujets
des RATs par souci de place, mais vous pouvez obtenir le texte complet
de chaque RAT sur simple demande.

\begin{tabbing}
\= xxxxxxxxxxxxxxxxxxxxxxxxxxxxxx\= xxxxxxxxxxxxxxxxxxxxxxxxxxxxxx\= xxxxxxxxxxxxxxxxxxxxxxxxxxxxxx \= \kill
\verb_Rats for lelisp selected by status-fixed as of 25/11/1991._\\ \verb_ _\\
\hspace{-5em}rat 297 \> area: {\it ports
\ }\> aspect: {\it tests
\ }\\ 
\> status: {\it fixed
\ }\> origin: {\it nuyens
\ }\\ 
\verb_tests des chargeurs ne testent pas assez #:ld:special-case-loader.
_\\ 
\verb_ _\\ 
\hspace{-5em}rat 344 \> area: {\it language
\ }\> aspect: {\it interpreter
\ }\\ 
\> status: {\it fixed
\ }\> origin: {\it kuczynsk
\ }\\ 
\verb_Le scheduleur se comporte differemment d'une machine a l'autre
_\\ 
\verb_ _\\ 
\hspace{-5em}rat 426 \> area: {\it memory-management
\ }\\ 
\> status: {\it fixed
\ }\> origin: {\it B. Serpette [INRIA]
\ }\\ 
\verb_simplification des llxxx.llm3 a propos de >ECONS/ECONS etc
_\\ 
\verb_ _\\ 
\hspace{-5em}rat 437 \> area: {\it memory-management
\ }\\ 
\> status: {\it fixed
\ }\> origin: {\it Marco NANNI [BULL]
\ }\\ 
\verb_probleme d'affichage des zones > 32767 par GCINFO
_\\ 
\verb_ _\\ 
\hspace{-5em}rat 462 \> area: {\it compiler
\ }\> aspect: {\it complice
\ }\\ 
\> status: {\it fixed
\ }\> origin: {\it meller
\ }\\ 
\verb_symbole de EXPORTABLE-DEFINITION a packager
_\\ 
\verb_ _\\ 
\hspace{-5em}rat 516 \> area: {\it run-time
\ }\> aspect: {\it external-functions
\ }\\ 
\> status: {\it fixed
\ }\> origin: {\it kuczynsk
\ }\\ 
\verb_commetn resoudre le pb des ">" devant les noms de fcts C 
_\\ 
\verb_ _\\ 
\hspace{-5em}rat 579 \> area: {\it compiler
\ }\> aspect: {\it complice
\ }\\ 
\> status: {\it fixed
\ }\> origin: {\it gregory@Gipsi.Gipsi.Fr
\ }\\ 
\verb_(UNLESS (AND) ... se compile mal sous complice
_\\ 
\verb_ _\\ 
\hspace{-5em}rat 641 \\ 
\> status: {\it fixed
\ }\> origin: {\it le
\ }\\ 
\verb_la TRACE de la fonction SEND est problematique
_\\ 
\verb_ _\\ 
\hspace{-5em}rat 667 \> area: {\it language
\ }\> aspect: {\it interpreter
\ }\\ 
\> status: {\it fixed
\ }\> origin: {\it A. Meller [Ilog]
\ }\\ 
\verb_nouvelle fonction ensembliste: POWER-SET
_\\ 
\verb_ _\\ 
\hspace{-5em}rat 668 \> area: {\it language
\ }\> aspect: {\it interpreter
\ }\\ 
\> status: {\it fixed
\ }\> origin: {\it A. Meller [Ilog]
\ }\\ 
\verb_nouvelle fonction ensembliste: CARTESIAN-PRODUCT
_\\ 
\verb_ _\\ 
\hspace{-5em}rat 677 \> area: {\it compiler
\ }\> aspect: {\it complice
\ }\\ 
\> status: {\it fixed
\ }\> origin: {\it MH Meffre [Ilog]
\ }\\ 
\verb_message d'erreur non adapt\'{e} pour PRINTDEFMODULE
_\\ 
\verb_ _\\ 
\hspace{-5em}rat 678 \> area: {\it compiler
\ }\> aspect: {\it complice
\ }\\ 
\> status: {\it fixed
\ }\> origin: {\it Roland Ducournau cerci2!coltrane.cerci2.fr!ducour
\ }\\ 
\verb_compilation de &NOBIND + ARG parfois fantaisiste
_\\ 
\verb_ _\\ 
\hspace{-5em}rat 679 \> area: {\it compiler
\ }\> aspect: {\it complice
\ }\\ 
\> status: {\it fixed
\ }\> origin: {\it Roland Ducournau  ducour@cerci2.uucp
\ }\\ 
\verb_ mauvaise compilation du LET o\`{u} l'ordre d'evaluation est inverse
_\\ 
\verb_ _\\ 
\hspace{-5em}rat 680 \> area: {\it language
\ }\> aspect: {\it interpreter
\ }\\ 
\> status: {\it fixed
\ }\> origin: {\it Roland Ducournau  ducour@cerci2.uucp
\ }\\ 
\verb_comportements heterogene de Makestring / Fillstring
_\\ 
\verb_ _\\ 
\hspace{-5em}rat 689 \> area: {\it memory-management
\ }\\ 
\> status: {\it fixed
\ }\> origin: {\it R. Ducourneau [Sema]
\ }\\ 
\verb_RESTORE-CORE ramene T sur m88k au lieu de ()
_\\ 
\verb_ _\\ 
\hspace{-5em}rat 691 \\ 
\> status: {\it fixed
\ }\> origin: {\it le
\ }\\ 
\verb_DIRECTORYP devrait rendre le nom d'un directory plutot qu'un fichier
_\\ 
\verb_ _\\ 
\hspace{-5em}rat 694 \> area: {\it run-time
\ }\> aspect: {\it o/s-interface
\ }\\ 
\> status: {\it fixed
\ }\> origin: {\it Kuczynski P. [Ilog]
\ }\\ 
\verb_il arrive que TTYIN (dans llstdio.c) recupere des codes d'erreur
_\\ 
\verb_ _\\ 
\hspace{-5em}rat 702 \> area: {\it language
\ }\> aspect: {\it interpreter
\ }\\ 
\> status: {\it fixed
\ }\> origin: {\it kuczynsk
\ }\\ 
\verb_(EXPAND-PATHNAME "<dir>*/") n'expanse pas les repertoire s "dir*"
_\\ 
\verb_ _\\ 
\hspace{-5em}rat 703 \> area: {\it compiler
\ }\> aspect: {\it complice
\ }\\ 
\> status: {\it fixed
\ }\> origin: {\it kuczynsk
\ }\\ 
\verb_une fct definie par SYNONYMQ n'est pas reconnue par complice
_\\ 
\verb_ _\\ 
\hspace{-5em}rat 716 \> area: {\it ports
\ }\> aspect: {\it loader
\ }\\ 
\> status: {\it fixed
\ }\> origin: {\it ing
\ }\\ 
\verb_typos d'un message d'erreur du loader 68K
_\\ 
\verb_ _\\ 
\hspace{-5em}rat 717 \> area: {\it compiler
\ }\> aspect: {\it complice
\ }\\ 
\> status: {\it fixed
\ }\> origin: {\it cerci2!ducour%coltrane.cerci2.fr (Roland Ducournau)
\ }\\ 
\verb_compilation d'une fct utilisant une MSUBR realisant un effet de bord
_\\ 
\verb_ _\\ 
\hspace{-5em}rat 724 \> area: {\it ports
\ }\> aspect: {\it tests
\ }\\ 
\> status: {\it fixed
\ }\> origin: {\it Shumacher [Madicia]
\ }\\ 
\verb_le lancement des recettes completes impose une recompilation des binaires
_\\ 
\verb_ _\\ 
\hspace{-5em}rat 725 \> area: {\it run-time
\ }\> aspect: {\it error-handling
\ }\\ 
\> status: {\it fixed
\ }\> origin: {\it Shumacher [Madicia]
\ }\\ 
\verb_l'affichage de l'erreur d'image memoire incompatible provoque 1 autre erreur
_\\ 
\verb_ _\\ 
\hspace{-5em}rat 726 \> area: {\it language
\ }\> aspect: {\it extended-types
\ }\\ 
\> status: {\it fixed
\ }\> origin: {\it Shumacher [Madicia]
\ }\\ 
\verb_limitation des noms de fichiers \`{a} 8.3 pour OS2
_\\ 
\verb_ _\\ 
\hspace{-5em}rat 729 \> area: {\it ports
\ }\> aspect: {\it tests
\ }\\ 
\> status: {\it fixed
\ }\> origin: {\it Shumacher [madicia]
\ }\\ 
\verb_certains tests sur les pathnames sont trop orient\'{e}s UNIX
_\\ 
\verb_ _\\ 
\hspace{-5em}rat 731 \> area: {\it ports
\ }\> aspect: {\it loader
\ }\\ 
\> status: {\it fixed
\ }\> origin: {\it kuczynsk
\ }\\ 
\verb_petites incoherences des loaders enttre eux
_\\ 
\verb_ _\\ 
\hspace{-5em}rat 740 \> area: {\it compiler
\ }\> aspect: {\it complice
\ }\\ 
\> status: {\it fixed
\ }\> origin: {\it dvergami@crios.inria.fr
\ }\\ 
\verb_COMPILE-ALL-IN-CORE ne tient pas compte de DONT-COMPILE
_\\ 
\verb_ _\\ 
\hspace{-5em}rat 755 \> area: {\it language
\ }\> aspect: {\it other
\ }\\ 
\> status: {\it fixed
\ }\> origin: {\it meller
\ }\\ 
\verb_Message au chargement d'un module incorrect
_\\ 
\verb_ _\\ 
\hspace{-5em}rat 758 \> area: {\it other
\ }\\ 
\> status: {\it fixed
\ }\> origin: {\it meller
\ }\\ 
\verb_make lelisp-
_\\ 
\verb_ _\\ 
\hspace{-5em}rat 777 \> area: {\it ports
\ }\> aspect: {\it loader
\ }\\ 
\> status: {\it fixed
\ }\> origin: {\it ducour@sema-taa.fr
\ }\\ 
\verb_la div. ent. LLM3  QUO ne sait pas diviser par 128
_\\ 
\verb_ _\\ 
\hspace{-5em}rat 779 \> area: {\it language
\ }\> aspect: {\it arithmetic
\ }\\ 
\> status: {\it fixed
\ }\> origin: {\it kerjean
\ }\\ 
\verb_la fonction div sur sun4
_\\ 
\verb_ _\\ 
\hspace{-5em}rat 813 \> area: {\it run-time
\ }\> aspect: {\it error-handling
\ }\\ 
\> status: {\it fixed
\ }\> origin: {\it kuczynsk
\ }\\ 
\verb_verification des tailles des zones au lancement d'un core insuffisante
_\\ 
\verb_ _\\ 
\hspace{-5em}rat 306 \> area: {\it language
\ }\> aspect: {\it extended-types
\ }\\ 
\> status: {\it fixed
\ }\> origin: {\it chaillou
\ }\\ 
\verb_type des types \'{e}tendus
_\\ 
\verb_ _\\ 
\hspace{-5em}rat 372 \> area: {\it run-time
\ }\> aspect: {\it external-functions
\ }\\ 
\> status: {\it fixed
\ }\> origin: {\it M Nanni [Bull]
\ }\\ 
\verb_la taiiles des entiers du langage ext. depend des machines
_\\ 
\verb_ _\\ 
\hspace{-5em}rat 391 \> area: {\it compiler
\ }\> aspect: {\it complice
\ }\\ 
\> status: {\it fixed
\ }\> origin: {\it H Bessonne [Ilog]
\ }\\ 
\verb_limite de la sucrerie lexicale des structures dans les exportaion des .lm
_\\ 
\verb_ _\\ 
\hspace{-5em}rat 489 \> area: {\it language
\ }\> aspect: {\it interpreter
\ }\\ 
\> status: {\it fixed
\ }\> origin: {\it berizzi
\ }\\ 
\verb_La doc de la fonction (err) est douteuse.
_\\ 
\verb_ _\\ 
\hspace{-5em}rat 613 \> area: {\it compiler
\ }\> aspect: {\it complice
\ }\\ 
\> status: {\it fixed
\ }\> origin: {\it pommier@ilog
\ }\\ 
\verb_dedoublement des clef dans les descripteurs de modules
_\\ 
\verb_ _\\ 
\hspace{-5em}rat 673 \> area: {\it i/o
\ }\> aspect: {\it virbitmap
\ }\\ 
\> status: {\it fixed
\ }\> origin: {\it J. Chailloux [Ilog]
\ }\\ 
\verb_code de BLTVECTOR est en avance sur la doc et les tests
_\\ 
\verb_ _\\ 
\hspace{-5em}rat 674 \> area: {\it language
\ }\> aspect: {\it interpreter
\ }\\ 
\> status: {\it fixed
\ }\> origin: {\it J. Chailloux [Ilog]
\ }\\ 
\verb_doc. francaise de GETL non conforme.
_\\ 
\verb_ _\\ 
\hspace{-5em}rat 684 \> area: {\it i/o
\ }\> aspect: {\it virbitmap
\ }\\ 
\> status: {\it fixed
\ }\> origin: {\it Roland Ducournau  ducour@cerci2.uucp
\ }\\ 
\verb_BITBLIT de l'undo d'un editeur d'icone semble se faire en mode XOR
_\\ 
\verb_ _\\ 
\hspace{-5em}rat 688 \> area: {\it i/o
\ }\> aspect: {\it virbitmap
\ }\\ 
\> status: {\it fixed
\ }\> origin: {\it Roland Ducournau  ducour@cerci2.uucp
\ }\\ 
\verb_WINDOW-CLEAR-REGION ne tient pas compte du CLIP
_\\ 
\verb_ _\\ 
\hspace{-5em}rat 693 \> area: {\it language
\ }\> aspect: {\it interpreter
\ }\\ 
\> status: {\it fixed
\ }\> origin: {\it G. Schmacher [MADICIA]
\ }\\ 
\verb_INDEX doit etre document\'{e} comme une fct \`{a} 2 ou 3 arguments
_\\ 
\verb_ _\\ 
\hspace{-5em}rat 727 \> area: {\it language
\ }\> aspect: {\it extended-types
\ }\\ 
\> status: {\it fixed
\ }\> origin: {\it Shumacher [madicia]
\ }\\ 
\verb_CONTROL-FILE-PATHNAME ne remplie pas tjrs correctement son r\^{o}le
_\\ 
\verb_ _\\ 
\hspace{-5em}rat 728 \> area: {\it language
\ }\> aspect: {\it extended-types
\ }\\ 
\> status: {\it fixed
\ }\> origin: {\it Shumacher [Madicia]
\ }\\ 
\verb_le wildcarding n'est pas possible sur les repertoire en OS2
_\\ 
\verb_ _\\ 
\hspace{-5em}rat 756 \> area: {\it language
\ }\> aspect: {\it other
\ }\\ 
\> status: {\it fixed
\ }\> origin: {\it kerjean
\ }\\ 
\verb_doc de la fonction probepathf
_\\ 
\verb_ _\\ 
\hspace{-5em}rat 778 \> area: {\it ports
\ }\> aspect: {\it loader
\ }\\ 
\> status: {\it fixed
\ }\> origin: {\it euzenat
\ }\\ 
\verb_Bugs documentation modules
_\\ 
\verb_ _\\ 
\hspace{-5em}rat 794 \> area: {\it i/o
\ }\> aspect: {\it basic-i/o
\ }\\ 
\> status: {\it fixed
\ }\> origin: {\it meller
\ }\\ 
\verb_pathnames namestring
_\\ 
\verb_ _\\ 
\end{tabbing}
\newpage
\bigskip

\tableofcontents
\listoftables

\End
