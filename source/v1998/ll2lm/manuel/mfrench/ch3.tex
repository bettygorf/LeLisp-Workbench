%%%%%%%%%%%%%%%%%%%%%%%%%%%%%%%%%%%%%%%%%%%%%%%%%%%%%%%%%%%%%%%%%%%%%%%%%%%%
\Chapter {3} {Le compilateur de modules}
%%%%%%%%%%%%%%%%%%%%%%%%%%%%%%%%%%%%%%%%%%%%%%%%%%%%%%%%%%%%%%%%%%%%%%%%%%%%

Dans ce chapitre, nous verrons comment utiliser le r\'{e}sultat de
l'{\em Analyseur de modules} avec le
{\em compilateur}.
Nous allons \'{e}galement rappeler bri\`{e}vement le mode d'emploi du
compilateur modulaire fourni avec \LeLisp\,: {\tt complice}. 


%---------------------------------------------------------------------------
\Section {Apr\`{e}s l'analyse, la compilation}
%---------------------------------------------------------------------------
Nous avons vu dans le chapitre pr\'{e}c\'{e}dent, comment fabriquer des
fichiers descripteurs de modules \LeLisp. 
En r\'{e}alit\'{e}, ces fichiers n'ont de raison d'\^{e}tre que
pour le compilateur {\tt complice} qui utilise les informations
contenues dans ces fichiers descripteurs de modules pour compiler 
le plus efficacement possible (cf Chap\^{\i}tre 13 du manuel de
r\'{e}f\'{e}rence \LeLisp ).\\

%---------------------------------------------------------------------------
\Section {Sp\'{e}cificit\'{e}s de l'analyseur pour la compilation}
%---------------------------------------------------------------------------
Un certain nombre de clefs de \|define-rt-project| et une option de la
commande d'analyse {\tt ll2lm} 
concernent directement la compilation des modules. Nous d\'{e}taillons
ces options ci-apr\`{e}s\,:

\SSection {Option d'analyse d\'{e}di\'{e}e \`{a} la compilation}
%-----------------------------------------------------
L'option de l'analyseur {\tt ll2lm} d\'{e}di\'{e}e \`{a} la compilation est
celle qui permet de 
fabriquer le {\tt Makefile} de compilation\,:
\begin{itemize}

\item {\Large \|-makefile|}: cette option principale permet
d'engendrer un {\tt Makefile} de compilation modulaire ({\tt cf
complice}) de tous les modules du projet concern\'{e} (cf option
\|-project|). Par d\'{e}faut le fichier r\'{e}sultat de la compilation 
d'un module, par {\tt complice} ({\tt
.lo}) est rang\'{e} dans le r\'{e}pertoire sp\'{e}cifi\'{e} par la clef {\tt
ll-object-directory} s'il existe, ou sinon \`{a} c\^{o}t\'{e} du fichier de
description modulaire ({\tt .lm}). \\
Ce {\tt Makefile} offre certaines entr\'{e}es pr\'{e}d\'{e}finies:
\begin {enumerate}
\item \|all| (d\'{e}faut)\ :\ 
permet de mettre \`{a} jour les compilations de tous les modules du projet.
\item \|clean|\ :\ 
permet d'effacer les fichiers objet ({\tt .lo}) de tous les
modules du projet.
\BeginLL
        % make -f myproject.mk clean
\EndLL
\item \|i|\ :\ 
permet d'entrer dans la boucle {\tt toplevel} interactive du
compilateur, avec tous les flags de compilation charg\'{e}s dans
l'environnement. 
\BeginLL
        % make -f myproject.mk i
\EndLL
\end{enumerate}
Voir \'{e}galement l'option \|-dependency| pour jouer sur le niveau de
d\'{e}pendance des entr\'{e}es du {\tt Makefile}.\\
En m\^{e}me temps que le {\tt Makefile}, est g\'{e}n\'{e}r\'{e} un fichier {\bf
project.pth} contenant une d\'{e}finition de la variable {\tt
\#:system:path} avec tous les chemins d'acc\`{e}s n\'{e}cessaires \`{a}
l'execution des modules du project {\tt project>}.

\end{itemize}

Il est conseill\'{e} d'utiliser
ce {\tt Makefile} dans la suite de la gestion du projet, afin de
compiler l'ensemble des modules. Ce {\tt Makefile} sera utile pour
r\'{e}aliser la premi\`{e}re s\'{e}rie de compilations (la premi\`{e}re fois pour
l'ensemble des modules) mais \'{e}galement par la suite, pour la 
maintenance des modules. \\
En g\'{e}n\'{e}ral, on se contentera de lancer le {\tt Makefile} pour mettre
\`{a} jour l'ensemble des modules modifi\'{e}s et leurs d\'{e}pendances.
Ces d\'{e}pendances varient en fonction de l'option d'analyse \|-dependency|
utilis\'{e}e lors de la fabrication du {\tt Makefile} (cf chap\^{\i}tre 6).

\SSection {Clefs de d\'{e}finition de projet d\'{e}di\'{e}es \`{a} la compilation}
%--------------------------------------------------------------------
Les clefs de \|define-rt-project| concernant la compilation sont\,:
\begin{itemize}

\item {\Large \|ll-object-directory| {\em path}}: cette clef, si elle existe,
doit d\'{e}signer le r\'{e}pertoire o\`{u} seront rang\'{e}s tous les fichiers
r\'{e}sultat de la compilation des modules ({\tt .lo}) par {\tt
complice}. Cette
valeur peut \^{e}tre du type {\tt string} ou {\tt pathname}.  \\
Par d\'{e}faut, ces fichiers seront
rang\'{e}s dans le r\'{e}pertoire d\'{e}sign\'{e} par la clef
\|ll-module-directory| ou, si cette clef n'existe pas non plus, ils
seront rang\'{e}s dans le m\^{e}me r\'{e}pertoire que le fichier descripteur
de module ({\tt .lm}).
Cette clef est exploit\'{e}e par l'option \|-makefile| de
l'{\em Analyseur de Modules}.

\item {\Large \|make-file| {\em path}}: cette clef, si elle existe, doit
d\'{e}signer le nom du {\tt Makefile} de compilation engendr\'{e} par l'analyseur
de modules avec l'option \|-makefile|. Cette
valeur peut \^{e}tre du type {\tt string} ou {\tt pathname}. \\
Par d\'{e}faut, le nom de ce {\tt Makefile} sera calcul\'{e} \`{a}
partir du nom du projet, suffix\'{e} par l'extension {\tt ".mk"}, et
rang\'{e} dans le r\'{e}pertoire d\'{e}sign\'{e} par la clef \|system-directory|
(ou sa valeur par d\'{e}faut s'il elle n'existe pas).
Cette clef est exploit\'{e}e par l'option \|-makefile| de l'{\em Analyseur de
Modules}.

\item {\Large \|complice-options| {\em A-list}}: cette clef, si elle existe,
contient une A-liste o\`{u} le premier \'{e}l\'{e}ment de chaque sous-liste est le
nom d'un fichier descripteur de module, et le reste de la sous liste
est une liste de cha\^{\i}nes de caract\`{e}res repr\'{e}sentant des options
utilis\'{e}es lors de la compilation du module d\'{e}sign\'{e}. \\
Par exemple, si {\tt mymod} doit \^{e}tre
compil\'{e} avec le flag de compilation \|-parano T|\, on \'{e}crira\,:
\BeginLL
(define-rt-project myproject
  ...
  complice-options ((mymod "-parano T"))
  ...
)
\EndLL
Il est possible d'imposer des options de compilation pour tous les modules
du projet en utilisant le mot clef {\em "all"} sous forme de {\it
string} \`{a} la place du nom d'un module\,:
\BeginLL
(define-rt-project myproject
  ...
  complice-options ((mymod1 "-parano T")
                    ("all" "-o /tmp/"))
  ...
)
\EndLL
Le comportement de "all" dans la clef \|complice-options| est
identique \`{a} celui de "all" dans la clef \|analyzer-options|.

Cette clef est exploit\'{e}e par l'option \|-makefile| de l'{\em Analyseur de
Modules}.

\end{itemize}


%---------------------------------------------------------------------------
\Section {Compilation d'un module}
%---------------------------------------------------------------------------
Il existe un certain nombre d'options du compilateur utilisables
avec la commande {\tt complice}. Ces options peuvent \^{e}tre utilis\'{e}es
module par module, ou globalement, dans le {\tt Makefile} engendr\'{e}
par l'{\em Analyseur de modules}, au 
moyen du champ \|complice-options| de {\tt define-rt-project}.

{\it Rappel: Sous Unix, ces options sont d\'{e}taill\'{e}es dans le manuel
Unix fourni avec la bande de distribution: {\tt lelisp/manl/complice.l}}

\begin{itemize}
\item {\Large \|-callext flagp|} : permet de d\'{e}cider si on r\'{e}alise
ou non les accesseurs aux symboles externes, en fonction de \|flagp|.
Si \|flagp| est NIL (d\'{e}faut), alors les accesseurs sont red\'{e}finis
de fa\c{c}on \`{a} ce qu'ils n'aient aucun effet durant la compilation : les
fonctions {\tt getglobal} et {\tt defextern-cache} sont red\'{e}finies de
fa\c{c}on \`{a} rester inop\'{e}rantes. Si \|flagp| est diff\'{e}rent de NIL, ces
fonctions continuent de fonctionner normalement durant la compilation:
les symboles externes devront alors \^{e}tre pr\'{e}sents dans l'image
binaire du compilateur \|complice|.

\item {\Large \|-cons n|} : permet d'\'{e}tendre la zone des {\tt cons}
\`{a} une valeur \'{e}gale \`{a} {\tt \|n| * 8 Kcons}.

\item {\Large \|-cmplc cmd|} : permet de sp\'{e}cifier une autre commande
de lancement que l'image m\'{e}moire \|cmplc++| utilis\'{e}e par d\'{e}faut.

\item {\Large \|-e S-expr|} : permet d'ex\'{e}cuter la forme \LeLisp\
\|S-expr| dans l'environnement de compilation, avant de r\'{e}aliser la
compilation modulaire.

\item {\Large \|-g flagp|} : permet de faire afficher l'\'{e}tat de la
m\'{e}moire lisp \`{a} l'aide de la commande (GC T), apr\`{e}s la compilation.

\item {\Large \|-i|} : permet d'entrer sous le {\tt toplevel}
interactif de \|complice|. Tr\`{e}s utile en phase de ``debug{''}, pour
observer interactivement le comportement du compilateur. Cette option
est activ\'{e}e avec l'entr\'{e}e \|i| du {\tt Makefile} de compilation
engendr\'{e} par l'{\em Analyseur de Modules}.

\item {\Large \|-lldir path|} : permet d'aller chercher un core de
compilation (par d\'{e}faut: \|cmplc++|) dans le r\'{e}pertoire d\'{e}crit par
\|path|. 

\item {\Large \|-o path|} : permet de sp\'{e}cifier le r\'{e}pertoire o\`{u}
sera rang\'{e} le fichier r\'{e}sultat de la compilation modulaire ({\tt
.lo}). Par d\'{e}faut, ces fichiers sont rang\'{e}s dans le r\'{e}pertoire
courant. 

\item {\Large \|-p path|} : permet d'ajouter le chemin \|path| dans la
variable \|#:system:path| avant la compilation modulaire. 

\item {\Large \|-parano flagp|} : permet de positionner la variable du
compilateur \|#:complice:parano-flag| \`{a} la valeur de \|flagp|. La
valeur par d\'{e}faut est NIL. cf chapitre 13 du Manuel de R\'{e}f\'{e}rence
\LeLisp\ pour plus de d\'{e}tails sur l'utilisation de cette variable.

\item {\Large \|-r|} : permet de positionner la variable 
\|#:system:read-case-flag| \`{a} \|T|. Cette option est
obligatoire si on d\'{e}sire influencer la valeur de 
\|#:system:read-case-flag|; ne pas utiliser l'option \|-e| qui n'a pas
d'effet dans ce cas.

\item {\Large \|-v|} : permet d'afficher les formes \LeLisp\
\'{e}valu\'{e}es dans l'environnement de compilation, avant la compilation
modulaire.

\item {\Large \|-w flagp|} : permet de positionner la variable 
\|#:complice:warning-flag| \`{a} la valeur de \|flagp|. cf chapitre 13 du
{\em Manuel de R\'{e}f\'{e}rence} \LeLisp\ pour plus de d\'{e}tails sur l'utilisation
de cette variable.
\end{itemize}
