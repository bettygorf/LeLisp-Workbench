%=======================================================================
\Chapter {5} {Manuel de r\'{e}f\'{e}rence}
%=======================================================================
Ce chapitre documente de fa\c{c}on exhaustive toutes les clefs accept\'{e}es
par la forme de d\'{e}finition
de projet {\tt define-rt-project}, et toutes les
options d'analyse de la commande {\tt ll2lm}.

%-----------------------------------------------------------------------
\Section {DEFINE-RT-PROJECT}
%-----------------------------------------------------------------------
La forme de d\'{e}finition d'un projet accepte un certain nombre de clefs
dont les principales sont abord\'{e}es dans le chap\^{\i}tre 2.  Nous documentons
ici l'ensemble des clefs, pr\'{e}sent\'{e}es dans l'ordre
alphab\'{e}tique. 

\begin{itemize}

\item {\Large \|activate-function|}\,: permet de sp\'{e}cifier une
fonction sans argument qui sera appel\'{e}e lorsque l'on r\'{e}f\'{e}rencera ce
projet dans l'analyse d'un module. Par exemple, pour les projets
\Aida, il est n\'{e}cessaire de positionner la variable
\|#:system:defstruct-all-access-flag| \`{a} \|nil|. Pour cela, il suffit
de sp\'{e}cifier:

\begin{Code*}
activate-function (lambda () 
                    (setq #:system:defstruct-all-access-flag ()))
\end{Code*}

Cette fonction sert g\'{e}n\'{e}ralement \`{a} modifier l'environnement \Lisp \
avant de commencer l'analyse d'un module.

Cette clef est h\'{e}rit\'{e}e lorsque ce projet est "import\'{e}" par un
autre projet ({\bf Attention\,: Ce comportement est diff\'{e}rent 
de celui de l'ancien analyseur de modules}).

Cette fonction est appel\'{e}e lorsque l'on active un projet pour
l'{\em Analyseur de Modules} par la fonction \|activate-rt-project|,
soit \`{a} chaque fois qu'un projet se trouve import\'{e} r\'{e}cursivement via
la clef \|required-project|

Voir aussi la clef \|initialize-function|.

\item {\Large \|analyzer-options|}\,: cette clef, si elle existe,
contient une A-liste o\`{u} le premier \'{e}l\'{e}ment de chaque sous-liste est le
nom d'un fichier descripteur de module, et le reste de la sous liste
est une liste de cha\^{\i}nes de caract\`{e}res repr\'{e}sentant des options
utilis\'{e}es lors de l'analyse du module d\'{e}sign\'{e}. \\
Par exemple, si {\tt mymod} doit \^{e}tre le seul module analys\'{e} avec les options
\|-includeflag|\ et \|-dynamic|\ :
\begin{Code*}
(define-rt-project myproject
  ...
  analyzer-options ((mymod "-includeflag" "-dynamic"))
  ...
)
\end{Code*}
Il est possible d'imposer des options d'analyse pour tous les modules
du projet en utilisant le mot clef {\em "all"} sous forme de {\it
string} \`{a} la place du nom d'un module\,:
\begin{Code*}
(define-rt-project myproject
  ...
  analyzer-options ((mymod "-includeflag" "-dynamic")
                    ("all" "-v 1"))
  ...
)
\end{Code*}
Dans l'exemple pr\'{e}c\'{e}dent, le module \|mymod| ne b\'{e}n\'{e}ficie pas du
flag pr\'{e}cis\'{e} par "all". Pour analyser \|mymod| avec \|-v 1|, on
devra le pr\'{e}ciser explicitement avec \|mymod|\,:
\begin{Code*}
(define-rt-project myproject
  ...
  analyzer-options (;; only mymod
                    (mymod "-v 1 -includeflag" "-dynamic")
                    ;; all others
                    ("all" "-v 1"))
  ...
)
\end{Code*}


Cette clef est exploit\'{e}e par l'option \|-init| de l'{\em Analyseur de
Modules}. 

\item {\Large \|complice-options|}\,: cette clef, si elle existe,
contient une A-liste o\`{u} le premier \'{e}l\'{e}ment de chaque sous-liste est le
nom d'un fichier descripteur de module, et le reste de la sous liste
est une liste de cha\^{\i}nes de caract\`{e}res repr\'{e}sentant des options
utilis\'{e}es lors de la compilation du module d\'{e}sign\'{e}. On choisira
d'\'{e}crire une cha\^{\i}ne de caract\`{e}res par option pour \'{e}viter les
lignes trop longues dans un \|Makefile|.\\
Par exemple, si {\tt mymod} doit \^{e}tre
compil\'{e} avec le flag de compilation \|-parano T|\, et en chargeant le
fichier {\tt myfile}, on \'{e}crira\,:
\begin{Code*}
(define-rt-project myproject
  ...
  complice-options ((mymod "-e ""(loadfile 'myfile.ll t)"""
                           "-parano T"))
  ...
)
\end{Code*}
Il est possible d'imposer des options de compilation pour tous les modules
du projet en utilisant le mot clef {\em "all"} sous forme de {\it
string} \`{a} la place du nom d'un module:
\begin{Code*}
(define-rt-project myproject
  ...
  complice-options ((mymod "-e ""(loadfile 'myfile.ll t)"""
                           "-parano T")
                    ("all" "-o /tmp/"))
  ...
)
\end{Code*}
Le comportement de "all" dans la clef \|complice-options| est
identique \`{a} celui de "all" dans la clef \|analyzer-options|.

Cette clef est exploit\'{e}e par l'option \|-makefile| de l'{\em Analyseur de
Modules}.

\item {\Large \|crunch-directory|}\,: le r\'{e}pertoire qui va contenir
le fichier contenant la table des r\'{e}f\'{e}rences. Par d\'{e}faut,
\|crunch-directory| prend la valeur de \|root-directory|.

\item {\Large \|directories|}\,: pour sp\'{e}cifier la liste des
r\'{e}pertoires qui contiennent les sources et/ou les modules du projet
\|name|.

\item {\Large \|exclude-modules|}\,: cette clef permet de sp\'{e}cifier les
modules \`{a} ne pas prendre en compte
pour construire les tables d'analyse. Cela
peut par exemple concerner les modules de patch. Par exemple\,:

\begin{Code*}
(define-rt-project mycasetool
        required-projects (aida smeci)
        root-directory #.mycasetool-dir
        directories ("src" "include" "modules")
        crunch-directory #u"mycasetool/work/"
        exclude-modules (graphpatches devpatches))
\end{Code*}

\item {\Large \|extensions-list|}\,: cette clef , si elle est
sp\'{e}cifi\'{e}e, doit contenir une liste d'extensions de fichiers (type
string) servant \`{a} calculer les noms des fichiers sources \`{a} partir de
la racine du nom du module. Il est fr\'{e}quent de voir un module
\|mymod| avoir comme description modulaire\,:
\begin{Code*}
defmodule mymod
files (mymod.ll mymod.lh mymod.li)
\end{Code*}
Si cela est r\'{e}p\'{e}titif dans le projet, on utilisera la clef
\|extensions-list|\ : 
\begin{Code*}
(define-rt-project myproject
  ...
  extensions-list ("ll" "lh" "li")
  ...
)
\end{Code*}
Bien entendu, l'{\em Analyseur de Modules} ne range alors dans le champ
\|files| de la description modulaire que les fichiers qui existent.
Ainsi dans notre exemple, si un autre module {\tt othermod.lm}
utilise les fichiers source {\tt othermod.ll} et {\tt othermod.lh}
mais que le fichier {\tt othermod.li} n'existe pas, nous aurons comme
description modulaire:
\begin{Code*}
defmodule othermod
files (othermod.ll othermod.lh)
\end{Code*}
Cette clef est exploit\'{e}e par les options \|-init| et \|-defmodule| de
l'analyseur.

\item {\Large \|init-makefile|}\,: cette clef, si elle existe, doit
d\'{e}signer le nom du {\tt Makefile} d'initialisation engendr\'{e} par
l'{\em Analyseur de Modules} avec l'option \|-init|. Cette
valeur peut \^{e}tre du type {\tt string} ou {\tt pathname}.  \\
Par d\'{e}faut, le nom de ce {\tt Makefile} sera calcul\'{e} \`{a}
partir du nom du projet, suffix\'{e} par l'extension {\tt ".mki"}, et
rang\'{e} dans le r\'{e}pertoire d\'{e}sign\'{e} par la clef \|system-directory|
(ou sa valeur par d\'{e}faut s'il elle n'existe pas).
Cette clef est exploit\'{e}e par l'option \|-init| de l'{\em Analyseur de
Modules}.

\item {\Large \|initialize-function|}\,: cette clef permet de sp\'{e}cifier
une fonction sans argument qui sera appel\'{e}e lorsque l'on
r\'{e}f\'{e}rencera ce projet lors de l'analyse d'un module. \`{A} la
diff\'{e}rence de la clef \|activate-function|, cette clef n'est appel\'{e}e
qu'une seule fois par analyse et par session. 

Par exemple pour certains des sous-projets d'\Aida, 
il est n\'{e}cessaire d'initialiser l'environnement
de mani\`{e}re sp\'{e}cifique avant de d\'{e}marrer l'analyse\,:

\begin{Code*}
(define-rt-project mdacurve
  required-projects (mdakerne)
  root-directory #.rt-aida-directory
  ...
  initialize-function aida-init-func
  ...
  )
\end{Code*}

Cette fonction est appel\'{e}e lorsque l'on charge les tables d'un projet
par les fonctions \|load-rt-project| ou \|reload-rt-project|.

\item {\Large \|ll-module-directory|}\,: cette clef, si elle existe,
doit d\'{e}signer le r\'{e}pertoire o\`{u} seront rang\'{e}s tous les fichiers
descripteurs de modules ({\tt .lm}) fabriqu\'{e}s automatiquement. Cette
valeur peut \^{e}tre du type {\tt string} ou {\tt pathname}. \\
Par d\'{e}faut, les nouveaux modules seront
rang\'{e}s dans le r\'{e}pertoire d\'{e}sign\'{e} par la clef
\|ll-object-directory| ou, si cette clef n'existe pas non plus, chaque
fichier descripteur de module sera rang\'{e} dans le m\^{e}me r\'{e}pertoire
que le premier fichier source ({\tt .ll}) trouv\'{e} sous le champ \|files|
du module.
Cette clef est exploit\'{e}e par les options \|-init| et \|-defmodule| de
l'{\em Analyseur de Modules}.

\item {\Large \|ll-object-directory|}\,: cette clef, si elle existe,
doit d\'{e}signer le r\'{e}pertoire o\`{u} seront rang\'{e}s tous les fichiers
r\'{e}sultat de la compilation des modules ({\tt .lo}) par {\tt
complice}. Cette
valeur peut \^{e}tre du type {\tt string} ou {\tt pathname}.  \\
Par d\'{e}faut, ces fichiers seront
rang\'{e}s dans le r\'{e}pertoire d\'{e}sign\'{e} par la clef
\|ll-module-directory| ou, si cette clef n'existe pas non plus, ils
seront rang\'{e}s dans le m\^{e}me r\'{e}pertoire que le fichier descripteur
de module ({\tt .lm}).
Cette clef est exploit\'{e}e par l'option \|-makefile| de
l'{\em Analyseur de Modules}.

\item {\Large \|make-file|}\,: cette clef, si elle existe, doit
d\'{e}signer le nom du {\tt Makefile} de compilation engendr\'{e} par l'analyseur
de modules avec l'option \|-makefile|. Cette
valeur peut \^{e}tre du type {\tt string} ou {\tt pathname}. \\
Par d\'{e}faut, le nom de ce {\tt Makefile} sera calcul\'{e} \`{a}
partir du nom du projet, suffix\'{e} par l'extension {\tt ".mk"}, et
rang\'{e} dans le r\'{e}pertoire d\'{e}sign\'{e} par la clef \|system-directory|
(ou sa valeur par d\'{e}faut s'il elle n'existe pas).
Cette clef est exploit\'{e}e par l'option \|-makefile| de l'{\em Analyseur de
Modules}.

\item {\Large \|module-extension|}\,: cette clef, si elle existe,
contient une cha\^{\i}ne de caract\`{e}res. Cette cha\^{\i}ne de carat\`{e}res
d\'{e}signe l'extension utilis\'{e}e pour d\'{e}signer le nom du fichier
servant de r\'{e}f\'{e}rence de description modulaire. Si l'utilisateur
d\'{e}sire conserver une version de ses fichiers de descriptions modulaires, non
``pollu\'{e}e {''} par {\tt complice} (rappel: {\tt complice} \'{e}crit ses
propres informations dans les fichiers de description modulaire), il
choisira une nouvelle extension de fichier diff\'{e}rente de {\tt
"lm"}, afin de cr\'{e}er un nouveau fichier. On choisi habituellement
l'extension {\tt "lc"}. \\
Cette clef est exploit\'{e}e par les options \|-init|, \|-makefile| et
\|-build| de l'{\em Analyseur de Modules}.
Les {\tt Makefile} engendr\'{e}s par l'utilisation de ces options utilisent alors
les {\tt "lc"} comme point de d\'{e}part. 
\begin{itemize}
\item Dans le cas du {\tt Makefile}
d'analyse, on utilisera le {\tt .lc} comme fichier source du {\tt .lm}
sur lequel s'effectuera l'analyse. Le r\'{e}sultat de l'analyse sera
ensuite recopi\'{e} dans le {\tt .lc}. Bien entendu cela n'est
op\'{e}rationnel pour chacun des modules que si le fichier {\tt .lc}
pr\'{e}-existe au momnet de la fabrication du {\tt Makefile}.
Le champ \|module-extension| est particuli\`{e}rement utile
pour l'utilisateur d\'{e}sireux de maintenir sa connaissance des modules
dans les {\tt .lc} consid\'{e}r\'{e}s alors comme fichiers sources des {\tt
.lm}. \`{A} l'oppos\'{e}, l'utilisateur d\'{e}sireux de stocker toutes les
informations relatives \`{a} l'analyse de ses modules, dans le champ
\|analyzer-options| (\|-import|, \|-export|, \|-allexport|, \|-files|,
\|-include|, \|-includeflag|) devra \'{e}viter
d'utiliser le champ 
\|module-extension| sous peine de voir des informations redondantes
ou incompatibles d'une
m\'{e}thodes de stockage \`{a} l'autre.

\item Dans le cas du {\tt Makefile} de compilation, on commencera par
recopier le {\tt .lc} en {\tt .lm} avant de lancer la compilation du
module. Le compilateur {\tt complice} \'{e}crira ses donn\'{e}es dans le
{\tt .lm} mais ne touchera pas au {\tt .lc}, le laissant ainsi plus
lisible.

\item Dans le cas de l'option \|-build|, la base sera fabriqu\'{e}e \`{a}
partir des informations contenues dans les {\tt ".lc"}.

\end{itemize}
\begin{Side}{\bf Attention} 
L'utilisation de \|module-extension| est d\'{e}conseill\'{e}e
avec l'option d'analyse \|-r|. En effet, une analyse r\'{e}cursive ne
garantit pas la coh\'{e}rence des "{\tt .lc}" avec les "{\tt .lm}",
cette garantie \'{e}tant r\'{e}alis\'{e}e par le {\tt Makefile} d'analyse, 
\end{Side}

\item {\Large \|modules|}\,: si elle existe, cette clef contient la
liste exhaustive de tous les noms de fichiers de description de
modules du projet. Cette clef est prioritaire sur \|modules-lists|.
Si cette liste contient la cha\^{\i}ne de caract\`{e}res
{\tt "all"}, alors tous les fichiers dans tous les r\'{e}pertoires du
projet seront pris en compte.

\item {\Large \|modules-files|}\,: si on d\'{e}sire g\'{e}rer compl\`{e}tement 
les noms des fichiers source de chaque module dans une
d\'{e}finition de projet, on utilisera la clef \|modules-files|. Il doit
s'agir d'une A-liste dont le premier \'{e}l\'{e}ment d'une sous-liste
est le nom du module (cf
option d'analyse \|-defmodule|) et le reste de la sous liste, les noms
complets (extension comprise) des fichiers sources tels qu'il
apparaitront dans le champ \|files| de la description modulaire.
Les noms des modules sont eux-m\^{e}mes g\'{e}r\'{e}s par les m\'{e}caniques
d\'{e}crites pour le champs \|modules-lists|.
\begin{Code*}
Le cas de mymod peut se resoudre de la facon suivante :

(define-rt-project myproject
  ...
  modules-files ((mymod mymod.ll mymod.lh mymod.li))
  ...
\end{Code*}
Cette clef est exploit\'{e}e par les options \|-init| et \|-defmodule| de
l'{\em Analyseur de Modules}. Voir \'{e}ventuellement le contenu
des {\tt .lc}, si la clef \|module-extension| est utilis\'{e}.

\item {\Large \|modules-lists| {\em list}}: cette clef est
destin\'{e}e \`{a} informer 
pr\'{e}cis\'{e}ment le projet sur les modules \`{a} traiter. 
Si \|modules-lists| existe, elle permet de
sp\'{e}cifier une liste de fichiers ({\tt *.lst}) contenant la liste des modules
composant le projet. Les fichiers ({\tt *.lst})  ainsi r\'{e}f\'{e}renc\'{e}s
par \|modules-lists| 
sont recherch\'{e}s dans chacun des 
r\'{e}pertoires de la clef \|directories| ou \|ll-module-directory|.

\begin{Side}{\bf Remarque}
Si aucun des fichiers 
sp\'{e}cifi\'{e}s avec la clef \|modules-list| ne figure dans un r\'{e}pertoire
donn\'{e} par la cl\'{e} \|directories| alors {\bf aucun}
module de {\em ce r\'{e}pertoire} ne sera pris en compte pour construire
les tables d'analyse. En mode \|-verbose 2|, un message est affich\'{e}
dans ce cas.
\end{Side}

\item {\Large \|project-file|}\,: cette clef, si elle existe, doit
d\'{e}signer le nom du fichier contenant la d\'{e}finition du projet (en
fait, l\`{a} o\`{u} se trouve le {\tt define-rt-project} d\'{e}finissant le
projet concern\'{e}). Cette
valeur peut \^{e}tre du type {\tt string} ou {\tt pathname}.  \\
Par d\'{e}faut, le nom de ce fichier sera calcul\'{e} \`{a} partir du nom du
projet, suffix\'{e} par l'extension {\tt ".prj"} et rang\'{e} dans le
r\'{e}pertoire d\'{e}sign\'{e} par la clef \|system-directory| (ou sa valeur
par d\'{e}faut s'il elle n'existe pas).
Cette clef est exploit\'{e}e par l'option \|-init| de l'{\em Analyseur de
Modules}.

\item {\Large \|ref-file|}\,: cette clef, si elle existe, doit d\'{e}signer
le nom du fichier des r\'{e}f\'{e}rences du projet. C'est dans ce fichier
que l'{\em Analyseur de Modules} stocke tous les informations relatives aux
analyses de tous les modules du projet. Cette
valeur peut \^{e}tre du type {\tt string} ou {\tt pathname}.  \\
Par d\'{e}faut, le nom de ce fichier sera calcul\'{e} \`{a} partir du nom du
projet, suffix\'{e} par l'extension {\tt ".ref"}, et rang\'{e} dans le
r\'{e}pertoire d\'{e}sign\'{e} par la clef \|crunch-directory| (ou sa valeur
par d\'{e}faut si elle n'existe pas).
Cette clef est exploit\'{e}e par toutes les options principales
de l'{\em Analyseur de Modules}.

\item {\Large \|required-projects|}\,: les ``sous-projets{''} qui sont
n\'{e}cessaires \`{a} l'utilisation de ce projet. La valeur de cette clef
est une liste de symboles d\'{e}signant les noms des projets requis. Par
exemple: 

\begin{Code*}
(define-rt-project smeci
   required-projects (smstr)
   ...)
\end{Code*}


\item {\Large \|root-directory|}\,: le r\'{e}pertoire ``racine{''} 
du projet \|name|. L'utilisation de cette clef
permet de sp\'{e}cifier des chemins relatifs pour les autres clefs de
r\'{e}pertoires telles que \|directories| par exemple.
Lorsque cette clef est omise, il faut sp\'{e}cifier la clef \|directories|
avec des chemins absolus. Pour \Aida:

\begin{Code*}
root-directory #u"/usr/ilog/aida/"
\end{Code*}

\item {\Large \|system-directory|}\,: le r\'{e}pertoire qui va contenir
tous les fichiers directement utiles \`{a} la gestion du projet:
{\tt Makefiles}, d\'{e}finitions de projets, etc. Par d\'{e}faut,
\|system-directory| prend la valeur de \|root-directory|.

\end{itemize}

%-----------------------------------------------------------------------
\Section {LL2LM}
%-----------------------------------------------------------------------
La commande d'analyse accepte un certain nombre d'options dont les
principales sont abord\'{e}es dans le chap\^{\i}tre 2. Nous allons \'{e}num\'{e}rer
ici l'ensemble des options, pr\'{e}sent\'{e}es dans l'ordre alphab\'{e}tique.

\begin{itemize}

\item {\Large \|-allexport|}\,: dans le cas o\`{u} l'utilisateur demande de
g\'{e}n\'{e}rer (ou de mettre \`{a} jour) une description modulaire, toutes les
fonctions du module sont export\'{e}es (champ \|export| de la
description). Cela permet de lib\'{e}rer une contrainte lors du passage
de l'interpr\'{e}t\'{e} en compil\'{e}, qui concerne la visibilit\'{e} des
fonctions d'un module. Rappelons que les m\'{e}thodes, qui sont
appel\'{e}es dynamiquement {\em doivent} \^{e}tre export\'{e}es (voir
\'{e}galement \|-dynamic| pour ce cas pr\'{e}cis).
\begin{Side} {\bf Remarque}
L'utilisateur d\'{e}sireux de ne prendre aucun risque d\^{u} \`{a}
d'eventuelles fonctions non export\'{e}es, utilisera \|-allexport| dans
un premier temps, quite \`{a} optimiser cela par la suite.
\end{Side}

\item {\Large \|-build|}\,: cette option principale est utilis\'{e}e pour
construire les tables de description d'un contexte d'analyse. 
Cette option est destin\'{e}e \`{a} l'utilisateur 
qui d\'{e}sire fabriquer la table de descriptions de son projet sans
analyser les modules du projet concern\'{e} (cf option \|-project|). Cela
suppose donc que les fichiers descripteur de module ({\tt file.lm} ou
{\tt file.lc}) existent d\'{e}j\`{a}.

\item {\Large \|-delete| {\em module-name}, \|-del| {\em
module-name}}\,: cette option principale 
permet d'effacer toutes les occurences d'un module dans une base de
r\'{e}f\'{e}rence d'un projet ({\it project.ref}).

\item {\Large \|-defmodule| {\em module-name}}\,: cette option principale
permet de 
sp\'{e}cifier le nom du module devant apparaitre sous la clef \|defmodule| du
module. L'utilisation de cette option signifie \`{a} l'analyseur de
fabriquer un nouveau fichier de description modulaire du module {\em
module-name}.
Par d\'{e}faut le nom de ce fichier est calcul\'{e} \`{a}
partir du nom du module (cf \|defmodule|) suffix\'{e} par l'extension {\tt
.lm}. Ce fichier est alors plac\'{e} dans le r\'{e}pertoire sp\'{e}cifi\'{e} par
la clef {\bf ll-module-directory} s'il existe, sinon par celui
sp\'{e}cifi\'{e} par {\bf ll-object-directory} s'il existe, ou sinon dans
le m\^{e}me r\'{e}pertoire que le premier fichier source trouv\'{e} dans le champ
\|files|. Il est toujours possible d'imposer un path et un nom de
fichier avec l'option \|-output|. \\
Voir \'{e}galement les clefs de {\tt define-rt-project}\ : 
les fichiers source devant appara\^{\i}tre sous le champ \|files| sont
calcul\'{e}s par d\'{e}faut \`{a} partir du nom du module\,: 
si la clef {\tt modules-files} est pr\'{e}sente dans 
la description du projet concern\'{e} (cf option \|-project|) et
concerne ce module, c'est elle qui fournira la liste exhaustive
des fichiers source. Dans le cas contraire, 
si la clef {\tt extensions-list} est
pr\'{e}sente, elle doit
contenir une liste d'extensions ({\tt ll, li}, ...) \`{a} adjoindre \`{a}
cette racine pour fabriquer les noms des fichiers source, sinon, si
cette clef n'est pas pr\'{e}sente, le
nom du fichier source sera {\em module-name}.{\tt ll}.

\item {\Large \|-dependency| {\em level}, \|-dep| {\em level}}\,: pour
d\'{e}terminer le niveau de 
d\'{e}pendance du {\tt Makefile} engendr\'{e} (cf options d'analyse
\|-makefile| et \|-init|). \|level| peut prendre les valeurs 0, 1, 2\ :
\begin{itemize}
\item {\Large {\bf 0}}\, :
il y a une d\'{e}pendance minimale sur les points d'entr\'{e}e du
{\tt Makefile}\,:\\
\begin{itemize}
\item
Dans le cas d'une conjugaison avec l'option \|-makefile|, 
les {\tt .lo} d\'{e}pendent du {\tt .lm} et des {\tt .ll} des champs {\tt
files} et {\tt include} du module.
\item
Dans le cas d'une conjugaison avec l'option \|-init|, il n'y a aucune
d\'{e}pendance. 
\end{itemize}
\item {\Large {\bf 1}} (d\'{e}faut) \ :
il y a une d\'{e}pendance normale sur chaque entr\'{e}e du {\tt Makefile}\,: \\
\begin{itemize}
\item
Dans le cas d'une conjugaison avec l'option \|-makefile|, 
idem {\em level 0}, plus une d\'{e}pendance sur
les {\tt .lm} des modules import\'{e}s. 
\item
Dans le cas d'une conjugaison avec l'option \|-init|, on force
syst\'{e}matiquement l'ex\'{e}cution au moyen d'une d\'{e}pendance fictive sur
une entr\'{e}e \|work| jamais mise \`{a} jour.
\end{itemize}
\item {\Large {\bf 2}} \ :
on cr\'{e}e une d\'{e}pendance forte\,: \\
\begin{itemize}
\item
Dans le cas d'une conjugaison avec l'option \|-makefile|, 
idem {\em level 0}, plus une d\'{e}pendance sur
les {\tt .lo} des modules import\'{e}s.
\item
Dans le cas d'une conjugaison avec l'option \|-init|, le lancement de
l'analyse d'un module d\'{e}pend de ses fichiers sources, et de ses
modules import\'{e}s. On notera que cette derni\`{e}re possibilit\'{e} est
particuli\`{e}rement int\'{e}ressante lorsqu'on utilise le {\tt Makefile} d'analyse
en mode {\it -update} (en mode {\it -defmodule}, on ne connait
\'{e}videmment pas encore les modules import\'{e}s!).
\end{itemize}
\end{itemize}
Cette option se combine avec les options \|-init| ou \|-makefile|.

\item {\Large \|-dynamic|}\,: cette option sp\'{e}cifie \`{a} l'analyseur de
reconnaitre les fonction appel\'{e}es dynamiquement de la forme {\tt
'function}. Voir \'{e}galement le d\'{e}finisseur de formes dynamiques\,: {\tt
defdynamiccall} .
Cette option permet d'automatiser la d\'{e}tection des importations
n\'{e}cessaires pour une bonne utilisation des fonctions appel\'{e}es
dynamiquement.
\|-dynamic| se combine avec les options \|-defmodule| ou \|-update|.

\item {\Large \|-export function|}\,: pour ajouter une fonction {\tt
function} dans le champ \|export| du module analys\'{e}. 
Cette option
est cumulative et se combine avec les options \|-defmodule| ou
\|-update|. \\
\begin{Side}{\bf Remarque}
L'{\em Analyseur de Modules} utilise cette option pour ses
besoins propres 
lors d'analyses r\'{e}cursives destin\'{e}es \`{a} mettre \`{a} jour
les modules et le fichier de r\'{e}f\'{e}rences concern\'{e}s.
\end{Side}

\item {\Large \|-files file.ll|}\,: pour ajouter le fichier {\tt
file.ll} dans le champ \|files| du module analys\'{e}. 
Cette option est cumulative.
\|-files| se combine avec les options \|-defmodule| ou \|-update|.

\item {\Large \|-import module|}\,: pour ajouter
le module {\tt module} dans le champ \|import| du module analys\'{e}. 
Cela est
particuli\`{e}rement utile si l'{\em Analyseur de Modules} ne ``voit''
pas certains 
modules, ou bien lorsque le module n'est pas encore int\'{e}gr\'{e} dans un
contexte d'analyse. Cette option est cumulative.
\|-import| se combine avec les options \|-defmodule| ou \|-update|.

\item {\Large \|-include file.ll|}\,: pour ajouter le fichier {\tt
file.ll} dans le champ \|include| du module analys\'{e}. Cette option
est cumulative.
\|-include| se combine avec les options \|-defmodule| ou \|-update|.

\item {\Large \|-includeflag|}\,: cette option sp\'{e}cifie \`{a} l'analyseur
d'utiliser le champ \|include| lorsqu'il fabrique ou met \`{a} jour un
module. Cela peut \^{e}tre utile lorsque le programmeur ne d\'{e}sire pas
revoir certaines formes {\tt eval-when} dans ses sources d\'{e}j\`{a}
analys\'{e}s. 
\|-includeflag| se combine avec les options \|-defmodule| ou \|-update|.

\item {\Large \|-init|}\,: cette option principale permet d'engendrer un
{\tt Makefile} d'analyse pour un projet donn\'{e}. Ce {\tt Makefile} r\'{e}alise
par d\'{e}faut l'\'{e}quivalent de l'option \|-update| sur tous les modules
du projet. \\
L'ensemble des modules du projet sont d\'{e}finis \`{a} partir de la
d\'{e}finition du projet d'une des fa\c{c}ons suivantes\,:
\begin{enumerate}
\item la clef \|modules| permet de sp\'{e}cifier explicitement la liste
des modules composant ce projet. Cette clef peut contenir le mot clef
{\tt all}\,: nous sommes alors dans le cas 3;
\item la clef \|modules-lists| permet de sp\'{e}cifier une liste de
fichiers dans lesquels on trouvera les listes des modules d\'{e}crivant
ce projet. Cette liste est amput\'{e}e des modules \'{e}num\'{e}r\'{e}s dans la
clef  \|exclude-modules|. \\
Cette clef peut contenir le mot clef {\tt all}\,: nous sommes alors
dans le cas 3;
\item la liste des modules est constitu\'{e}e \`{a} partir de l'ensemble des
fichiers source trouv\'{e}s dans les r\'{e}pertoires sp\'{e}cifi\'{e}s par la clef
\|directories|. 
\end{enumerate}
Le {\tt Makefile} poss\`{e}de des entr\'{e}es pr\'{e}d\'{e}finies: {\tt init1}, {\tt
init2} , {\tt update} et {\tt clean}\,:
\begin{itemize}
\item \|init1| \ : n'est \`{a} utiliser qu'une seule fois lors de la
toute premi\`{e}re phase d'analyse. En effet de nombreux messages de
l'analyseur sont uniquement dus \`{a} l'absence de fichier de
r\'{e}f\'{e}rence du projet.
\item \|init2| \ : est \`{a} utiliser une seule fois \'{e}galement apr\`{e}s
\|init1|, ou, par la suite si l'utilisateur d\'{e}sire refabriquer tous ses
modules. 
\item \|update| (d\'{e}faut)\ : est \`{a} utiliser apr\`{e}s \^{e}tre intervenu
manuellement sur chaque message important ({\tt ** ...}) issus de la
pr\'{e}c\'{e}dente analyse.
\item \|clean| \ : efface tous les fichiers de description modulaire
du projet et la base de r\'{e}f\'{e}rences du projet.
\end{itemize}
Il est possible de pr\'{e}ciser des options d'analyse, module par module,
ou globalement\,: voir la clef {\tt analyzer-options} de {\tt
define-rt-project}. \\
Voir \'{e}galement l'option \|-dependency|.

\item {\Large \|-load file|}\,: permet de charger le fichier {\em file}
dans l'environnement d'analyse. En g\'{e}n\'{e}ral, il s'agit d'un fichier
de description de projet du type {\tt prjname.prj} et se place avant
l'option \|-project|. 
Cette option est cumulative.

\item {\Large \|-makefile|, \|-make|}\,: cette option
principale permet 
d'engendrer un {\tt Makefile} de compilation modulaire ({\tt cf
complice}) de tous les modules du projet concern\'{e} (cf option
\|-project|). Par d\'{e}faut le fichier r\'{e}sultat de la compilation 
d'un module, par {\tt complice} ({\tt
.lo}) est rang\'{e} dans le r\'{e}pertoire sp\'{e}cifi\'{e} par la clef {\tt
ll-object-directory} s'il existe, ou sinon \`{a} c\^{o}t\'{e} du fichier de
description modulaire ({\tt .lm}). \\
Ce {\tt Makefile} offre certaines entr\'{e}es pr\'{e}d\'{e}finies:
\begin {enumerate}
\item \|all| (d\'{e}faut)\ :\ 
permet de mettre \`{a} jour les compilations de tous les modules du projet.
\item \|clean|\ :\ 
permet d'effacer les fichiers de compilation ({\tt .lo}) de tous les
modules du projet.
\item \|i|\ :\ 
permet d'entrer dans la boucle {\tt toplevel} interactive du
compilateur, avec tous les flags de compilation charg\'{e}s dans
l'environnement. Tr\`{e}s utile pour le d\'{e}bbogage.
\end{enumerate}
Voir \'{e}galement l'option \|-dependency| pour jouer sur le niveau de
d\'{e}pendance des entr\'{e}es du {\tt Makefile}.\\
En m\^{e}me temps que le {\tt Makefile}, est g\'{e}n\'{e}r\'{e} un fichier {\bf
project.pth} contenant une d\'{e}finition de la variable {\tt
\#:system:path} avec tous les chemins d'acc\`{e}s n\'{e}cessaires \`{a}
l'execution des modules du project {\tt project>}.

\item {\Large \|-merge|}\,: cette option principale permet de
concat\'{e}ner les fichiers de 
r\'{e}f\'{e}rences de deux projets diff\'{e}rents, en un seul fichier de
r\'{e}f\'{e}rences. Cela peut \^{e}tre utile pour simplifier des d\'{e}pendances
de projets import\'{e}s r\'{e}cursivement au moyen de la clef
\|required-project|. Cette option est \'{e}galement utile dans certains
cas d'utilisation des {\em meta-modules}.

\item {\Large \|-meta|}\,: cette option principale permet 
de g\'{e}n\'{e}rer un module r\'{e}unissant les clefs \|import| et \|export| de
tous les modules du projet. On appelle un tel module un {\bf
meta-module}. Ce {\tt meta-module} peut entre autre \^{e}tre utilis\'{e}
pour charger tous les modules du projet en une seule op\'{e}ration (voir
{\tt loadmodule}, {\em Manuel de R\'{e}f\'{e}rence} \LeLisp).

\item {\Large \|-nowrite|}\,: permet de r\'{e}aliser une analyse compl\`{e}te
(nouveau module ou mise \`{a} jour) sans en \'{e}crire le r\'{e}sultat ni dans
le fichier de description modulaire, ni dans le fichier de r\'{e}f\'{e}rence
du projet. Cette option est utile pour ``voir{''} ce que va faire
l'analyseur. 
\|-nowrite| se combine avec les options \|-defmodule| ou \|-update|.

\item {\Large \|-output| {\em target}, \|-o| {\em target}}\,: pour
sp\'{e}cifier le nom 
du fichier o\`{u} \'{e}crire les r\'{e}sultats de l'analyse. Cette option peut
se conjuguer avec les options\,:
\begin {itemize}
\item \|-defmodule| lorsqu'il s'agit de
fabriquer un fichier descripteur de module pour la 1\`{e}re fois,
\item \|-makefile| lorsqu'il s'agit d'engendrer le fichier {\tt Makefile} de
compilation modulaire,
\item \|-init| lorsqu'il s'agit d'engendrer le fichier {\tt Makefile}
d'initialisation d'analyse,
\item \|-meta| lorsqu'il s'agit d'engendrer un {\em meta-module}.
\item \|-merge| lorsqu'il s'agit d'engendrer un nouveau fichier de
r\'{e}f\'{e}rences. 
\end{itemize}
Cette option est incompatible avec l'option \|-update| qui sp\'{e}cifie
elle-m\^{e}me le nom du fichier sur lequel l'analyse se fait.

\item {\Large \|-project| {\em project}, \|-p| {\em project}}\,: pour sp\'{e}cifier un
contexte d'analyse. La fonction
\|(declared-rt-projects)| permet de conna\^{\i}tre l'ensemble des projets
qui sont d\'{e}clar\'{e}s (mais non activ\'{e}s) dans l'environnement de
l'{\em Analyseur de Modules}. Cette option est l'une des plus importantes
puisqu'elle doit presque toujours \^{e}tre sp\'{e}cifi\'{e}e, quelque soit
l'option principale.

\item {\Large \|-recursive|, \|-r|}\,: cette option permet de mettre \`{a} jour,
r\'{e}cursivement, l'analyse des modules import\'{e}s. Si un module utilise
une fonction d\'{e}finie dans un autre module du m\^{e}me projet, mais non
export\'{e}e, et si l'option \|-r| est mise, l'analyseur lancera une
analyse r\'{e}cursive sur ce module import\'{e}, en imposant d'exporter la
fonction en question. Cette analyse r\'{e}cursive se fait mode {\tt -v
0}. Si l'analyse r\'{e}cursive \'{e}choue (erreur...), le message \|W.120|
s'affiche. \\
Cette option est positionn\'{e}e par les entr\'{e}es du {\tt Makefile}
d'analyse : {\tt init1, init2} et {\tt update}. Elle n'est pas
activ\'{e}e par d\'{e}faut, mais est tr\`{e}s utile dans une optique ``tout
automatique{''}. 

\item {\Large \|-update| {\em file.lm}, \|-u| {\em file.lm}}\,: cette option
principale est utilis\'{e}e pour demander la
mise \`{a} jour d'un module d\'{e}j\`{a} existant, dont on d\'{e}sire conserver
les informations d\'{e}j\`{a} existantes. Des commentaires seront
\'{e}ventuellement ajout\'{e}s dans le fichier {\tt file.lm} mais ni les
importations, ni les exportations semblant inutiles ne seront
supprim\'{e}es. Les fonctions export\'{e}es sans \^{e}tre d\'{e}finies dans
les sources de ce module engendrent le warning {\tt W.105}.

\item {\Large \|-usage|, \|-help|}\,: pour obtenir la description des options de
la commande \|ll2lm|.

\item {\Large \|-verbose| {\em level}, \|-v| {\em level}}\,: pour d\'{e}terminer le
niveau de 
messages de l'analyse. \|level| peut prendre les valeurs 0, 1, 2\,: 
\begin{itemize}
\item 0 (d\'{e}faut)\ :
niveau minimun de messages: seuls les messages importants
n\'{e}cessitant une intervention, sont imprim\'{e}s \`{a} l'\'{e}cran,
\item 1 \ :
l'analyseur d\'{e}taille l'ensemble des op\'{e}rations effectu\'{e}es,
\item 2 \ :
l'analyseur \'{e}met en plus un diagnostic de son analyse en commentant
ses choix.
\end{itemize}
Le niveau de messages requis lors de la fabrication d'un {\tt Makefile}
influence le niveau de messages durant l'utilisation de ce {\tt Makefile}.
Cette option se combine avec toutes les autres options.

\end{itemize}

