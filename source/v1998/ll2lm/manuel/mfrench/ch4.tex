%=========================================================================
\Chapter {4} {Messages d'analyse}
%=========================================================================

Nous avons vu pr\'{e}c\'{e}demment dans les options d'analyse, qu'il est
possible de choisir un {\it niveau de messages} (option
\|-verbose|). L'id\'{e}e est de 
permettre \`{a} l'utilisateur de ne pas se sentir submerg\'{e}
d'informations au moment o\`{u} il n'en attend pas, et inversement, de
lui offrir un maximun de renseignements d\`{e}s qu'il le demande. \\
Les 3 niveaux disponibles sont, dans l'ordre croissant en quantit\'{e}
d'informations\,: 
\begin{itemize}
\item {\Large {\bf -v 0}}\,: le niveau "{\bf corrections}", le plus
silencieux\,: seuls les
messages n\'{e}cessitant une 
intervention du programmeur, sont affich\'{e}s. Ils sont tous pr\'{e}c\'{e}d\'{e}s
d'une  {\em double \'{e}toile}\,: {\tt **}
\item {\Large {\bf -v 1}}\,: le niveau "{\bf avertissements}"\ :
l'{\em Analyseur de Modules} 
affiche en plus un
message pour chaque op\'{e}ration effectu\'{e}e (chargement d'un
fichier, recherche d'une fonction inconnue, d\'{e}clenchement d'une
importation, etc). Ils sont tous pr\'{e}c\'{e}d\'{e}s
d'un  {\em double point}\,: {\tt ..}
\item {\Large {\bf -v 2}}\,: le niveau "{\bf diagnostic}"\ : l'{\em
Analyseur de Modules} affiche 
en plus un diagnostic complet \`{a} propos de l'analyse r\'{e}alis\'{e}e.
\end{itemize}

Les messages les plus importants sont ceux du niveau "corrections"
facilement rep\'{e}rables \`{a} leur {\em double
\'{e}toile}. Les messages les plus importants sont num\'{e}rot\'{e}s afin
d'\^{e}tre plus facilement identifiables, et mieux document\'{e}s.

Ces messages importants, peuvent appara\^{\i}tre de deux fa\c{c}ons
diff\'{e}rentes\,: {\tt warning} ({\bf W.xxx}) ou {\tt error} ({\bf E.xxx}).
Cela ne modifie pas leur signification\,: un {\tt warning} veut
dire que l'analyse peut continuer sans probl\`{e}me, mais qu'une
intervention manuelle sera certainement n\'{e}cessaire pour le corriger.
Une {\tt error} signifie que l'analyseur a d\^{u} interrompre 
l'analyse en partie ou compl\`{e}tement.

\begin{Code*}
Exemple:
** W.102 : unknown abbrev : I can't find it anywhere : myabbrev
** W.101 : unknown function : I can't find it anywhere : turlututu
\end{Code*}
Une {\tt error} signifie que l'analyse n'a pas pu se faire sur la
forme compl\`{e}te concern\'{e}e. Il s'agit sans doute de l'\'{e}valuation
d'une {\tt toplevel-form}: une intervention manuelle devra corriger
cela afin de pouvoir relancer l'analyse. En g\'{e}n\'{e}ral, une {\tt error}
est suivi d'un avertissement du style:
\begin{Longcode*}
** <intitule' de l'erreur LL> : unrecoverable error; skip expression : <S-expr>

Exemple:
** E.101 : unknown function : I can't find it anywhere : mydefun
** undefined function : unrecoverable error; skip expression : (mydefun foo)
\end{Longcode*}

\`{A} la fin d'une analyse, un message r\'{e}capitulatif des warnings et
erreurs rencontr\'{e}s pendant l'analyse, apparait sous la forme
suivante:
\begin{itemize}
\item {\tt "** <module> : **! There were warnings during the analyzis : (W.n ..)"}
s'il y a eu des warning, et que le mode verbose est sup\'{e}rieur \`{a} 0.
La liste des warnings rencontr\'{e}s pendant l'analyse est affich\'{e}e.
\item {\tt "** <module> : **! There were errors during the analyzis : (E.n ..)"}
s'il y a eu des erreurs. La liste des erreurs rencontr\'{e}es pendant
l'analyse est affich\'{e}e. 
\end{itemize}
Ces 2 messages ont pour vocation de faciliter la recherche des modules
erron\'{e}s dans une grosse liste de messages, concernant de nombreux
modules. \\
Tous les messages qui suivent sont propos\'{e}s en anglais par d\'{e}faut,
mais existent \'{e}galement en fran\c{c}ais (cf {\tt current-language}).
Voici la liste exhaustive des messages importants et leur
signification\,:
\begin{itemize}
\item {\Large {\bf 100}}\,: {\tt  more than one analysis in a session may
give incorrect results : {\em module}} \\
Ce message appara\^{\i}t si deux analyses sont faites l'une apr\`{e}s
l'autre, dans une m\^{e}me session. Cela est {\it fortement}
d\'{e}conseill\'{e} car la premi\`{e}re analyse aura fait des effets de bord
dans la session (d\'{e}finitions de fonctions, chargement de fichiers,
etc) qui amputeront d'autant l'analyse suivante. \\
{\bf Correction}\,: lancer deux analyses dans deux sessions distinctes,
ou utiliser syst\'{e}matiquement {\tt ll2lm}.

\item {\Large {\bf 101}}\ : {\tt unknown function: I can't find it anywhere
: {\em function}}\\
Ce message appara\^{\i}t \`{a} chaque fois que l'{\em Analyseur de Modules}
rencontre une fonction dont il ne trouve pas la d\'{e}finition. En
g\'{e}n\'{e}ral, d\`{e}s qu'il rencontre une fonction inconnue, il  
cherche le module qui d\'{e}finit cette fonction, et d\'{e}clenche son
importation. Si aucun des modules pr\'{e}sents dans le fichier des
r\'{e}f\'{e}rences, ne d\'{e}finit cette fonction (export\'{e}e ou non), le message
{\tt 101} est \'{e}mis. \\
{\bf Correction}\,: le module d\'{e}finissant la
fonction en question n'a sans doute pas encore \'{e}t\'{e} analyse.: Le message
dispara\^{\i}tra \`{a} l'analyse suivante. Sinon forcer l'importation du
module en question soit via l'option d'analyse \|-import|, soit en
ajoutant ce module dans la liste des importations du {\tt .lm} ou du
{\tt .lc} en mode \|update|.

\item {\Large {\bf 102}}\ : {\tt  unknown abbrev: I can't find it anywhere :
{\em abbrev} }\\
Idem {\tt 101}, mais pour les abr\'{e}viations inconnues.

\item {\Large {\bf 103}}\ : {\tt unknown sharp macro: I can't find it
anywhere : {\em sharp}}\\
Idem {\tt 101}, mais pour les {\tt sharp-macro} inconnues.

\item {\Large {\bf 104}}\ : {\tt parent structure package {\em symbol} not found
for : {\em struct}}\\
Idem {\tt 101}, mais pour les structures ``parent{''} inconnues.

\item {\Large {\bf 105}}\ : {\tt unknown exported user function(s) : {\em function}}\\
Ce message apparait lorsque l'utilisateur a impos\'{e} d'exporter des
fonctions que l'analyseur ne connait pas. Il peut arriver qu'une
d\'{e}finition de fonction reste cach\'{e}e pour l'analyseur\,:
l'utilisateur utilisera alors l'option d'analyse \|-export| ou
imposera cet exportation via la clef \|export| directement dans le
{\tt .lm} ou le {\tt .lc} en mode \|-update|.
Ce warning est \'{e}mis afin que l'utilisateur puisse v\'{e}rifier que cette
fonction doit bien \^{e}tre export\'{e}e. Si tel n'est pas le cas {\tt
complice} \'{e}mettra le warning 6.\\
{\bf Correction}\,: Il s'agit soit d'un cas de d\'{e}finition de fonction
non reconnue par l'analyseur, auquel cas on conservera le {\tt W.105},
soit il s'agit d'un export abusif, auquel cas, l'utilisateur devra
enlever cette fonction de la liste des exportations du {\tt .lm} ou
{\tt .lc}, ou supprimer l'option d'analyse \|-export| correspondante.

\item {\Large {\bf 106}}\ : {\tt structure {\em struct} not found when
running : {\em form}}\\
Idem {\tt 101}, mais pour les structures inconnues, et plus
pr\'{e}cis\'{e}ment pour les objets ayant subis l'erreur {\tt
errnotarecordoratclass} du module {\tt microceyx}.

\item {\Large {\bf 109}}\ : {\tt the module {\em module} appears in several
project directories : ({\em dir1} {\em dir2} ...)}\\
Ce message appara\^{\i}t \`{a} la cr\'{e}ation du {\tt Makefile} d'analyse (option
\|-init|) lorsque que l'analyseur a trouv\'{e} deux fichiers de m\^{e}me
nom dans deux r\'{e}pertoires diff\'{e}rents. Cette phase de fabrication du
{\tt Makefile} d'analyse devant calculer automatiquement les noms des modules
\`{a} partir des noms des fichiers trouv\'{e}s dans les r\'{e}pertoires du
projet, il est pr\'{e}ferable que le programmeur r\'{e}solve ce conflit et
recommence cette \'{e}tape. Sinon l'analyseur ne conserve qu'un seul nom
dans sa liste des modules, et le fichier utilis\'{e} sera le premier
trouv\'{e} en fonction des r\'{e}pertoires du projet (clef {\tt
directories}). \\
{\bf Correction}\,: effacer ou renommer un des deux fichiers.

\item {\Large {\bf 110}}\ : {\tt the projects {\em project} and {\em project} share
a module name : {\em module}}\\
Ce message appara\^{\i}t lors du chargement du fichier de r\'{e}f\'{e}rences
d'un projet (soit directement, 
soit via la clef {\tt required-projects}), si un m\^{e}me nom de module
appara\^{\i}t dans les deux projets.\\
{\bf Correction}\,: le programmeur doit imp\'{e}rativement intervenir, et
renommer un des deux modules dans un projet. Sinon, c'est le dernier
projet charg\'{e} qui impose son module.

\item {\Large {\bf 111}}\ : {\tt function {\em function} must be defined inside
an EVAL-WHEN(COMPILE) in module : {\em module}}\\
Ce message appara\^{\i}t lors de l'\'{e}valuation d'une {\tt
toplevel-form} inconnue. Comme pour le message {\tt 101}, la fonction
inconnue est recherch\'{e}e dans l'environnement d'analyse.
Son statut dans le fichier ({\tt
toplevel-form}) impose son \'{e}valuation compl\`{e}te.
Cette fonction doit \^{e}tre d\'{e}finie \`{a} l'int\'{e}rieur d'un {\tt (eval-when
(compile ...) ...)} car le compilateur aura le m\^{e}me probl\`{e}me\,: \'{e}valuer
cette forme avant de la compiler. \\
{\bf Correction}\,: englober cette d\'{e}finition dans un {\tt (eval-when
(compile ...) ...)} dans le fichier qui la d\'{e}finit, ou recommencer
l'analyse avec l'option \|-includeflag| qui va forcer le chargement du
fichier o\`{u} se trouve la d\'{e}finition.

\item {\Large {\bf 112}}\ : {\tt abbreviation {\em abbrev} must be defined
inside an EVAL-WHEN(COMPILE) in module : {\em module}}\\
idem {\tt 111} mais l'erreur appara\^{\i}t \`{a} la lecture d'une
abr\'{e}viation.

\item {\Large {\bf 113}}\ : {\tt sharp macro {\em sharp} must be defined inside
an EVAL-WHEN(COMPILE) in module : {\em module}}\\
idem {\tt 111} mais l'erreur appara\^{\i}t \`{a} la lecture d'une
{\tt sharp-macro}

\item {\Large {\bf 114}}\ : {\tt structure {\em struct} must be defined inside
an EVAL-WHEN(COMPILE) in module : {\em module}}\\
idem {\tt 111} mais l'erreur appara\^{\i}t \`{a} la lecture d'une
structure \`{a} h\'{e}ritage.\\
{\em Remarque\,:} il peut arriver que ce message soit incongru. En
effet, si l'analyseur voit une structure dont le nom est lui-m\^{e}me
packag\'{e}, il va croire qu'il s'agit d'une structure \`{a} h\'{e}ritage et va
chercher \`{a} d\'{e}clencher l'importation du module qui d\'{e}finit la
structure parent, alors qu'il n'en est rien. Il existe alors plusieurs
interventions possibles\,:
\begin{itemize}
\item Les structures \`{a} h\'{e}ritage n'\'{e}tant consid\'{e}r\'{e}es qu'avec la
variable syst\`{e}me\\
{\tt \#:system:defstruct-all-access-flag} \`{a} {\tt
T}, on peut mettre cette valeur \`{a} {\tt NIL}. cf chap\^{\i}tre 5 du Manuel
de R\'{e}f\'{e}rence \LeLisp .
\item Une variable interne de l'analyseur contient la liste des noms
de packages qui ne sont pas des structures parentes : {\tt
\#:crunch:not-parent-structures}. Cette liste contient le symbole {\tt
tclass} par d\'{e}faut.
\end{itemize}

\item {\Large {\bf 115}}\ : {\tt this module seems have been modified; you
must analyze it from scratch : {\em module}}\\
Ce message appara\^{\i}t lorsqu'un fichier source a \'{e}t\'{e} modifi\'{e}
sans \^{e}tre re-analys\'{e}. L'environnement d'analyse, et le fichier de
r\'{e}f\'{e}rences du projet en particulier, ne sont pas \`{a} jour. \\
{\bf Correction}\,: recommencer l'analyse du module modifi\'{e}. Si le
message {\tt 115} appara\^{\i}t justement pendant l'analyse du module
modifi\'{e}, il conviendra de recommencer l'analyse avec l'option
\|-defmodule|, et apr\`{e}s avoir effac\'{e} ce module du fichier des
r\'{e}f\'{e}rences (cf option d'analyse \|-delete|).

\item {\Large {\bf 116}}\ : {\tt function {\em function} is not in an
EVAL-WHEN(COMPILE); putting module in INCLUDE key : {\em module}}\\
Ce message appara\^{\i}t dans le m\^{e}me cas que le message {\tt 111} et le
remplace, si l'option
d'analyse \|-includeflag| est positionn\'{e}e.

\item {\Large {\bf 117}}\ : {\tt abbrev {\em abbrev} is not in an
EVAL-WHEN(COMPILE); putting module in INCLUDE key : {\em module}}\\
idem {\tt 116} mais pour les abr\'{e}viations.

\item {\Large {\bf 118}}\ : {\tt sharp macro {\em sharp} is not in an
EVAL-WHEN(COMPILE); putting module in INCLUDE key : {\em module}}\\
idem {\tt 116} mais pour les {\tt sharp-macro}.

\item {\Large {\bf 119}}\ : {\tt structure {\em struct} is not in an
EVAL-WHEN(COMPILE); putting module in INCLUDE key : {\em module}}\\
idem {\tt 116} mais pour les structures \`{a} h\'{e}ritage.

\item {\Large {\bf 120}}\ : {\tt Error during recursive analysis. You must
reanalyze this module : {\em module}}\\
Ce message appara\^{\i}t lorsqu'on a utilis\'{e} l'option d'analyse \|-r| et
qu'une analyse r\'{e}cursive s'est mal pass\'{e}e. 
Avec l'option \|-r|, l'{\em Analyseur de Modules} peut d\'{e}clencher une
analyse r\'{e}cursive sur les 
modules d\'{e}finissant des fonction internes devant \^{e}tre export\'{e}es
pour les besoins du module en cours d'analyse. \\
{\bf Correction}\,: le programmeur devra relancer l'analyse sur les
modules d\'{e}sign\'{e}s par ce message avant de recommencer l'analyse en
cours.

\item {\Large {\bf 121}}\ : {\tt Circular dependancy between modules. Please
redesign. : ({\em module} ...)}\\
Ce message appara\^{\i}t lorsque l'{\em Analyseur de Modules} d\'{e}c\`{e}le une
d\'{e}pendance circulaire entre plusieurs modules qui s'importent les uns
les autres. Cela n'emp\^{e}che pas
l'analyse de continuer, mais il convient de corriger cela avant la
phase de compilation par {\tt complice} pour lequel cela peut
s'av\'{e}rer tr\`{e}s grave (cf {\tt Warning 10} de complice, chapitre 13
de la documentation \LeLisp).\\
{\bf Correction}\,: concevoir un nouvel arbre de d\'{e}pendance des
modules. S'il n'y a pas d'autre solution, on cassera
une d\'{e}pendance en cr\'{e}ant un (ou des) appel
dynamique (voir aussi \|-dynamic|) \`{a} l'aide d'une forme comme {\tt
funcall}: 
\begin{Code*}
remplacer :  (foo 1 2 3)
par : (funcall 'foo 1 2 3)
\end{Code*}

\item {\Large {\bf 122}}\ : {\tt error in {\em module} during recoverable read
error - only partial analysis of module : {\em error}}\\
L'{\em Analyseur de Modules} \'{e}met ce message lorsqu'un certain type d'
erreur se d\'{e}clenche alors qu'il \'{e}tait d\'{e}j\`{a} en train de tenter de
rattraper une erreur de lecture (abr\'{e}viation inconnue,
macro-caract\`{e}re ...). Pour rattraper une erreur de lecture, l'{\em
Analyseur de Modules}  cherche dans les modules de son environnement
la d\'{e}finition permettant de reprendre cette erreur de lecture. Pour
cela, il charge soit les {\tt CPENV} de ces modules, soit les sources
(voir les options d'analyse \|-includeflag| ou \|-include|).
Les types d'erreur g\'{e}n\'{e}rant ce message {\bf 122} surviennent soit
pendant l'\'{e}valuation d'un {\tt CPENV}, soit pendant le chargement
d'un fichier source. Il est important de noter que l'analyse est
stopp\'{e}e dans la lecture du fichier \`{a} l'endroit exact o\`{u} a \'{e}t\'{e}
d\'{e}clench\'{e}e l'erreur de lecture initiale. Le reste de l'analyse ne se
fera donc que sur cette premi\`{e}re partie du fichier. Il conviendra de
relancer cette analyse apr\`{e}s correction de l'erreur. \\
{\bf Correction}\,: s'il
s'agit d'une erreur pendant l'\'{e}valuation du {\tt CPENV}, il
conviendra de mettre \`{a} jour l'environnement d'ex\'{e}cution (clef
\|required-projets| ou option d'analyze \|-import|). S'il s'agit d'une
erreur lors du chargement du fichier, v\'{e}rifier l\`{a} encore,
l'environnement d'ex\'{e}cution, et/ou v\'{e}rifier la pr\'{e}sence du fichier
ou les chemins d'acc\`{e}s selon la nature de l'erreur.

\item {\Large {\bf 123}}\ : {\tt function {\em function}\,: unknown
function type\,: {\em typ}}\\ 
L'{\em Analyseur de Modules} emet ce message lorsqu'il rencontre une
fonction de type inconnu (cf {\tt typefn}), ou qu'il ne sait pas
traiter dans ce contexte d'analyse. \\
{\bf Correction}\,: aucune. Cette erreur est \'{e}quivalente au {\tt
Warning 3} de {\tt complice} (voir chap\^{\i}tre 13 du manuel de
r\'{e}f\'{e}rence \LeLisp).

\item {\Large {\bf 124}}\ : {\tt bad construction in {\em function} : {\em error}}\\
L'{\em Analyseur de Modules} emet ce message lorsqu'il rencontre une
forme sp\'{e}ciale ({\tt lock}, {\tt letv}) inconnue de {\tt
complice}, ou lorsqu'il rencontre un probl\`{e}me \`{a} la macroexpansion
d'une macro. \\
{\bf Correction}\,: aucune.

\item {\Large {\bf 125}}\ : {\tt several modules ({\em module} ...) define :
{\em form}}\\
Ce message appara\^{\i}t lorsque l'{\em Analyseur de Modules} a rencontr\'{e}
une forme inconnue (fonction, abr\'{e}viation, ...) dont il a trouv\'{e} la
d\'{e}finition dans plusieurs modules. Si aucun des modules trouv\'{e}s
n'\'{e}tait d\'{e}j\`{a} import\'{e} (via l'option d'analyse \|-import|, ou d\'{e}j\`{a}
pr\'{e}sent dans le champ \|import| du module analys\'{e}), l'{\em Analyseur
de Modules} \'{e}met le message {\tt 125}. \\
{\bf Correction}\,: choisir lequel, parmi tous les modules trouv\'{e}s,
doit \^{e}tre pris pour fournir la bonne d\'{e}finition de la forme
inconnue. Recommencer l'analyse avec l'option \|-import
<nom_du_module_choisi>|, ou bien, si vous \^{e}tes en mode \|-update|,
ajouter le nom du module choisi
dans le champs {\tt import} du module analys\'{e}. 

\item {\Large {\bf 126}}\ : {\tt module not found; check the analysis
environment : {\em module}}\\
Ce message appara\^{\i}t lorsque l'{\em Analyseur de Modules} a besoin de
charger la description modulaire du module cit\'{e}, 
et que ce fichier reste introuvable dans l'environnement
d'analyse fourni. \\
{\bf Correction}\,: v\'{e}rifier la clef \|directories| dans la d\'{e}finition
du projet ({\tt define-rt-project}), et y ajouter le r\'{e}pertoire
contenant le fichier de description modulaire introuvable, avant de
recommencer l'analyse.

\item {\Large {\bf 127}}\ : {\tt file not found; check analysis environment
and project definition : {\em file}}\\
Ce message appara\^{\i}t lorsque l'{\em Analyseur de Modules} v\'{e}rifie si
tous les fichiers sources appara\^{\i}ssant sous le champ \|files| du
module en cours d'analyse sont accessibles dans l'environnement
d'analyse. Il peut soit s'agir d'une erreur de la part du programmeur
ayant initialis\'{e} ce module, soit, si on utilise le {\tt Makefile} d'analyse
(cf option \|-init|), un manque de pr\'{e}cision dans la d\'{e}finition du
projet ({\tt define-rt-project}). V\'{e}rifier en particulier les clefs
\|extensions-list| et \|modules-files| qui influencent le contenu du
champ \|files| lors de la g\'{e}n\'{e}ration automatique des modules. Il se
peut \'{e}galement qu'il 
s'agisse d'un probl\`{e}me de mise \`{a} jour de la clef \|directories|. \\
{\bf Correction}\,: revoir la d\'{e}finition du projet, si on utilise le
{\tt Makefile} d'analyse. Revoir le champ \|files| de la  description
modulaire de module concern\'{e}.
V\'{e}rifier la clef \|directories| de la d\'{e}finition
du projet, dans tous les cas.

\item {\Large {\bf 128}}\ : {\tt module {\em module} uses unexported definitions
from : {\em module}}\\
Ce message appara\^{\i}t lorsque l'{\em Analyseur de Modules} a trouv\'{e} une
d\'{e}finition recherch\'{e}e, mais que cette d\'{e}finition n'est pas
export\'{e}e par ce module (c'est une fonction interne), et que de plus,
ce module ne fait pas partie du projet en cours d'analyse, mais d'un
projet requis (cf \|required-projects|).\\
{\bf Correction}\,: utiliser une autre d\'{e}finition, ou revoir le projet
requis, et forcer l'exportation de la d\'{e}finition.

\item {\Large {\bf 131}}\ : {\tt several modules ({\em module} ...) define
function : {\em function}}\\
Ce message indique que plusieurs modules definissent la m\^{e}me
fonction.\\
{\bf Correction}\,: intervenir sur les modules concern\'{e}s afin de rendre
la d\'{e}finition unique.

\item {\Large {\bf 132}}\ : {\tt several modules ({\em module} ...) define
abbrev : {\em abbrev}}\\
Idem {\tt 131}, mais pour les abr\'{e}viations.

\item {\Large {\bf 133}}\ : {\tt several modules ({\em module} ...) define
sharp-macro : {\em sharp}}\\
Idem {\tt 131}, mais pour les sharp-macro.

\item {\Large {\bf 134}}\ : {\tt several modules ({\em module} ...) define
structure : {\em struct}}\\
Idem {\tt 131}, mais pour les structures.

\item {\Large {\bf 135}}\ : {\tt ({\em file} ...) 
are module files and are not
included in environment (see doc.); module : {\em module}}

Ce message concerne les fichiers d\'{e}clar\'{e}s sous la clef \|include| du
module en cours d'analyse. Seuls seront charg\'{e}s
les fichiers source ainsi pr\'{e}sents
dans la cl\'{e} \|include| qui n'apparaissent
dans aucun champ \|files| des modules du projet. Si
un fichier source est r\'{e}f\'{e}renc\'{e} dans le champ
\|include| du module en cours d'analyse et est \'{e}galement 
r\'{e}f\'{e}renc\'{e} dans le champ \|files| d'un autre module (du m\^{e}me
projet), alors le {\tt Warning} {\bf 135} sera \'{e}mis, cet \'{e}tat
de fait pouvant \^{e}tre anormal.

{\bf Correction}\,: ne pas utiliser les fichiers source d\'{e}clar\'{e}s
comme \|files| de modules sous la clef \|include|.

\item {\Large {\bf 136}}\ : 
{\tt champ FILES vide pour le module : {\em module}}\\
Ce message appara\^{\i}t au d\'{e}part d'une analyse, lorsque l'analyseur
s'apercoit qu'il n'y a pas de fichier source correspondant \`{a} ce
module.\\
{\bf Correction}\,: mettre \`{a} jour la liste des fichiers source de ce
module soit \`{a} l'aide de l'option d'analyse \|-files|, soit en
\'{e}ditant le {\tt .lm} ou le {\tt .lc} si nous sommes en mode \|update|.

\end{itemize}

D'autres avertissements ou messages d'erreur peuvent appara\^{\i}tre lors
d'une analyse. Ces autres messages sont plus classiques (erreurs
syst\`{e}me), ou plus ponctuels et contextuels. Par exemple, si on commet
une erreur dans les arguments d'analyse\,:

\begin{Longcode*}
unix% ll2lm -defmodule foo -update foo.lm
**
** sh-analyze : exclusive options : (-update -defmodule)
**
 ...
\end{Longcode*}

ou encore\,:

\begin{Longcode*}
unix% ll2lm -load test.prj -p test
**
** sh-analyze : nothing to do! one of this options must be specified : (
-defmodule -update -files -makefile -meta -delete -merge -init)
**
 ...
\end{Longcode*}

ou encore\,:

\begin{Longcode*}
unix% ll2lm -load tets.prj -p test
** loadfile : unknown file : tets.prj
\end{Longcode*}

