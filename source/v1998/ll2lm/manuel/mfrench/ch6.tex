%=======================================================================
\Chapter {6} {Extensions et utilisation avanc\'{e}e}
%=======================================================================
Nous allons aborder dans ce chap\^{\i}tre certains traits de l'analyseur
permettant de contr\^{o}ler plus finement une analyse. 
Nous verrons 
comment utiliser l'{\it Analyseur de modules} de fa\c{c}on interactive, 
nous d\'{e}couvrirons de nouvelles fonctions de gestion de projet \`{a}
utiliser avec l'analyseur interactif, 
ou encore 
comment utiliser des d\'{e}clarations pour modifier le statut de
certaines formes.

%-----------------------------------------------------------------------
\Section {Comment analyser directement dans une session Le-Lisp}
%-----------------------------------------------------------------------
Nous d\'{e}crivons ici comment analyser un module directement sous un {\tt
toplevel} \LeLisp . 

Cette possibilit\'{e} est tr\`{e}s int\'{e}ressante en phase de mise au point, pour une
analyse un peu pointue. L'acc\`{e}s au {\tt toplevel} de l'analyseur nous
permet de mieux {\it observer} l'environnement de travail, le projet
(cf paragraphe suivant), voire de tracer des fonctions.

Nous disposons d'une macro correspondant
exactement \`{a} la commande \|ll2lm| : {\tt sh-analyze}. 

%**************************************************************************
\Macro {sh-analyze} {options} {N}
%**************************************************************************
Toutes les options de \|ll2lm| sont applicables \`{a} {\tt sh-analyze}.

\begin{Code*}
unix% ll2lm
; Welcome to Analyzer System
= interactive use
? (sh-analyze -load myproject.prj -project myproject -update mymod.lm)
 ...
\end{Code*}

ou bien, directement dans une session \LeLisp\, en tapant l'expression
suivante\,: 
\begin{Code*}
? ^Lloadll2lm
= /usr/ilog/lelisp/llib/loadll2lm.ll
? (sh-analyze -load myproject.prj -project myproject -update mymod.lm)
 ...
\end{Code*}

Ces deux formes sont exactement \'{e}quivalentes \`{a}\,:
\begin{Code*}
% ll2lm -load myproject.prj -project myproject -update mymod.lm
 ...
\end{Code*}

Il est \'{e}galement possible d'analyser un module directement \`{a} l'aide de
la fonction {\tt analyze}. Celle-ci permet de lancer directement l'analyse
du module sp\'{e}cifi\'{e}, sans passer par l'\'{e}tape de mise \`{a} jour de
l'environnement (activation des projets requis, mise \`{a} jour des
paths, ...). L'interface fonctionnelle est moins riche que {\tt
sh-analyze}, puisqu'on n'a plus que l'essentiel concernant uniquement
l'analyse.

%**************************************************************************
\Function {analyze} {mod prj [output exports files imports includes]} {N}
%**************************************************************************

Cette fonction d\'{e}clenche l'analyse compl\`{e}te d'un module {\tt
mod}, appartenant au projet {\tt prj}.
Les arguments suivants de la fonction {\tt analyse} sont optionnels\,:
{\tt output} permet
de pr\'{e}ciser le nom du fichier {\tt .lm} si n\'{e}cessaire,
{\tt exports}, {\tt files}
{\tt imports} et {\tt includes} sont des listes (\'{e}ventuellement
vides) qui influencent 
le contenu du module au travers des champs respectifs \|export|,
\|files|, \|import| et \|include|. 

\NoIndent
{\tt analyze} est utilis\'{e}e par la macro {\tt sh-analyze}.

Exemple\,: soit un module {\tt mymod} du projet {\tt myproject}.
On d\'{e}clenche son analyse
apr\`{e}s mise \`{a} jour de l'environnement par\,: 

\begin{Code*}
? (analyse 'mymod 'myproject)
 ...
\end{Code*}

Une d\'{e}finition pr\'{e}cise de la variable {\tt \#:system:path} est
indispensable pour permettre \`{a} l'{\em Analyseur de Modules}
d'acc\'{e}der aux fichiers.

%-----------------------------------------------------------------------
\Section {Gestion des projets}
%-----------------------------------------------------------------------
Cette interface fonctionnelle nous
permet de d\'{e}couvrir de nouvelles formes de projets, et de mieux
g\'{e}rer un projet dans une session interactive.

%**************************************************************************
\Function {declared-rt-projects} {}{0}
%**************************************************************************

Cette fonction retourne la liste des projets qui sont d\'{e}finis dans
l'environnement d'analyse.

\begin{Code*}
? (declared-rt-projects)
= (lisp x11 windows decw mdakerne mdatools)
\end{Code*}

{\em Attention:} il faut faire la distinction entre le fait qu'un
projet soit d\'{e}fini et le fait qu'il ait \'{e}t\'{e} ``activ\'{e} {''} par
l'{\em Analyseur de Modules}. Ce dernier cas signifie que les tables
destin\'{e}es \`{a} 
l'{\em Analyseur de Modules} sont charg\'{e}es. Pour savoir si un projet
est charg\'{e}, il suffit de faire:

\begin{Code*}
? (find-rt-project 'lisp)
= rtproject:<lisp,notloaded>
? (load-rt-project 'lisp t)
.. reading file(s) #p"/nfs/work/lelisp/modana/lisp.ref"
= rtproject:<lisp,loaded>
? (load-rt-project 'lisp t)
= rtproject:<lisp,loaded>
\end{Code*}

%%**************************************************************************
%\Function {declare-rt-project} {project} {1}
%%**************************************************************************
%
%L'argument \|project| est un objet lisp de type \|rtproject|. La fonction
%\|declare-rt-project| stocke le projet \|project| comme \'{e}tant un
%projet existant, sur lequel on peut lancer \|remove-rt-project|,
%\|find-rt-project|, etc.

%**************************************************************************
\Function {find-rt-project} {name} {1}
%**************************************************************************

Cette fonction retourne la structure interne \|rtproject|, d\'{e}finie par la
macro \|define-rt-project|.

\begin{Code*}
? (find-rt-project 'foo)
= ()
? (load-rt-project 'aida ())
= rtproject:<aida,notloaded>
\end{Code*}


%**************************************************************************
\Function {load-rt-project} {name env} {2}
%**************************************************************************

Pour charger les tables destin\'{e}es \`{a} l'{\em Analyseur de Modules}
pour le projet \|name|. 
Si un projet n\'{e}cessite pas d'autres
projets (clef \|required-projects|), on prendra soin de passer une
valeur vraie (diff\'{e}rente de NIL) via le param\`{e}tre \|env|. Les tables
de ces projets seront alors charg\'{e}es \'{e}galement.

\begin{Code*}
? (setq #:crunch:verbose 2) ; pour causer...
= 2
? (load-rt-project 'mdamlook t)
.. reading file(s) : #p"/nfs/work/lelisp/modana/lisp.ref"
.. reading file(s) : #p"/nfs/work/lelisp/modana/x11.ref"
.. reading file(s) : #p"/nfs/work/aida/modana/mdasimpl.ref"
.. reading file(s) : #p"/nfs/work/aida/modana/mdakerne.ref"
...
= rtproject:<mdamlook,loaded>
\end{Code*}

Si les tables d'un projet sont d\'{e}j\`{a} charg\'{e}es, cette fonction ne les
recharge pas. Il faut utiliser la fonction \|reload-rt-project| pour
forcer leur rechargement.

%**************************************************************************
\Function {reload-rt-project} {name env} {2}
%**************************************************************************

Cette fonction agit comme la fonction \|load-rt-projet| mais force le
chargement des tables destin\'{e}es \`{a} l'{\em Analyseur de Modules} pour
le projet \|name| 
(et uniquement celui-ci). Les tables des sous-projets sp\'{e}cifi\'{e}s avec
la clef \|required-projects| ne sont pas recharg\'{e}es si elles l'ont
d\'{e}j\`{a} \'{e}t\'{e}, ou si le param\`{e}tre \|env| a une valeur fausse (NIL).

%**************************************************************************
\Function {activate-rt-project} {name} {1}
%**************************************************************************

Pour "activer" un projet avant l'analyse d'un module. L'invocation
de cette fonction est effectu\'{e}e \`{a} chaque appel de la fonction
\|analyze| ou \|sh-analyze| sur les projets concern\'{e}s par l'analyse
du module en question. Cette fonction active la fonction sp\'{e}cifi\'{e}e
avec la clef \|activate-function|.

%**************************************************************************
\Function {remove-rt-project} {name} {1}
%**************************************************************************

Pour retirer un projet qui a \'{e}t\'{e} d\'{e}fini par \|define-rt-project|.

%****************************************************************************
\Macro {define-rt-group-project} {name key1 val1 keyn valn} {N}
%****************************************************************************

Permet de d\'{e}finir un projet, comme \'{e}tant
un simple regroupement d'autres projets. Un groupe de projets est une
notion interm\'{e}diaire entre les notions de projet, et de concat\'{e}nation
de projets. L'id\'{e}e est d'offrir la possibilit\'{e} d'\'{e}tiqueter un
groupe de projets.
Seule la clef \|required-projects| (et \'{e}ventuellement
\|root-directory|) est permise dans une d\'{e}finition de groupe de
projets : elle d\'{e}crit les noms des projets composant le groupe.\\
Le nom du groupe de projets ainsi d\'{e}fini sera utilis\'{e} dans la clef
\|required-projects| d'autres projets.

%%**************************************************************************
%\Function {declare-rt-group-project} {project} {1}
%%**************************************************************************
%
%stocke le projet \|project| comme \'{e}tant un groupe de projet, existant,
%sur lequel on peut lancer \|remove-rt-group-project|,
%\|find-rt-group-project|, etc.

%**************************************************************************
\Function {find-rt-group-project} {name} {1}
%**************************************************************************

L'argument \|name| est le nom d'un groupe de projets, et
\|find-rt-group-project| retourne une structure \|rtproject|, d\'{e}finie par la
macro \|define-rt-group-project|.

%**************************************************************************
\Function {remove-rt-group-project} {name} {1}
%**************************************************************************

Pour retirer un groupe de projets de nom \|name|, qui a \'{e}t\'{e} d\'{e}fini
par \|define-rt-group-project|. 


%-----------------------------------------------------------------------
\Section {Extensions de l'analyseur}
%-----------------------------------------------------------------------

D'autres fonctions de l'{\em Analyseur de Modules} sont disponibles
pour mieux contr\^{o}ler le d\'{e}roulement d'une analyse.

%**************************************************************************
\Macro {defdynamiccall} {symbol number} {2}
%**************************************************************************

Cette forme d\'{e}clarative doit permettre \`{a} l'{/em Analyseur de Modules} de
reconnaitre des appels de fonctions calcul\'{e}s. \\
Le param\`{e}tre {\tt symbol} d\'{e}signe le nom d'une fonction capable de
lancer l'ex\'{e}cution d'une autre fonction calcul\'{e}e. 
Le param\`{e}tre {\tt number} d\'{e}signe l'argument dans la
liste des param\`{e}tres de cette fonction qui contient la fonction
calcul\'{e}e. L'analyseur ne sait exploiter cela que lorsque les
fonctions calcul\'{e}es apparaissent sous la forme {\tt 'function}. \\
Dans le cas classique de {\tt funcall} par exemple, la fonction
calcul\'{e}e se trouve dans le premier argument de {\tt funcall}\,:
\begin{Code*}
(funcall 'foo 1 2) 
\end{Code*}
Pour que l'analyseur reconnaisse que la fonction {\tt foo} est
appel\'{e}e par ce code, et d\'{e}clenche \'{e}ventuellement l'importation du
module qui la d\'{e}fini, on \'{e}crira:
\begin{Code*}
;; le 1er argument de FUNCALL est une fct appelee dynamiquement
(defdynamiccall funcall 1)
;; le 1er argument de APPLY est une fct appelee dynamiquement
(defdynamiccall apply 1)
;; le 1er argument de mapoblist est une fct appelee dynamiquement
(defdynamiccall mapoblist 1)
\end{Code*}
Un autre exemple pour \Aida:
\begin{Code*}
;; le 2eme argument de SET-ACTION est une fct appelee dynamiquement
(defdynamiccall set-action 2)
\end{Code*}
Ces exemples sont les d\'{e}clarations pr\'{e}sentes par d\'{e}faut dans les
projets respectifs \LeLisp\ et \Aida. Les {\tt defdynamiccall} de
l'utilisateur pourront \^{e}tre install\'{e}es dans le fichier projet ({\tt
project.prj}) du projet concern\'{e}.\\
L'effet de {\tt defdynamiccall} est conditionn\'{e}e par l'option d'analyse
\|-dynamic| qui doit \^{e}tre positionn\'{e}e lors de l'analyse pour que
{\tt defdynamiccall} soit op\'{e}rationnel. 

%**************************************************************************
\Macro {defdefiner} {symbol [function]} {1-2}
%**************************************************************************

Cette forme d\'{e}clarative permet \`{a} l'{\em Analyseur de Modules} de
reconnaitre de nouveaux d\'{e}finisseurs d'entit\'{e}s qui ne sont pas des
fonctions, mais qui doivent d\'{e}clencher tout de m\^{e}me l'importation du
module qui d\'{e}fini cette entit\'{e} (pour l'environnement
de compilation. cf CPENV). \\
Le param\`{e}tre {\tt symbol} est le nom d'un tel d\'{e}finisseur.
Le param\`{e}tre optionnel {\tt function} est une fonction \`{a} 1 argument.
Cette fonction s'applique au premier param\`{e}tre du d\'{e}finisseur {\tt
symbol}, et ram\`{e}ne le nom pertinent \`{a} retrouver dans l'environnement de
compilation.  \\
L'exemple le plus frappant est celui de {\tt defabbrev} qui ne d\'{e}finit
pas une fonction, mais qui est souvent emboit\'{e} dans un {\tt
eval-when} afin d'\^{e}tre connu au moment de la compilation. Tous les
modules utilisant cette abr\'{e}viation, doivent importer le module qui
la d\'{e}finit.
Pour que l'analyseur reconnaisse {\tt defabbrev} comme un tel
d\'{e}finisseur, on \'{e}crira:
\begin{Code*}
(defdefiner defabbrev)
\end{Code*}
Un autre exemple nous est fourni par {\tt defsharp}. Le nom de
fonction associ\'{e} au {\em sharp-character} est toujours calcul\'{e} \`{a}
partir du package \|#:sys-package:sharp|. On \'{e}crira\,:
\begin{Code*}
(definer defsharp (lambda(x)(symbol #:sys-package:sharp x)))
\end{Code*}
Les deux exemples ci-dessus sont d\'{e}clar\'{e}s en standard dans l'image
m\'{e}moire de l'{\em Analyseur de modules}. \\
Les d\'{e}finisseurs {\tt dmc} et {\tt dms} sont d\'{e}finis dans le projet
\LeLisp. Les {\tt defdefiner} de
l'utilisateur pourront \^{e}tre install\'{e}es dans le fichier projet ({\tt
<project>.prj}) du projet concern\'{e}. \\
Classiquement, si de telles formes sont inconnues, lors de l'analyse,
une des erreurs {\tt errudf, errnotanabbrev, ...} sera d\'{e}clench\'{e}e.
L'utilisateur peut toutefois d\'{e}clencher lui-m\^{e}me un autre type
d'erreur moins sp\'{e}cifique\,: {\tt ERRUNK} : "fct : I don't know :
arg". Lorsque  cette erreur est d\'{e}clench\'{e}e lors de l'analyse,
l'analyseur importe 
le module d\'{e}finissant {\tt arg}. Ce fonctionnement impose que la
forme d\'{e}finissant {\tt arg} soit d\'{e}clar\'{e}e par {\tt
defdefiner}.\\ 
Les symboles d\'{e}clar\'{e}s par {\tt defdefiner} influencent directement
l'analyse d'un module utilisant un tel symbole en {\tt toplevel form},
puisqu'il sera alors ajout\'{e} dans le fichier des r\'{e}f\'{e}rences
(\|<project>.ref|), et par effet de bord, ce module pourra \^{e}tre import\'{e}
lors de l'analyse d'un autre module utilisant ce symbole. De m\^{e}me,
ces symboles seront reconnus dans les champs \|CPENV| des modules lors
de la fabrication d'un fichier de r\'{e}f\'{e}rences \`{a} partir des modules
(option d'analyse \|-build|).

%**************************************************************************
\Function {func-from} {symbol} {1}
%**************************************************************************

Cette fonction prend un nom de fonction \LeLisp\ ou \Aida\ et rend une
liste contenant le nom du module l'exportant.  Cette liste contient
g\'{e}n\'{e}ralement un seul \'{e}l\'{e}ment, sauf dans les cas o\`{u} plusieurs
modules exportent la m\^{e}me fonction.

%**************************************************************************
\Function {abbrev-from} {symbol} {1}
%**************************************************************************

Cette fonction prend un symbole d\'{e}signant une abr\'{e}viation et rend
une liste contenant le (ou les) module(s) d\'{e}finissant cette
abr\'{e}viation.
