\documentstyle{ilogmanuel}


\Begin
\Title {Diffusion Le-Lisp}
\SuperTitle {Le-Lisp  de  l'INRIA  version 15.25 \\
sous-version 2}

\Author {Septembre 1992}

                 
                            
\chapter {Version "15.25.2"}
Voici les corrections et les extensions de la nouvelle diffusion de
\LeLisp\ version 15.25 dat\'{e}e du 22 septembre 1992.

Il s'agit de la deuxi\`{e}me diffusion de maintenance de la version
15.25.  Cette diffusion comprend uniquement des 
contournements et corrections effectu\'{e}s sur les sp\'{e}cificit\'{e}s des
portages sans toucher au noyau commun.
Comme d\'{e}cid\'{e} au Club des Porteurs, aucune expansion LLM3 n'est n\'{e}cessaire.


La bande de diffusion est explicit\'{e}e ci-dessous.

Les catalogues avec tous les fichiers (pour coh\'{e}rence):

\begin {itemize}
\item   benchmarks:
\item   bignum:
\item   common: 
\item   info:
\item   llib:           
\item   llmod:
\item   llobj:
\item   lltest:         
\item   llub:   
\item   manl:
\item   sun:
\item   virbitmap:
\item   virtty: 
\end {itemize}

Les catalogues non inclus:

\begin {itemize}
\item   llm3:   aucun fichier
\item   manual: aucun fichier
\end {itemize}


Les diff\'{e}rences notables sont les suivantes:

\begin {itemize}

\item Dans les r\'{e}pertoires {\it machine}:
\begin {itemize}
\item {\bf Makefile}\\
Les fichiers {\tt Makefile} ont \'{e}t\'{e} modifi\'{e} sur un plan
essentiellement syntaxique. Ainsi les entr\'{e}es g\'{e}n\'{e}riques
fonctionnent maintenant sur toutes les plateformes.

\item {\bf config}\\
Am\'{e}lioration de la gestion du r\'{e}pertoire des cores lisp \`{a} la
fabrication des images m\'{e}moire Le-Lisp.
Dor\'{e}navant, la variable {\tt \#:system:core-directory} est mise
\`{a} jour automatiquement \`{a} partir de l'option fournie \`{a} {\tt config}.

\item {\bf complice}\\
Introduction de deux nouvelles options: {\tt -lldir <path>} pour
imposer un r\'{e}pertoire de r\'{e}f\'{e}rence diff\'{e}rent (r\^{o}le pricipal: o\`{u}
aller chercher le binaire?); {\tt -cmplc <cmd>}
pour imposer une autre commande que {\tt cmplc++} qui est le core lisp
utilis\'{e} en standard pour compiler les modules Le-Lisp. Le manuel Unix
a \'{e}t\'{e} revu en cons\'{e}quence. 

\item {\bf cmplc++}\\
le core utilis\'{e} par {\tt complice} en standard, int\`{e}gre maintenant
le modules {\tt hash} et {\tt format}, tr\`{e}s couramment utilis\'{e}s.
Cette modification intervient dans le fichier {\tt Cmplcconf.ll}.
\end{itemize}

\item Concernant les modules lisp:
\begin{itemize}
\item {\bf edlin}\\
Optimisation de l'effacement des caract\`{e}res.

\item {\bf evloop}\\
Prise en compte du {\it foreground} dans la boucle d'\'{e}v\'{e}nements.

\item {\bf path}\\
Introduction de la gestion des pathnames DOS et OS2.

\item {\bf product}\\
Une d\'{e}finition de {\tt macro-open} manquait, et faisait que ce module
ne fonctionnait pas en compil\'{e}. De plus ce module est pass\'{e} de {\tt
llub} en {\tt llib}.

\item {\bf cpmac}\\
Le fait de recharger le module {\bf cpmac}
introduisait des erreurs de compilation, dues au {\tt CPENV} qui
\'{e}tait \'{e}cras\'{e}. C'est r\'{e}par\'{e}.

\item {\bf complice}\\
Lorsqu'un module import\'{e} {\tt M}, exporte une macro qui n'est pas
dans un EVAL-WHEN, alors ce module doit \^{e}tre charg\'{e} (LOADMODULE)
dans l'environnement de la compilation du module qui importe {\tt M}.
Ce chargement \'{e}tait jusqu'alors fait au moyen de la forme: {\tt
(loadmodule M t)}. le {\tt T} en second argument de LOADMODULE
imposait le {\bf re}chargement de tout l'arbre de d\'{e}pendance de {\tt M}. On
utilise maintenant la forme: {\tt (loadmodule M ())} afin d'all\'{e}ger
l'environnement de compilation.

\item {\bf Bitmap Virtuel}\\
Introduction de la gestion des \'{e}crans 24 plans.
Introduction des {\it stipples}. Ainsi que quelques corrections
mineures. Tous les fichiers {\tt .c} sont concern\'{e}s, seuls quelques
fichiers {\tt .ll} ont \'{e}t\'{e} modifi\'{e}s ({\tt x11init, x11graph,
x11window, x11bitmap, x11draw, x11\_def}).
\end{itemize}

\item Concernant les sources C:
\begin{itemize}
\item Les probl\`{e}mes de {\tt PAGESIZE} qui rendaient parfois les
cores lisp incompatibles sur une m\^{e}me machine, sont r\'{e}solus. (Cela ne
concernait, \`{a} priori, que Sun4 et Apollo).

\item L'introduction du portage {\it MSDOS} a introduit de nombreuses
modifications du code C. Toutefois, ces modifications n'affectent pas 
le code Unix, puisque ce code a \'{e}t\'{e} ajout\'{e} sous
la condition: {\tt \#ifdef LLDOS}.

\item Introduction de code C adapt\'{e} au format {\bf ELF} des fichiers objets
(remplacant le format {\bf COFF} sur M88K, et SVR4 en particulier). Cela
conserne principalement la fonctionnalit\'{e} {\tt GETGLOBAL}.

\item Le fichier {\tt C/Machine.h} de chaque portage s'est vu
l\'{e}g\`{e}rement modifi\'{e}, suite \`{a} l'introduction de nouvelles macros
(LLGETCWD,LLBMEMALIGN).
\end{itemize}

\item Concernant la distribution:
\begin{itemize}
\item {\bf fichiers distribu\'{e}s}\\
Les fichiers {\tt llib/RCSDIFF} et {\tt common/RCSDIFF}, contenant les
diff\'{e}rences depuis la derni\`{e}re version, sont
maintenant pr\'{e}sents dans {\tt USERFILES} au lieu de {\tt PORTFILES}.
Ceci a pour cons\'{e}quence de distribuer ces deux fichiers \`{a} l'ensemble
des clients, et non plus seulement aux porteurs. Les clients pourront
d\`{e}s lors observer eux-m\^{e}mes l'ensemble des modifications r\'{e}alis\'{e}es
entre deux distributions.
\end {itemize}

\item Concernant plus sp\'{e}cifiquement les clients Ilog:
\begin{itemize}
\item {\bf plombage}\\
Le plombage introduit par Ilog met en jeu un fichier(ILOGACCESS), 
qui n'\'{e}tait pas toujours referm\'{e}. C'est r\'{e}par\'{e}.\\
Introduction de la notion de {\bf clef infinie}. \\
Possibilit\'{e} de d\'{e}signer le fichier de clefs au travers d'une
variable d'environnement Unix. 

\item {\bf RS6000}\\
Le loader du RS6000 a vu l'optimisation de la gestion du cache
m\'{e}moire. Le passage des arguments de Lisp \`{a} C \'{e}tait d\'{e}fectueux
dans certains cas d'utilisation de LISPCALL. Tout cela est r\'{e}par\'{e}.

\item {\bf MAGNUM}\\
Une erreur d'alignement de pile provoquait des erreurs de
transmission de flottants entre C et Le-Lisp. C'est r\'{e}par\'{e}.

\end {itemize}
\end{itemize}

\chapter{Corrections d'anomalies}

Les changements et corrections de \LeLisp\ sont g\'{e}r\'{e}s avec des \|RATs|
(Requ\^{e}tes d'Action Technique).  Nous n'avons fourni que les sujets
des RATs par souci de place, mais vous pouvez obtenir le texte complet
de chaque RAT sur simple demande.

\begin{tabbing}
\= xxxxxxxxxxxxxxxxxxxxxxxxxxxxxx\= xxxxxxxxxxxxxxxxxxxxxxxxxxxxxx\= xxxxxxxxxxxxxxxxxxxxxxxxxxxxxx \= \kill
\verb_Rats for le-lisp selected by status-fixed? as of 22/09/1992._\\ \verb_ _\\
\hspace{-5em}rat 240 \> area: {\it  kernel
\ }\> aspect: {\it  compiler
\ }\> origin: {\it  parquier
\ }\\ 
\> status: {\it  fixed
\ }\> type: {\it  bug
\ }\\ 
\verb_select compiles into nested if
_\\ 
\verb_ _\\ 
\hspace{-5em}rat 1084 \> area: {\it  user-contributed-software
\ }\> origin: {\it  meffre@ilog (Marie Francoise Meffre)
\ }\\ 
\> status: {\it  fixed
\ }\> type: {\it  bug
\ }\\ 
\verb_detail dans un message d'erreur de LLUB/LOADFILE
_\\ 
\verb_ _\\ 
\hspace{-5em}rat 1142 \> area: {\it  language
\ }\> aspect: {\it  interpreter
\ }\> origin: {\it  E. Canton [Ilog]
\ }\\ 
\> status: {\it  fixed
\ }\> type: {\it  bug
\ }\\ 
\verb_MAP-EXPAND-PATHNAME traite mal certains cas combines avec DIRECTORYP
_\\ 
\verb_ _\\ 
\hspace{-5em}rat 1248 \> area: {\it  user-contributed-software
\ }\> origin: {\it  kuczynsk
\ }\\ 
\> status: {\it  fixed
\ }\> type: {\it  bug
\ }\\ 
\verb_liaison dynamyque dnageureuse dans le module PRODUCT
_\\ 
\verb_ _\\ 
\hspace{-5em}rat 1371 \> area: {\it  ports
\ }\> aspect: {\it  tools
\ }\> origin: {\it  J. Chailloux [Ilog]
\ }\\ 
\> status: {\it  fixed
\ }\> type: {\it  bug
\ }\\ 
\verb_le plombeur semble mal fermer le fichier ILOGACCESS
_\\ 
\verb_ _\\ 
\hspace{-5em}rat 1599 \> area: {\it  run-time
\ }\> aspect: {\it  o/s-interface
\ }\> origin: {\it  kuczynsk
\ }\\ 
\> status: {\it  fixed
\ }\> type: {\it  bug
\ }\\ 
\verb_le PLOMBEUR oublie de fermer le fichier ILOGACCESS
_\\ 
\verb_ _\\ 
\hspace{-5em}rat 1608 \> area: {\it  compiler
\ }\> aspect: {\it  complice
\ }\> origin: {\it  le
\ }\\ 
\> status: {\it  fixed
\ }\> type: {\it  bug
\ }\\ 
\verb_cpenv
_\\ 
\verb_ _\\ 
\hspace{-5em}rat 1651 \> area: {\it  compiler
\ }\> aspect: {\it  complice
\ }\> origin: {\it  le@ilog
\ }\\ 
\> status: {\it  fixed
\ }\> type: {\it  bug
\ }\\ 
\verb_la commande shell COMPLICE devrait ouvrir ses variables SHELL
_\\ 
\verb_ _\\ 
\hspace{-5em}rat 1661 \> area: {\it  i/o
\ }\> aspect: {\it  virtty
\ }\> origin: {\it  eisenman@ilog
\ }\\ 
\> status: {\it  fixed
\ }\> type: {\it  bug
\ }\\ 
\verb_certains messages qui font peu industriel...
_\\ 
\verb_ _\\ 
\end{tabbing}

\End
