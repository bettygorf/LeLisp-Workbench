\documentstyle[11pt,fr]{iarticle}

\begin{document}
\frenchspacing
\setlength{\parindent}{0in}
\setlength{\parskip}{3ex}
\raggedbottom

\Title {Diffusion Le-Lisp}
\SuperTitle {Le-Lisp de l'INRIA version 15.26}

\Author {d\'{e}cembre 1993}
\bigskip

\Section {Version "15.26"}
Voici les corrections et les extensions de la nouvelle diffusion de
\LeLisp\ version 15.26 dat\'{e}e du 1 d\'{e}cembre 1993.

Une fois par an, l'INRIA et ILOG livrent aux porteurs une nouvelle
version de Le-Lisp comprenant des am\'{e}liorations, extensions et
corrections.  Seule cette diffusion annuelle contient des changements
\`{a} la souche de Le-Lisp \'{e}crite en LLM3, et au manuel de l'utilisateur.

\Section {Contenu de cette diffusion}

La bande de diffusion contient les 4 fichiers suivant au format {\tt tar}
compress\'{e} :
\begin{itemize}
\item \|ll_port.tar.Z| : l'ensemble des fichiers communs \`{a} tous
les portages, uniquement destin\'{e}s aux porteurs,
\item \|ll_user.tar.Z| : l'ensemble des fichiers communs \`{a} tous
les portages, distribu\'{e}s aux utilisateurs,
\item \|sun_port.tar.Z| : l'ensemble des fichiers sp\'{e}cifiques au
portage {\tt SUN 3}, uniquement destin\'{e}s aux porteurs,
\item \|sun_user.tar.Z| : l'ensemble des fichiers sp\'{e}cifiques au
portage {\tt SUN 3}, distribu\'{e}s aux utilisateurs.
\end{itemize}

La bande de diffusion se d\'{e}compose selon les r\'{e}pertoires suivants:

\begin {itemize}
\item   68k : l'expanseur commun \`{a} tous les 680x0;
\item   benchmarks : pour r\'{e}aliser des mesures de performances des syst\`{e}mes Le-Lisp;
\item   benchmarks/Results : les r\'{e}sultats de ces mesures de performances;
\item   bignum : le module BIGNUM de DEC/INRIA;
\item   bignum/l : l'interface des bignum avec Le-Lisp;
\item   common : les fichiers C communs \`{a} tous les portages utilisant C;
\item   info : ce fichier;
\item   ll2lm : l'analyseur de modules;
\item   llib : la librairie Le-Lisp;
\item   llm3 : le noyau de l'interpr\`{e}te Le-Lisp;
\item   llmod : la description modulaire de tous les fichiers des
librairies Le-Lisp;
\item   llobj : le r\'{e}sultat de la compilation modulaire de tous les
fichiers des librairies Le-Lisp;
\item   lltest : les jeux de tests pour tester, verifier, recetter Le-Lisp;
\item   lltool : quelques outils utiles aux porteurs;
\item   llub : une librairie suppl\'{e}mentaire, remplie par les
utilisateurs, non maintenue;
\item   manl : un manuel UNIX des principales commandes pour Le-Lisp;
\item   manuel : le manuel de r\'{e}f\'{e}rence en francais, au format {\tt
tcomp};
\item   modana : les fichiers engendr\'{e}s par l'analyseur de modules;
\item   sun : le syst\`{e}me Sun3;
\item   sun/C : les fichiers C sp\'{e}cifiques \`{a} Sun3;
\item   sun/conf : les configurateurs d'images Le-Lisp sp\'{e}cifiques \`{a} Sun3;
\item   sun/recette : les commandes permettant de lancer les recettes
sur Sun3;
\item   virbitmap : les incarnations de BV et leurs chargeurs;
\item   virbitmap/bvtty : l'incarnation du BV sur un {\tt tty}
alphanum\'{e}rique simple;
\item   virbitmap/X11 : l'incarnation du BV sur X11;
\item   virtty : les descriptions de {\tt tty} d\'{e}j\`{a} rencontr\'{e}s par
le-Lisp. 
\end {itemize}

Le contenu exact de la bande de diffusion est explicit\'{e}e dans les
fichiers {\tt LLUSERFILES}, {\tt LLPORTFILES} et {\tt LLPORTMJFILES}
ainsi que dans les fichiers {\tt UserFILES} et {\tt PortFILES} dans
les r\'{e}pertoires des machines.

\Section{Features \`{a} prendre en compte pour la 15.26}
Nous allons \'{e}num\'{e}rer les points int\'{e}gr\'{e}s en 15.26 :

\SSection{Portages natifs: suppression de GELL}
Il est g\'{e}n\'{e}ralement admis qu'un portage natif est plus performant
dans tous les sens du terme qu'un portage GELL. Il est donc d\'{e}cid\'{e}
d'abandonner les portages GELL.

Les probl\`{e}mes soulev\'{e}s par cette d\'{e}cision sont\ : 

Que faire des clients ne connaissant aujourd'hui que les portages
GELL\ ? \\
Il existe en effet des clients n'ayant jamais utilis\'{e} de portage
natif. Ces clients ont bien souvent souffert de l'installation GELL,
et seraient m\'{e}contents d'avoir \`{a} se refabriquer une cha\^{\i}ne de
production, m\^{e}me si celle-ci est plus simple. En g\'{e}n\'{e}ral ces clients
utilisent le GRT pour fabriquer leurs applis. Il revient \`{a} Ilog de
fournir l'effort n\'{e}cessaire pour que l'utilisation du GRT sous
LL15.25.4 en natif, puis en 15.26 ne soit pas incompatible avec l'utilisation
actuelle qu'en font les clients sous GELL. Cette t\^{a}che sera
\'{e}galement facilit\'{e}e par la r\'{e}daction d'une documentation de passage
de GELL vers NATIF sur HP9700.

\SSection{Cruncher et G\'{e}n\'{e}rateur de RunTime}
L'id\'{e}e consiste \`{a} harmoniser la cha\^{\i}ne de production de code, pour
l'ensemble des produits de la ligne LL. Tous les produits devront
utiliser le cruncher de LL ({/tt ll2lm}) et le GRT en cons\'{e}quence.

Les probl\`{e}mes soulev\'{e}s par cette d\'{e}cision sont: \\

De quoi est compos\'{e} le GRT et que doit passer dans LL\ ? \\
On consid\'{e}rera d'une part le {\it cruncher} qui fabrique des modules
({\tt .lm}) et un {\it Makefile} permettant de {\it complicer} ces
modules, d'autre part le g\'{e}n\'{e}rateur de runtime qui permet d'avoir
une chaine de production engendrant une application (un core LL) de
fa\c{c}on automatique. Une troisi\`{e}me partie due \`{a} GELL (gestion des .o,
fabrication des binaires, etc) devient obsol\`{e}te avec la disparition de
GELL. Seule la 1ere partie : {\bf ll2lm}, passe dans \LeLisp\ .
La partie, {\it g\'{e}n\'{e}rateur de runtime} reste un module \Aida\ .

\SSection{plombage de la 15.26}
Un nouveau type de plombeur, d\'{e}velopp\'{e} en interne \`{a} Ilog, est
install\'{e} dans la 15.26 distribu\'{e}e par Ilog. L'ancien plombeur,
attach\'{e} au hostid des 
machines, et ne permettant que des licences type machine, est
remplac\'{e} par un plombeur de type serveur de jetons, sur un r\'{e}seau de
machines. La possibilit\'{e} de plomber sur le hostid des machines avec le
nouveau plombeur est \'{e}galement possible afin de fonctionner sur les machines
ne disposant pas de r\'{e}seau (msdos, standalone,...). \\
Au niveau du code, l'interface d'utilisation a \'{e}t\'{e} modifi\'{e}e puisque
la fonction C {\tt tryaccess(int)} a \'{e}t\'{e} remplac\'{e}e par {\tt
lmaccess(char *)}, et la fonction {\tt lmalive(char *)} a \'{e}t\'{e}
ajout\'{e}e.
Le code \LeLisp\ change en cons\'{e}quence dans {\tt llib/llpboot.ll}: \\
\BeginLL
(defextern lmaccess(string fix))
(defextern lmalive(string))
\EndLL

\Section{liste des principales corrections}

\begin {itemize}

\item Interventions sur LLIB :\\
\begin{enumerate}
\item
Reprise de tous les commentaires en anglais.

\item
Encore des bugs sur le module PATHNAME (cf rats) ainsi que qq
optimisations \`{a} faire.

\item 
Optimisation du module FORMAT.

\item
Quelques rats concernant LLIB (voir section RATS).
\end{enumerate}

\item Intervention sur les codes C et assembleur :\\

\begin {itemize}

\item
Utilisation de {\bf ANSI C}  et de {\bf POSIX}.
Cela devient de + en + indispensable avec des syst\`{e}mes comme Solaris
et SVR4 plus g\'{e}n\'{e}ralement. Mais aussi parce que de + en + de clients
utilisent ANSI C (intervention dans nos .h au minimun, recompilation
possible dans leur environnement).

\item
Am\'{e}lioration de la gestion des d\'{e}tections d'anomalies dans les clefs
d'acc\`{e}s du plombage. En particulier, le {\it warning} affich\'{e} en pr\'{e}vention
d'expiration de date n'apparait plus que pour le produit test\'{e} et lui
seul. De plus le message affich\'{e} est plus explicite (num\'{e}ro de clef
et nom de produit sont indiqu\'{e}s).

\item
Affichage du {\it hostid} et de sa {\it checksum}.

\item 
Introduction de la gestion des r\'{e}pertoires en \LeLisp\ , au travers
de fonctionnalit\'{e}es telles que {\tt create-directory}, {\tt
remove-directory}.

\item 
Introduction d'un processus fant\^{o}me \`{a} la {\tt lelispgo} permettant
de r\'{e}aliser des appels syst\`{e}mes sans exploser en encombrement
m\'{e}moire, sur les machines ne disposant pas de VFORK. Ce processus
est lanc\'{e} d\`{e}s le d\'{e}marrage de \LeLisp\ , et c'est lui qui est
utiliser pour {\it forker} lors d'appels syst\`{e}me tels que
COMLINE. Cette possibilit\'{e}e d\'{e}pend du flag de compilation {\tt
LLGHOSTPROC} et n'est pas op\'{e}rationnelle sur tous les portages.

\end {itemize}

\item Interventions sur le code LLM3:

\begin{itemize}

\item Optimisations diverses d'algorithmes LLM3.

\item Am\'{e}lioration de la gestion des entiers en vue d'un passage
ult\'{e}rieur \'{e}ventuel en entiers d'un format sup\'{e}rieur \`{a} 16bits.
Pour cela, ajoput des fonctions (MOST-POSITIVE-FIXNUM),
(MOST-NEGATIVE-FIXNUM) et (MINUS-0-FIXNUM).

\item Correction du SELECTQ en interpr\'{e}t\'{e} en cas de listes EQ.

\item Renomage de pas mal de symboles internes LLM3 afin d'\'{e}viter
toute collision de nommage avec des symboles externes (C, ...).

\item Harmonisation des messages d'erreur sur les nombres d'arguments
entre fonctions compil\'{e}es et fonctions interpr\'{e}t\'{e}es.

\item LHOBLIST, EXPLODECH, IMPLODECH passent de startup.ll \`{a}
fntstd.llm3. 

\item Ajout d'un {\it finalizer} permettant de lancer une fonction \`{a}
chaque fois que le GC r\'{e}cup\`{e}re un vecteur LL.

\item Modification de l'affichage des zones par  (GC T) afin d'\^{e}tre
plus pr\'{e}cis. A noter qu'on reste compatible avec les codes exploitant
ce r\'{e}sultat.

\item Introduction de la macro LLM3 HOVNI qui lance un bout de code C
permettant d'afficher l'\'{e}tat de la m\'{e}moire.

\item Introduction de l'itsoft AT-END pour ``faire des choses{''} avant
de sortir avec la fonction (END).

\item Ajout des fonctions (CREATE-DIRECTORY {\tt path}) et
(DELETE-DIRECTORY {\tt path}).

\item Ajout de la fonction (SHOW-STACK) qui montre le nombre d'octets
utilis\'{e}s sur la pile LL pour l'ex\'{e}cution d'une forme donn\'{e}e.

\item Ajout de la variable {\tt \#:system:line-length-max} d\'{e}signant
la longueur maximale d'un buffer d'impression. Cette valeur est
syst\`{e}me d\'{e}pendante. On utilisera principalement {\tt
\#:system:line-length-max} avec {\tt rmargin}.

\end{itemize}
\item Corrections propres \`{a} COMPLICE:
\begin{itemize}
\item [10/03/92]
        Corrections pour J. Chambon: Pbs dans interd\'{e}pendance de modules.
         Ordre de chargement des fichiers non respect\'{e}.
         La fonction :make-single-module1 fait des effets de bord sur
          le cache :list-of-loaded-file au troisi\`{e}me appel (i.e. la
          copie n'\'{e}tait faite qu'une seule fois).
         cpenv \&co ne sont pas respect\'{e}s (revoir list-of-loaded-file)...

\item [23/03/92]
        On avait oubli\'{e} le warning 5 pour l'appel des SUBR0s....

\item [21/04/92]
        Correction pour (memq x '()).
        Pas de registres qui contiennent ce ?? bizarre.

\item [05/07/93]
        Correction pour (selectq E ... () ...)

\item [06/07/93]
        Enl\`{e}ve les error et warnings inutiles.
        (loadmodule ? NIL) pour les macros.
        On fait une copie des instructions LAP dans le cas du precompile.
        Double \'{e}valuation dans cpmac.ll pour le catcherror.
        macro-expand-error vir\'{e} de cpmac.ll.
        
\item [02/12/93]
        Correction pour la fausse tail-rec des nsubrs via une fermeture:
          (de f1 (a . b) (mon-mapc (lambda (x) (f1 x)) a))
        On rajoute une globale :funarg? qui sp\'{e}cifie si on compile une
        fermeture..

\item {\tt rmargin}
        Par d\'{e}faut, {\tt rmargin} prend la valeur maximale autoris\'{e}e sur
le syst\`{e}me h\^{o}te. 
\item Nouvelles options de la commande {\tt complice}:
\begin{itemize}
\item option {\bf -h}
        Pour brancher la m\'{e}moire cache des fichiers de description
des modules: si cette option est utilis\'{e}e, chaque fichier de
description modulaire ne sera charg\'{e} qu'une seule fois par sessions
et stock\'{e} en 
m\'{e}moire. Les autres consultations se feront sur les donn\'{e}es
stock\'{e}es.
\item option {\bf -callext}
        Dor\'{e}navant, {\tt complice} n'ex\'{e}cute plus les {\tt
callextern} par d\'{e}faut. Cela permet de compiler les modules contenant
des {\tt defextern} sans se soucier de la pr\'{e}sence du code C
correspondant. Cette d\'{e}cision peut \^{e}tre d\'{e}connect\'{e}e gr\^{a}ce \`{a}
l'option {\tt -callext}. Si cette option est utilis\'{e}e, {\tt complice}
ex\'{e}cutera les {\tt defextern} et autres {\tt getglobal} normalement.

\end{itemize}
\end{itemize}

\item Intervention sur la technologie de portage:

\begin {itemize}

\item 
Introduction de nouvelles recettes, en particulier pour tester le
{\it Makefile} et ses entr\'{e}es g\'{e}n\'{e}riques.

\end {itemize}

\end {itemize}

\newpage
\tableofcontents

\end{document}




